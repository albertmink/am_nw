\documentclass[9pt,german]{beamer}%
\usepackage{master/templates/beamerthemeKIT}
\usepackage{master/templates/exercises}

%\documentclass[9pt,trans]{beamer}%

%%%%%%%%%%%%%%%%%%%%%%%%%%%%%%%%%%%%%%%%%%%%%%%%%%%%%%%%%%%%%%%%%%%%%%%%
% Packages
%%%%%%%%%%%%%%%%%%%%%%%%%%%%%%%%%%%%%%%%%%%%%%%%%%%%%%%%%%%%%%%%%%%%%%%%
%\usepackage{beamerthemeKIT}
\usepackage[utf8]{inputenc}
% \usepackage[german]{babel}
\usepackage{helvet}
\usepackage[T1]{fontenc}
\usepackage{amsmath, amsthm, amssymb}
\usepackage{graphicx}
\usepackage{listings}
\usepackage{hyperref}
\usepackage{KITcolors}
%%%%%%%%%%%%%%%%%%%%%%%%%%%%%%%%%%%%%%%%%%%%%%%%%%%%%%%%%%%%%%%%%%%%%%%%
% Layout
%%%%%%%%%%%%%%%%%%%%%%%%%%%%%%%%%%%%%%%%%%%%%%%%%%%%%%%%%%%%%%%%%%%%%%%%
\renewcommand{\rmdefault}{phv}
\newlength{\movetitlegrafics}
\setlength{\movetitlegrafics}{0.0\paperwidth}
%%%%%%%%%%%%%%%%%%%%%%%%%%%%%%%%%%%%%%%%%%%%%%%%%%%%%%%%%%%%%%%%%%%%%%%%
% Definitions
%%%%%%%%%%%%%%%%%%%%%%%%%%%%%%%%%%%%%%%%%%%%%%%%%%%%%%%%%%%%%%%%%%%%%%%%
%%% Counter
\newcounter{ctlit}%
\setcounter{ctlit}{1}%
\newcounter{kap}

%%% Textfarben
\newcommand\BLUE[1]{\textcolor{KITblue}{#1}}%
\newcommand\BLACK[1]{\textcolor{KITblack}{#1}}%
\newcommand\GREEN[1]{\textcolor{KITgreen}{#1}}%
\newcommand\LBLUE[1]{\textcolor{KITblue15}{#1}}%
\newcommand\RED[1]{\textcolor{KITred}{#1}}%
%%% Layout
\def\cmt#1{\BLUE{#1}}%
\def\code#1{\texttt{#1}}%
\def\codeblock#1{\GREEN{\texttt{#1}}}%
\def\ftext#1{\textbf{{#1}}}%
\def\headline#1{\large\textbf{\GREEN{#1}}}%
\def\KITitem{\KITBullet[2mm]\ }%
\def\q{\quad}%
\def\qq{\qquad}%
\def\subtitle#1{{\small #1}}%
\def\tab{\hspace*{0.5cm}}%
%%% Standards
%\input{D:/TeX/formate/mystyle/defs}%
%\input{/home/ae01/tex/formate/mystyle/defs}%
%%% Text
%%% Symbols
\def\bs{$\backslash$}%
\def\hpm{\hphantom{$-$}}%
\def\meps{\epsilon}%
%%% code environment
\lstset{
  showstringspaces=false,
  numbers=none,
  keywordstyle=\color{KITgreen},  % coloring and formatting of keywords as public, class, import 
  commentstyle=\color{KITblue}\small\ttfamily, % Color of comments
  stringstyle=\color{KITred},
  breaklines=true
}

\lstdefinestyle{JAVA}
{
  language=Java,
  basicstyle=\ttfamily, % defines text formatting
  frame=tb  % print top and bottom lines, frame=single, L
}

\lstdefinestyle{JAVAsmall}
{
  language=Java,
  basicstyle=\small\ttfamily, % defines text formatting
  frame=tb  % print top and bottom lines, frame=single, L
}

\lstdefinestyle{JAVAlines}
{
  language=Java,
  numbers=left,
  numberstyle=\color{KITblack50}\ttfamily,
  numbersep=4pt, % distance between numbers and code1_1
  xleftmargin=4pt, % distance from frame to listing
  xrightmargin=4pt, % distance from frame to listing
  basicstyle=\ttfamily, % defines text formatting
  keywordstyle=\color{KITgreen},  % coloring and formatting of keywords as public, class, import 
  commentstyle=\color{KITblue}\ttfamily, % Color of comments
  frame=tb  % print top and bottom lines, frame=single, L
}

\lstdefinestyle{JAVAsmalllines}
{
  language=Java,
  numbers=left,
  numberstyle=\color{KITblack50}\ttfamily,
  numbersep=4pt, % distance between numbers and code1_1
  xleftmargin=4pt, % distance from frame to listing
  xrightmargin=4pt, % distance from frame to listing
  basicstyle=\small\ttfamily, % defines text formatting
  keywordstyle=\color{KITgreen},  % coloring and formatting of keywords as public, class, import 
  commentstyle=\color{KITblue}\ttfamily, % Color of comments
  frame=tb  % print top and bottom lines, frame=single, L
}


\lstdefinestyle{BASH}{
  basicstyle=\Large\ttfamily, % defines text formatting
  frame=none,
  language=bash
}


%%%%%%%%%%%%%%%%%%%%%%%%%%%%%%%%%%%%%%%%%%%%%%%%%%%%%%%%%%%%%%%%%%%%%%%%
% Titlepage
%%%%%%%%%%%%%%%%%%%%%%%%%%%%%%%%%%%%%%%%%%%%%%%%%%%%%%%%%%%%%%%%%%%%%%%%
\title[Institut f\"ur Angewandte und Numerische Mathematik]%
 {\fontsize{15}{15}\selectfont{}
  \"Ubung:
  \textit{Einstieg in die Informatik}\\[1.5mm]
  \textit{\phantom{\"Ubung:}\; und algorithmische Mathematik}\\[1.5mm]
  }%\hspace*{3cm}\normalsize{f\"ur Mathematiker}}
\author{\fontsize{9}{9}\selectfont{}
 Albert Mink\
  }
\institute[Institut f\"ur Angewandte und Numerische Mathematik]
 {\fontsize{6}{6}\selectfont{}%
  Institut f\"ur Angewandte und Numerische Mathematik}
\date{Wintersemester 2018/19}%
\subject{}%
%\beamerdefaultoverlayspecification{<+->}
%%%%%%%%%%%%%%%%%%%%%%%%%%%%%%%%%%%%%%%%%%%%%%%%%%%%%%%%%%%%%%%%%%%%%%%%



\makeatletter
\def\input@path{{uebungsfolien/}}
\graphicspath{{uebungsfolien/}}
\makeatother


%%%%%%%%%%%%%%%%%%%%%%%%%%%%%%%%%%%%%%%%%%%%%%%%%%%%%%%%%%%%%%%%%%%%%%%%
% Document
%%%%%%%%%%%%%%%%%%%%%%%%%%%%%%%%%%%%%%%%%%%%%%%%%%%%%%%%%%%%%%%%%%%%%%%%
\begin{document}
\maketitle%
\addtocounter{framenumber}{-1}%
%%%%%%%%%%%%%%%%%%%%%%%%%%%%%%%%%%%%%%%%%%%%%%%%%%%%%%%%%%%%%%%%%%%%%%%%

%%%%%%%%%%%%%%%%%%%%%%%%%%%%%%%%%%%%%%%%%%%%%%%%%%%%%%%%%%%%%%%%%%%%%%%%
%%%%%%%%%%%%%%%%%%%%%%%%%%%%%%%%%%%%%%%%%%%%%%%%%%%%%%%%%%%%%%%%%%%%%%%%
\begin{frame}
  \frametitle{Arbeitsblatt 4}%
\tableofcontents[hideallsubsections]
\end{frame}
\setcounter{exercise}{11}
%%%%%%%%%%%%%%%%%%%%%%%%%%%%%%%%%%%%%%%%%%%%%%%%%%%%%%%%%%%%%%%%%%%%%%%%


%%%%%%%%%%%%%%%%%%%%%%%%%%%%%%%%%%%%%%%%%%%%%%%%%%%%%%%%%%%%%%%%%%%%%%%%
%%%%%%%%%%%%%%%%%%%%%%%%%%%%%%%%%%%%%%%%%%%%%%%%%%%%%%%%%%%%%%%%%%%%%%%%
\section{Wiederholung}\label{K:wdh}
\begin{frame}
  \frametitle{\ref{K:wdh} Wiederholung}%
\tableofcontents[current]
\end{frame}
%%%%%%%%%%%%%%%%%%%%%%%%%%%%%%%%%%%%%%%%%%%%%%%%%%%%%%%%%%%%%%%%%%%%%%%%

%%TODO auf VL abstimmen

%%%%%%%%%%%%%%%%%%%%%%%%%%%%%%%%%%%%%%%%%%%%%%%%%%%%%%%%%%%%%%%%%%%%%%%%
\def\stitle{Rundungsfehler: Ausl\"oschung}
%%%%%%%%%%%%%%%%%%%%%%%%%%%%%%%%%%%%%%%%%%%%%%%%%%%%%%%%%%%%%%%%%%%%%%%%
%%%%%%%%%%%%%%%%%%%%%%%%%%%%%%%%%%%%%%%%%%%%%%%%%%%%%%%%%%%%%%%%%%%%%%%%
\subsection{\stitle}\label{S:rund}
\begin{frame}[t]%
 \frametitle{\ref{K:wdh}.\ref{S:rund} \stitle}

Der Fehler bei Subtraktion fast gleich gro\ss er Gleitkomma Zahlen wird \emph{Ausl\"oschung} genannt.
\medskip

\textbf{Wdh.: } Eine Gleitkommazahl $x$ wird wie folgt dargestellt
$$x = 0.x_1 x_2 x_3 ... x_m \cdot B^e.$$
F\"ur die einzelnen Ziffern $x_i$ gilt dabei $0 \leq x_i < B$ zu einer Basis $B$ mit Exponent $e$ und maximalen Mantissen-L\"ange $m$.
\begin{exampleblock}{Beispiel f\"ur Ausl\"oschung}
Betrachtet man eine Gleitkommazahl im Dezimalsystem ($B=10$) f\"ur welche $6$ Nachkommastellen dargestellt werden k\"onnen ($m=6$), so gilt:\\
\begin{eqnarray*}
0.239859 \cdot 10^0 - 0.239841 \cdot 10^0 &=& 0.000018 \cdot 10^0\\
 &=& 0.180000 \cdot 10^{-4}
\end{eqnarray*}
Die letzten $4$ Nullen sind nur korrekt, falls $0.239859$ und $0.239841$ exakte Zahlen waren.
Dabei k\"onnen Abweichungen in den letzten Nachkommastellen durch vorangegangene Rundungsfehler entstanden sein.
Der Fehler ist hier ca. $40\%$!
\medskip
\end{exampleblock}
\end{frame}
%%%%%%%%%%%%%%%%%%%%%%%%%%%%%%%%%%%%%%%%%%%%%%%%%%%%%%%%%%%%%%%%%%%%%%%%

%%%%%%%%%%%%%%%%%%%%%%%%%%%%%%%%%%%%%%%%%%%%%%%%%%%%%%%%%%%%%%%%%%%%%%%%
%%%%%%%%%%%%%%%%%%%%%%%%%%%%%%%%%%%%%%%%%%%%%%%%%%%%%%%%%%%%%%%%%%%%%%%%
\def\stitle{Rundungsfehler: Maschinengenauigkeit}
\subsection{\stitle}\label{S:Maschinengenauigkeit}
\begin{frame}[t]%
 \frametitle{\ref{K:wdh}.\ref{S:Maschinengenauigkeit} \stitle}
\medskip

Wann sind zwei Zahlen vom Datentyp \code{double} f\"ur den Rechner identisch?
\lstinputlisting[style=JAVAlines]{rundungsfehler/MaschinenGenauigkeit_double.java}

\end{frame}
%%%%%%%%%%%%%%%%%%%%%%%%%%%%%%%%%%%%%%%%%%%%%%%%%%%%%%%%%%%%%%%%%%%%%%%%


%%%%%%%%%%%%%%%%%%%%%%%%%%%%%%%%%%%%%%%%%%%%%%%%%%%%%%%%%%%%%%%%%%%%%%%%
%%%%%%%%%%%%%%%%%%%%%%%%%%%%%%%%%%%%%%%%%%%%%%%%%%%%%%%%%%%%%%%%%%%%%%%%
\def\stitle{Rundungsfehler: \code{int}-Maschinengenauigkeit}
\subsection{\stitle}\label{S:Maschinengenauigkeit1}
\begin{frame}[t]%
 \frametitle{\ref{K:wdh}.\ref{S:Maschinengenauigkeit1} \stitle}
\medskip

Wann sind zwei Zahlen vom Datentyp \code{int} f\"ur den Rechner identisch?
\lstinputlisting[style=JAVAlines]{rundungsfehler/MaschinenGenauigkeit_int.java}

\end{frame}
%%%%%%%%%%%%%%%%%%%%%%%%%%%%%%%%%%%%%%%%%%%%%%%%%%%%%%%%%%%%%%%%%%%%%%%%


\setcounter{exercise}{14}
\begin{exercise}{Maschinengenauigkeit}

\description{Berechnung der Maschinengenauigkeiten für \code|double| und \code|float|}

\source{Markus Richter}

\difficulty{leicht}

\utilization{WS 2009 (Praktikum)}


\begin{body}
Eine reelle Zahl $x \in \mathbb{R}$ wird im Rechner durch eine so genannte \emph{Maschinenzahl} oder \emph{Gleitkomma-Darstellung} $fl(x)$ eines bestimmen Datentyps (z.B. \code|double|, \code|float|, \code|int|) dargestellt. Bedingt durch Rundungsfehler stimmen $x$ und $fl(x)$ im allgemeinen nicht überein. Der \emph{relative Rundungsfehler}, der bei der Abbildung 
$x \mapsto fl(x)$ entsteht, ist durch
\[ e(x) := \frac{\lvert x - fl(x) \rvert }{\lvert x \rvert}  \]
definiert. Eine wichtige Kenngröße ist die so genannte \emph{Maschinengenauigkeit} $\epsilon$, die folgendermaßen definiert ist: $\epsilon$ ist die kleinste, positive Zahl, für die
\[ e(x) \leq \epsilon \quad\text{für alle } x \in \mathbb{R} \text{ mit } x_{Min} \leq x \leq x_{Max} \]
gilt. Hierbei bezeichnen $x_{Min}$ und $x_{Max}$ die kleinste und die größte Maschinenzahl des betreffenden Datentyps. Die Maschinengenauigkeit kann mit folgendem Algorithmus berechnet werden:
\begin{center}
\begin{minipage}{0.8\textwidth}
\begin{enumerate}
\item[(1)] Initialisiere $x$ mit $1$, und gehe zu (2).
\item[(2)] Solange $x + 1 \neq 1$ gilt, halbiere $x$. Ansonsten gehe zu (3).
\item[(3)] Die Maschinengenauigkeit $\epsilon$ ist $2x$.
\end{enumerate}
\end{minipage}
\end{center}
\bigskip
Bearbeiten Sie die folgenden Aufgaben:
\begin{parts}
\item
Schreiben Sie ein Java-Programm namens \code|EpsilonDouble|, welches die Maschinengenauigkeit für den Datentyp \code|double| berechnet und auf der Konsole ausgibt.

\item
Schreiben Sie ein Java-Programm namens \code|EpsilonInt|, welches die Maschinengenauigkeit für den Datentyp 
\code|int| berechnet und auf der Konsole ausgibt.
\end{parts}
Orientieren Sie sich an dem oben angegebenen Algorithmus. Definieren Sie für $x$ eine Variable vom Typ \code|double|,
und setzen Sie den Schritt (2) des Algorithmus mit einer \code|while|-Schleife um. Achten Sie in den Aufgabenteilen (b) und (c) darauf, die linke Seite der Ungleichung ${x + 1} \neq 1$ in den jeweils richtigen Datentyp zu konvertieren. Verwenden Sie dazu die expliziten Typenkonvertierung \code|(int)|.
\end{body}

\begin{solution}
\begin{small}
\inputcode[frame=lines,title=EpsilonDouble.java]{\filename{src/EpsilonDouble.java}}
\inputcode[frame=lines,title=EpsilonInt.java]{\filename{src/EpsilonInt.java}}
\end{small}

\bigskip
\noindent
\paragraph{Hinweise}
\begin{itemize}
\item
Die Maschinengenauigkeit für \code|double| ist $2{,}220446049250313\cdot 10^{-16}$.

\item
Die Maschinengenauigkeit für \code|float| ist $1{,}1920928955078125\cdot 10^{-7}$.

\item
Die Maschinengenauigkeit für \code|int| ist $1$. 
\end{itemize}
\end{solution}
\end{exercise}

\setcounter{exercise}{15}
\def\stitle{\theexercise\ - Kompilerfehler}
\section{\stitle}

\begin{frame}[t]%
\frametitle{\stitle}
\medskip
Programmfehler anhand des Compilers finden.
\newline
Das Java-Program \code{Fehler.java} wurde kompiliert, wobei einige Fehler und Warnungen auftraten.
Finden Sie mit Hilfe der Kompilerausgaben die Fehler.
\lstinputlisting[style=JAVAlines]{\getexercisefolder/Fehler.java}
 \end{frame}

\setcounter{exercise}{16}
\begin{frame}[t]%

\begin{exercise}{Schleifen}
\begin{body}
Gegeben sei der folgende Ausschnitt eines Java-Programms.

\lstinputlisting[style=JAVAlines]{schleifen-2/Schleifen2.java}

\begin{parts}
\item[(a)] Was wird auf dem Bildschirm ausgegeben?
\item[(b)] Realisieren Sie diesen Programmausschnitt mit einer \code{for}-Schleife.
\end{parts}
\end{body}
\end{exercise}
\end{frame}


\begin{frame}[fragile]%
 \frametitle{a) L\"osungsweg}%
\lstinputlisting[style=JAVAlines]{schleifen-2/Schleifen2.java}
\end{frame}


\begin{frame}[fragile]%
 \frametitle{b) Als \code{for} Schleife}%
\lstinputlisting[style=JAVAlines]{schleifen-2/Schleifen2_for.java}
\end{frame}


\begin{frame}
\centering
\Huge\GREEN{Fragen?}
\vspace{2cm}

{\LARGE
N\"achste \"Ubung: 20. November\\
Besprechung Arbeitsblatt 5
}
\end{frame}


%%%%%%%%%%%%%%%%%%%%%%%%%%%%%%%%%%%%%%%%%%%%%%%%%%%%%%%%%%%%%%%%%%%%%%%%
\end{document}
