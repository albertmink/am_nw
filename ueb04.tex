\documentclass[c,18pt]{beamer}
\listfiles

\mode<presentation>
{
  \usetheme[deutsch,titlepage0]{KIT}
  \setbeamercovered{transparent}
  \setbeamertemplate{enumerate items}[ball]
}
\usepackage[utf8]{inputenc}
\date{12.11.2019}

\newlength{\Ku}
\setlength{\Ku}{1.43375pt}

\usepackage{templateSlide/exercises}
\usepackage[java]{code}

\exercisespath{uebungsfolien/}

%% title slide
\title[Übung 04: Einstieg in die Informatik und algorithmische Mathematik]
  {Übung 04: Einstieg in die Informatik und algorithmische Mathematik}
\subtitle{Albert Mink | Wintersemester 2019/2020}

\author[Albert Mink, ]{KIT}

\AuthorTitleSep{\relax}

\institute[Institut für Angewandte und Numerische Mathematik (IANM)]{Institut für Angewandte und Numerische Mathematik}

\TitleImage[width=\titleimagewd]{logos/KIT-Titel}
\logo{\includegraphics{logos/lbrg_logo}}

%%%%%%%%%%%%%%%%%%%%%%%%%%%%%%%%%%%%%%%%%%%%%%%%%%%%%%%%%%%%%%%%%%%%%%%%
% Document
%%%%%%%%%%%%%%%%%%%%%%%%%%%%%%%%%%%%%%%%%%%%%%%%%%%%%%%%%%%%%%%%%%%%%%%%
\begin{document}
\begin{frame}
  \maketitle
\end{frame}
%%%%%%%%%%%%%%%%%%%%%%%%%%%%%%%%%%%%%%%%%%%%%%%%%%%%%%%%%%%%%%%%%%%%%%%%

%%%%%%%%%%%%%%%%%%%%%%%%%%%%%%%%%%%%%%%%%%%%%%%%%%%%%%%%%%%%%%%%%%%%%%%%
%%%%%%%%%%%%%%%%%%%%%%%%%%%%%%%%%%%%%%%%%%%%%%%%%%%%%%%%%%%%%%%%%%%%%%%%
\begin{frame}
  \frametitle{Übung zum Arbeitsblatt 04}%
\tableofcontents
\end{frame}
%%%%%%%%%%%%%%%%%%%%%%%%%%%%%%%%%%%%%%%%%%%%%%%%%%%%%%%%%%%%%%%%%%%%%%%%

\setcounter{exercise}{13}
\inputexercise{schleifen-2}
\def\kap{1}
\inputexercise{rundungsfehler}
\setcounter{exercise}{14}
\inputexercise{rundf-epsilon}
\setcounter{exercise}{15}
\inputexercise{anweis-fehler}

%%%%%%%%%%%%%%%%%%%%%%%%%%%%%%%%%%%%%%%%%%%%%%%%%%%%%%%%%%%%%%%%%%%%%%%%
\section{Zusammenfassung}
\begin{frame}
  \frametitle{Zusammenfassung}%
\tableofcontents[hideallsubsections]
\end{frame}

\begin{frame}
\centering
\Huge\textcolor{KITgreen}{Fragen?}
\vspace{2cm}

{\LARGE
N\"achste \"Ubung: 19. November\\
Besprechung Arbeitsblatt 5
}
\end{frame}


%%%%%%%%%%%%%%%%%%%%%%%%%%%%%%%%%%%%%%%%%%%%%%%%%%%%%%%%%%%%%%%%%%%%%%%%
\end{document}
