\documentclass[c,18pt]{beamer}
\listfiles

\mode<presentation>
{
  \usetheme[deutsch,titlepage0]{KIT}
  \setbeamercovered{transparent}
  \setbeamertemplate{enumerate items}[ball]
}
\usepackage[utf8]{inputenc}
\date{30.10.2018}

\newlength{\Ku}
\setlength{\Ku}{1.43375pt}

\usepackage{templateSlide/exercises}
\usepackage[java]{code}

\makeatletter
\def\input@path{{uebungsfolien/}}
\graphicspath{{uebungsfolien/}}
\makeatother

%% title slide
\title[Übung 02: Einstieg in die Informatik und algorithmische]
  {Übung 02: Einstieg in die Informatik und algorithmische \\ Mathematik}
\subtitle{Albert Mink | Wintersemester 2018/2019}

\author[Albert Mink, ]{KIT}

\AuthorTitleSep{\relax}

\institute[Institut für Angewandte und Numerische Mathematik (IANM)]{Institut für Angewandte und Numerische Mathematik}

\TitleImage[width=\titleimagewd]{logos/KIT-Titel}
\logo{\includegraphics{logos/lbrg_logo}}

%%%%%%%%%%%%%%%%%%%%%%%%%%%%%%%%%%%%%%%%%%%%%%%%%%%%%%%%%%%%%%%%%%%%%%%%
% Document
%%%%%%%%%%%%%%%%%%%%%%%%%%%%%%%%%%%%%%%%%%%%%%%%%%%%%%%%%%%%%%%%%%%%%%%%
\begin{document}
\begin{frame}
  \maketitle
\end{frame}
%%%%%%%%%%%%%%%%%%%%%%%%%%%%%%%%%%%%%%%%%%%%%%%%%%%%%%%%%%%%%%%%%%%%%%%%
%%%%%%%%%%%%%%%%%%%%%%%%%%%%%%%%%%%%%%%%%%%%%%%%%%%%%%%%%%%%%%%%%%%%%%%%
\begin{frame}
  \frametitle{Arbeitsblatt 2}%
\tableofcontents
\end{frame}
%%%%%%%%%%%%%%%%%%%%%%%%%%%%%%%%%%%%%%%%%%%%%%%%%%%%%%%%%%%%%%%%%%%%%%%%
\setcounter{exercise}{4}
\section{Aufgabe 5 - Addition von Bin"arzahlen}
\begin{frame}[t]%
\medskip

\begin{exercise}{Addition von Binärzahlen}
\begin{body}
\medskip

Berechnen Sie die folgenden Binärzahlsummen, ohne eine Umrechung in ein anderes Stellenwertsystem (wie beispielsweise das der Dezimalzahlen) vorzunehmen.
\begin{center}
\begin{minipage}{0.45\textwidth}
\begin{itemize}
\item[(a)] $1010_2 +   11_2$
\item[(b)] $1101_2 + 1010_2$
\item[(c)] $1010_2 +  101_2$
\item[(d)] $1000_2 + 1000_2$
\end{itemize}
\end{minipage}
\begin{minipage}{0.45\textwidth}
\begin{itemize}
\item[(e)] $10010000_2 + 1111_2$
\item[(f)] $10001111_2 + 1_2$
\item[(g)] $10101010_2 + 11101_2$
\item[(h)] $10010001_2 + 1010011_2$
\end{itemize}
\end{minipage}
\end{center}
\end{body}

%%%%%%%%%%%%%%%%%%%%%%%%%%%%%%%%%%%%% solution %%%%%%%%%%%%%%%%%%%%%%%%%%%%%%%%%%%%%%%%%%%%%%%%%%%%%%%%%%%%%%%
\begin{solution}
\begin{center}
\begin{minipage}{0.45\textwidth}
\begin{itemize}
\item[(a)] $1010_2 +   11_2 = 1101_2$
\item[(b)] $1101_2 + 1010_2 = 10111_2$
\item[(c)] $1010_2 +  101_2 = 1111_2$
\item[(d)] $1000_2 + 1000_2 = 10000_2$
\end{itemize}
\end{minipage}
\begin{minipage}{0.45\textwidth}
\begin{itemize}
\item[(e)] $10010000_2 + 1111_2 = 10011111_2$
\item[(f)] $10001111_2 + 1_2 = 10010000_2$
\item[(g)] $10101010_2 + 11101_2 = 11000111_2$
\item[(h)] $10010001_2 + 1010011_2 = 11100100_2$
\end{itemize}
\end{minipage}
\end{center}
\end{solution}

\end{exercise}
\end{frame}



%%%%%%%%%%%%%%%%%%%%%%%%%%%%%%%%%%%%%%%%%%%%%%%%%%%%%%%%%%%%%%%%%%%%%%%%
\begin{frame}[t]%
  \frametitle{Aufgabe 5, L\"osungsweg}%
\medskip

\begin{solution}{Addition von Bin\"arzahlen}
\medskip

\begin{table}
\caption{$(a)$}
\begin{tabular}{r||r}
${1010}_2$ & $10$ \\
${11}_2$   & $3$  \\ \hline
${1101}_2$ & $13$
\end{tabular}
\end{table}

\begin{table}
\caption{$(d)$}
\begin{tabular}{r||r}
${1000}_2$  & $8$ \\
${1000}_2$  & $8$  \\ \hline
${10000}_2$ & $16$
\end{tabular}
\end{table}

\end{solution}

\end{frame}

%%%%%%%%%%%%%%%%%%%%%%%%%%%%%%%%%%%%%%%%%%%%%%%%%%%%%%%%%%%%%%%%%%%%%%%%
\setcounter{exercise}{6}
\def\stitle{\theexercise\ - Ganzzahl-Datentypen}
\section{\stitle}
\begin{frame}[t]
  \frametitle{\stitle}

In Java besitzen die Ganzzahl-Datentypen die folgenden Größen
\begin{center}
\begin{tabular}{|c|c|}
\hline
\textbf{Datentyp} & \textbf{Größe} \\
\hline
\code{byte}       &  8 Bit         \\
\code{short}      & 16 Bit         \\
\code{int}        & 32 Bit         \\
\code{long}       & 64 Bit         \\
\hline
\end{tabular}
\end{center}
Das führende Bit jedes Datentyps gibt das Vorzeichen der Zahl an:
Ein 0-Bit zeigt eine nichtnegative Zahl an, ein 1-Bit eine negative Zahl.
Jedem Bitmuster entspricht genau eine ganze Zahl.
Ist das führende Bit ein 0-Bit (nichtnegative Zahl), repräsentieren die übrigen Bits die Ziffern der Binärdarstellung dieser Zahl.
\begin{itemize}
\item Wie viele verschiedene ganze Zahlen können mit den Datentypen \code{byte}, \code{short}, \code{int} und \code{long} dargestellt werden?

\item[Lsg]
jedes Bitmuster eines Datentyps entspricht genau einer ganzen Zahl.
Jedes Bit kann genau zwei Werte annehmen.
Daher können mit den Datentypen \code{byte}, \code{short}, \code{int} und \code{long} jeweils $2^8$, $2^{16}$, $2^{32}$, bzw. $2^{64}$ verschiedene ganze Zahlen dargestellt werden.

\end{itemize}

\end{frame}

\begin{frame}[fragile]%
  \frametitle{Fortsetzung}%
\centering
\medskip

\begin{itemize}
\item
Welches sind jeweils die größten positiven, ganzen Zahlen, die mit den Datentypen \code{byte}, \code{short}, \code{int} und \code{long} dargestellt werden können?
\item[Lsg]
Der Datentyp \code{byte} mit 8 Bit kann maximal siebenstellige, positive Binärzahlen darstellen, das führende Bit bei solchen Zahlen ein 0-Bit sein muss.
Die größte positive, ganze Zahl lautet daher $1111111_2 = 127 = 128 - 1 = 2^7 - 1$.
Für die Datentypen \code{short}, \code{int} und \code{long} erhält man in gleicher Weise $2^{15} - 1$, $2^{31} - 1$ bzw. $2^{63} - 1$ als größte darstellbare positive, ganze Zahl.
\end{itemize}

\end{frame}

\begin{frame}[fragile]%
  \frametitle{Fortsetzung}%
\centering
\medskip

\begin{itemize}
\item
Betrachten Sie den folgenden Quelltextausschnitt:
\begin{verbatim}
byte a = 100;
a += 100;
\end{verbatim}
Stellt der Wert der Variable \code{a} nach Ausführung dieser Zeilen die Zahl $200$ dar?
Begründen Sie Ihre Antwort.

\item[Lsg]
Nein, der Wert der Variable a stellt nach Ausführung der Quelltextzeilen die Zahl $200$ nicht dar.
Der Datentyp der Variable ist \code{byte}.
Die größte positive, ganze Zahl, die von diesem Datentyp dargestellt werden kann, ist $2^7 - 1 = 127$. Die Zahl $200$ ist somit nicht darstellbar.
Man kann die Antwort noch weiter erläutern: Die Binärdarstellung der Zahl $100 = 64 + 32 + 4$ lautet $1100100_2$.
Der Datentyp \code{byte} stellt sie als das Bitmuster $01100100$ dar.
Das führende 0-Bit zeigt an, dass die Zahl nichtnegativ ist.
Addiert man nun $100$ zu der Zahl hinzu, so erhält man $200 = 2 \cdot 100 = 128 + 64 + 8$, in Binärdarstellung $11001000_2$.
Dem entspricht das Bitmuster $11001000$.
Da das führende Bit ein 1-Bit ist, stellt dieses Bitmuster im Datentyp \code{byte} eine negative Zahl dar.
\end{itemize}
\end{frame}


\def\stitle{Rechnerdemo}
\section{\stitle}
\begin{frame}[fragile]%
  \frametitle{\stitle}%
\medskip

\"Uberlauf
\lstinputlisting[style=JAVAsmall]
{range-integer/RangeInt.java}
\end{frame}

%%%%%%%%%%%%%%%%%%%%%%%%%%%%%%%%%%%%%%%%%%%%%%%%%%%%%%%%%%%%%%%%%%%%%%%%
\setcounter{exercise}{7}
\def\stitle{\theexercise\ - Gleikomma-Datentypen}
\section{\stitle}
\begin{frame}[t]
  \frametitle{\stitle}
\tableofcontents[current]
\end{frame}

\begin{frame}
\frametitle{\stitle}

In Java stellen die Gleitkomma-Datentypen Bruchzahlen dar, indem sie die Nachkommastellen und den Exponenten der normalisierten Binärbruchdarstellung abspeichern. Für die Speicherung der Nachkommastellen steht folgender Speicherplatz bereit.
\begin{center}
\begin{tabular}{|c|c|}
\hline
\textbf{Datentyp} & \textbf{Nachkommastellen} \\
\hline
\code{float}       & 23 Bit                   \\
\code{double}      & 52 Bit                   \\
\hline
\end{tabular}
\end{center}


\begin{enumerate}
\item[1.]
Welches ist die nächstgrößere Bruchzahl oberhalb der Eins, die vom Datentyp \code{float} bzw. vom Datentyp \code{double} exakt dargestellt werden kann?

\item[Lsg]
Die nächstgrößere, exakt darstellbare Bruchzahl oberhalb der Eins ist $1 + 2^{-23}$ für den Datentyp \code{float} und $1 + 2^{-52}$ für den Datentyp \code{double}.

\item[2.]
Was versteht man im Zusammenhang mit Gleitkomma-Datentypen unter Rundungsfehlern? Wodurch entstehen sie?

\item[Lsg]
Unter dem Begriff Rundungsfehler versteht man Fehler, die auftreten, wenn eine Bruchzahl vom jeweiligen Datentyp nicht exakt dargestellt wird.
Rundungsfehler entstehen dadurch, dass jeder Datentyp nur eine begrenzte Anzahl von Nachkommastellen der normalisierten Binärbruchdarstellung speichern kann.
\end{enumerate}

\end{frame}


\begin{frame}[t]%
  \frametitle{Fortsetzung}%
\centering
\medskip

\begin{enumerate}
\item[3.]
Welche Maschinengenauigkeit erzielt man mit dem Datentyp \code{float}, welche mit dem Datentyp \code{double}?

\item[Lsg]
Die Maschinengenauigkeit des Datentyps \code{float} beträgt $2^{-23} \approx 10^{-7}$, die des Datentyps \code{double} beträgt $2^{-52} \approx 2 \cdot 10^{-16}$.

\item[4.]
Was versteht man unter Absorption im Zusammenhang mit Gleitkomma-Arithmetik?

\item[Lsg]
Unter Absorption versteht man die Entstehung von Rundungsfehlern bei der Addition zweier Gleitkommazahlen, deren Beträge unterschiedliche Größenordnungen haben.

\emph{Beipiel}: Mit dem Datentyp \code{float} können ca{.} $8$ Dezimalstellen dargestellt werden.
Die Addition von $1{,}0000001 \cdot 10^7$ ($8$ Stellen) und $1 \cdot 10^{-1}$ ($1$ Stelle) ergibt $1{,}00000011 \cdot 10^7$ ($9$ Stellen).
Dieses Ergebnis kann mit dem Datentyp \code{float} nicht exakt dargestellt werden und wird auf $1{,}0000001 \cdot 10^7$ gerundet.
\end{enumerate}
\end{frame}

\begin{frame}[t]%
  \frametitle{Fortsetzung}%
\centering
\medskip

\begin{enumerate}
\item[5.]
Was versteht man unter Auslöschung im Zusammenhang mit Gleitkomma-Arithmetik?

\item[Lsg]
Unter Auslöschung versteht man die Entstehung von Rundungsfehlern bei der Subtraktion zweier fast gleicher Gleitkommazahlen.

\emph{Beispiel}: Subtrahiert man von der Zahl $1{,}0000011$ die Zahl $1{,}0000010$, so erwartet man das Ergebnis $1 \cdot 10^{-7}$.
Führt man diese Subtraktion in Gleitkomma-Arithmetik mit dem Datentyp \code{float} durch, so erhält man das Ergebnis $1{,}1920929 \cdot 10^{-7}$.
Nur eine einzige Stelle dieses Wertes ist exakt.
\end{enumerate}

\end{frame}


\def\stitle{Rechnerdemo}
\section{\stitle}
\begin{frame}[fragile]%
  \frametitle{\stitle}%
\medskip

Range float
\lstinputlisting[style=JAVAsmall]
{rangeFloat/RangeFloat.java}
\end{frame}



\def\stitle{Rechnerdemo}
\section{\stitle}
\begin{frame}[fragile]%
  \frametitle{\stitle}%
\medskip

Ausl"oschung
\lstinputlisting[style=JAVAsmall]
{ausloeschung/Ausloeschung.java}
\end{frame}

%%%%%%%%%%%%%%%%%%%%%%%%%%%%%%%%%%%%%%%%%%%%%%%%%%%%%%%%%%%%%%%%%%%%%%%%%



\setcounter{exercise}{8}
\def\stitle{\theexercise\ - Boolsche Algebra}
\section{\stitle}
\begin{frame}
    \frametitle{\stitle}%
\tableofcontents[current]
\end{frame}

\begin{frame}[t]
  \frametitle{\stitle}

In Java repräsentiert der Basistyp \code{boolean} eine Boolesche Algebra.
Die Symbole $0, 1$ sind in Java als \code{false}, \code{true} definiert.
Daneben gibt es das logische Und, logische Oder, bzw. \code{\&\&}, \code{||} und \code{!}.

Geben Sie das Ergebnis der nachfolgenden Java-Ausdrücke an.
\medskip

\begin{minipage}{0.49\textwidth}
\begin{itemize}
\item[(a)] \code{true && false}
\item[(b)] \code{true || false}
\item[(c)] \code{(0 == 1) || (1 < 2)}
\end{itemize}
\end{minipage}
\begin{minipage}{0.49\textwidth}
\begin{itemize}
\item[(d)] \code{(0 != 1) && !(2 < 1)}
\item[(e)] \code{!!!true}
\item[(f)] \code{(5 == 1) || false}
\end{itemize}
\end{minipage}
\end{frame}

\begin{frame}[t]
  \frametitle{\stitle\ - L\"osungsvorschlag}
\begin{center}
\begin{minipage}{0.49\textwidth}
\begin{itemize}
\item[(a)]\code{true && false}       $\rightarrow$ \code{false}
\item[(b)]\code{true || false}       $\rightarrow$ \code{true}
\item[(c)]\code{(0 == 1) || (1 < 2)} $\rightarrow$ \code{true}
\end{itemize}
\end{minipage}
\begin{minipage}{0.49\textwidth}
\begin{itemize}
\item[(d)]\code{(0 != 1) && !(2 < 1)} $\rightarrow$ \code{true}
\item[(e)]\code{!!!true}              $\rightarrow$ \code{false}
\item[(f)]\code{(5 == 1) || false}    $\rightarrow$ \code{false}
\end{itemize}
\end{minipage}
\end{center}

\end{frame}


\begin{frame}
\centering
\Huge\textcolor{KITgreen}{Fragen?}
\vspace{2cm}

{\LARGE
N\"achste \"Ubung: 06. November\\
Besprechung Arbeitsblatt 3
}
\end{frame}


%%%%%%%%%%%%%%%%%%%%%%%%%%%%%%%%%%%%%%%%%%%%%%%%%%%%%%%%%%%%%%%%%%%%%%%%
\end{document}
