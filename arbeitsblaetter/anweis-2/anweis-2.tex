\begin{exercise}{Anweisungen}
\begin{body}
Gegeben sei der folgende Ausschnitt eines Java-Programms. Dabei sei \code|i| eine Variable vom Typ \code|int|.
\begin{displaycode}
if ( i < 1 || i > 3 ) {
    System.out.println("A");
} else if ( i == 1 || i == 2  ) {
    System.out.println("B");
} else {
    System.out.println("C");
}
\end{displaycode}
\begin{parts}
\item[(a)] Welcher Buchstabe wird auf dem Bildschirm ausgegeben, falls \code|i| den Wert $1$, $2$, $3$ bzw. $4$ besitzt?
\item[(b)] Realisieren Sie diesen Programmausschnitt mit einer \code|switch|-Anweisung.
\end{parts}
\end{body}

\begin{solution}
\begin{parts}
\item[(a)] Für den vorliegenden Programmausschnitt erhält man
\begin{center}
\begin{tabular}{|c|c|}
\hline
$\quad$\code|n|$\quad$  & \textbf{Ausgabe} \\
\hline
$1$ & B \\
$2$ & B \\
$3$ & C \\
$4$ & A \\
\hline
\end{tabular}
\end{center}

\item[(b)] Realisation mit einer \code|switch|-Anweisung:
\begin{displaycode}
switch ( i ) {
    case 1:
    case 2: 
        System.out.println("B");
        break;
    case 3:
        System.out.println("C");
        break;
    default:
        System.out.println("A");
}
\end{displaycode}
\end{parts}
\end{solution}
\end{exercise}
