\begin{exercise}{Klassenreferenzen}
\source{Florian Eheim}
\begin{body}
\noindent
Wird ein Objekt als Argument an eine Methode "ubergeben, so geschieht das in Java auf unterschiedliche Weise. W"ahrend die Methode
bei Grundtypen (z.B. \code{int}, \code{double} oder \code{char}) eine einfache Kopie benutzt, wird bei zusammengesetzten Datentypen
(Feldern, Strings, Klassen,...) eine \emph{Referenz} "ubergeben, welche auf das urspr"ungliche Feld bzw. die urspr"ungliche Klasse verweist.
Ziel dieser Aufgabe ist es, zu sehen, wie sich "Anderungen innerhalb einer Methode auf das urspr"ungliche Objekt auswirken.

\noindent
Betrachten Sie die folgende Klasse:
\medskip

\begin{displaycode}
class MeineKlasse {
  public int zahl;
} 
\end{displaycode}
\medskip

\noindent
Au\ss erdem ist die folgende \code{main}-Methode eines Java-Programms gegeben:
\begin{displaycode}
public static void main(String[] args) {
  /*** Grunddatentypen: ***/ 
  int zahl_a = 0;
  int zahl_b = 0;
  
  zahl_b = ErhoeheZahl(zahl_a);
  System.out.println(zahl_a + " : " + zahl_b);
  zahl_b = ErhoeheZahlEintrag(zahl_a);
  System.out.println(zahl_a + " : " + zahl_b);
  
  /*** Felder: ***/
  int[] feld_a = new int[1];
  feld_a[0] = 0;
  int[] feld_b;
  
  feld_b = ErhoeheFeld(feld_a);
  System.out.println(feld_a[0] + " : " + feld_b[0]);	
  feld_b = ErhoeheFeldEintrag(feld_a);
  System.out.println(feld_a[0] + " : " + feld_b[0]);	
  
  /*** Klassen: ***/
  MeineKlasse Klasse_a = new MeineKlasse();
  MeineKlasse Klasse_b;
  Klasse_a.zahl = 0;
  
  Klasse_b = ErhoeheKlasse(Klasse_a);
  System.out.println(Klasse_a.zahl + " : " + Klasse_b.zahl);
  Klasse_b = ErhoeheKlassenEintrag(Klasse_a);
  System.out.println(Klasse_a.zahl + " : " + Klasse_b.zahl);
}
\end{displaycode}
\newpage

\noindent
Dabei werden die folgenden Methoden aufgerufen:
\begin{displaycode}
static int ErhoeheZahl(int zahl) {
  int hilfsZahl = 0;
  hilfsZahl = zahl + 1;
  return hilfsZahl;
}
  
static int ErhoeheZahlEintrag(int zahl) {
  zahl = zahl + 1;
  return zahl;
}

static int[] ErhoeheFeld(int[] feld) {
  int[] hilfsFeld = new int[1];
  hilfsFeld[0] = feld[0] + 1;
  return hilfsFeld;
}

static int[] ErhoeheFeldEintrag(int[] feld) {
  feld[0] = feld[0] + 1;
  return feld;
}

static MeineKlasse ErhoeheKlasse(MeineKlasse Klasse) {
  MeineKlasse HilfsKlasse = new MeineKlasse();
  HilfsKlasse.zahl = Klasse.zahl + 1;
  return HilfsKlasse;
}

static MeineKlasse ErhoeheKlassenEintrag(MeineKlasse Klasse) {
  Klasse.zahl = Klasse.zahl + 1;
  return Klasse;
}
\end{displaycode}
Wie lautet die Ausgabe des Programms? Geben Sie beim Aufruf jeder Methode an, ob eine Referenz oder Kopie "ubergeben wird.

\end{body}
\begin{solution}
Die Ausgabe lautet:
 \begin{itemize}
  \item \code{0 : 1} ``Kopie''
  \item \code{0 : 1} ``Kopie''
  \item \code{0 : 1} ``Referenz''
  \item \code{1 : 1} ``Referenz''
  \item \code{0 : 1} ``Referenz''
  \item \code{1 : 1} ``Referenz''
 \end{itemize}

\end{solution}
\end{exercise}
