\begin{exercise}{Felder: Bin"are Suche}
\begin{body}
Gegeben sei untenstehende Implementierung der Bin"aren Suche. Das Feld \verb|feld| sei dabei ein aufsteigend geordnetes Feld aus ganzen Zahlen (z.B. 1, 3, 5, 8, 9, 100).
\begin{verbatim}
import java.util.*;

class binsearch {
  public static void main(String[] args){
    Locale.setDefault(Locale.US);
    Scanner sc = new Scanner(System.in);
    
    int[] feld = {2, 4, 6, 8, 10, 12, 14};
    
    System.out.println("Bitte Wert eingeben : ");
    int wert=sc.nextInt();
    
    int min = 0;
    int max = feld.length-1;
    
    while(min!=max){
      int mid = (min+max)/2;
      System.out.println(" feld("+mid+") = " + feld[mid]);
    
      if (wert < feld[mid])
        max = mid-1;
      if (wert > feld[mid])
        min = mid+1;
      if (wert == feld[mid] || min > max)
        min = max = mid;
    }
    
    if (wert != feld[min])
      System.out.println("Der Wert "+wert+" ist nicht in feld gespeichert ");
    else
      System.out.println("Der Wert "+wert+" ist in feld("+min+") gespeichert ");
  }
}
\end{verbatim}
Welche Ausgabe liefert das Programm f"ur die Eingaben 9 bzw. 10 zur"uck?
\end{body}

%%%%%%%%%%%%%%%%%%%%%%%%%%%%%%%%%%%%% solution %%%%%%%%%%%%%%%%%%%%%%%%%%%%%%%%%%%%%%%%%%%%%%%%%%%%%%%%%%%%%%%
\begin{solution}

Eingabe 9:\\
 feld(3) = 8\\
 feld(5) = 12\\
Der Wert 9 ist nicht in feld gespeichert\\

Eingabe 10:\\
 feld(3) = 8\\
 feld(5) = 12\\
Der Wert 10 ist in feld(4) gespeichert
\end{solution}

\end{exercise}
