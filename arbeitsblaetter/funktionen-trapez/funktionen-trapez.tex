\begin{exercise}{Funktionen-Trapezregel}
\begin{body}
Berechnen Sie numerisch das Integral der Funktion~$f$ definiert als $f(x) = -x^2+4$ in den Grenzen~$0$ und~$3$.

\begin{parts}
\item[(a)] Verwenden Sie die Trapezregel mit zwei St\"utzstellen
\[
I_1
= \int_a^b f(x)dx
\simeq (b-a)\frac{f(a)+f(b)}{2}
.\]
Hier soll $a=0$ und $b=3$ sein.
\item[(b)] Verwenden Sie die zusammengesetzte Trapezregel f\"ur $N$ St\"utzstellen
\[
I_2
= \int_a^b f(x)dx
\simeq h \left[\frac{1}{2}f(a)+\frac{1}{2}f(b)+\sum_{n=1}^{N-1}f\left(a+nh \right)\right]
,\]
mit Gewicht $h = \frac{b-a}{N}$ und Integrationsgrenzen $a=0$ und $b=3$.
\hint{Implementieren Sie eine Java-Funktion}

\begin{displaycode}
public static function(double x) {
  return -x*x + 4;
}
\end{displaycode}
\end{parts}

\end{body}


\begin{solution}
Analytische Berechnung von Integral liefert $\int_0^3 -x^2 +4 \; dx = 3$.
Beispiel Implementierung.
\inputcode[frame=lines,title=Trapez.java]{\filename{Trapez.java}}
\end{solution}

\end{exercise}
