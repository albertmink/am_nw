\begin{exercise}{Vektorrechnung}

\begin{body}
Im Praktikum haben Sie die Implementation grundlegender Vektoroperationen im $\mathbb R^n$ f"ur $n>0$ kennengelernt (Aufgabenblatt 4, Aufgabe 8).
Diese wurden direkt auf zwei eingelesene Vektoren $x=[x_1,\ldots,x_n]\in\mathbb R^n$ und $y=[y_1,\ldots,y_n]\in\mathbb R^n$ angewendet.
In vielen Anwendungen werden Vektoroperationen allerdings so h"aufig verwendet, dass es sich lohnt diese in Methoden auszulagern. Dadurch
wird Programmieraufwand eingespart und die "Ubersichtlichkeit erh"oht. Außerdem k"onnen Methoden seperat getestet werden. Dieses Vorgehen
vermeidet Fehler und erleichtert die Fehlersuche.
\bigskip

\noindent
Entwickeln Sie Methoden für die folgenden Vektoroperationen:
\begin{itemize}
 \item Die {\em Addition} und {\em komponentenweise Multiplikation} zweier Vektoren
    \mbox{\boldmath $x=[x_1,\ldots,x_n]\in \mathbb{R}^n$} und
    \mbox{\boldmath $y=[y_1,\ldots,y_n]\in \mathbb{R}^n$} ist definiert gem"a{\ss}
    \begin{eqnarray*}        
      \mbox{\boldmath $x$} + \mbox{\boldmath $y$} &:=&  \bigl[ x_1 + y_1, \ldots, x_n + y_n \bigr], \\
      \mbox{\boldmath $x$}  * \mbox{\boldmath $y$} &:=&  \bigl[ x_1 y_1,\ldots, x_n y_n \bigr].
    \end{eqnarray*}
\item Das {\em Skalarprodukt} zweier Vektoren
\mbox{\boldmath $x$} und  \mbox{\boldmath $y$} ist gegeben 
    durch
    \begin{eqnarray*}          
      \mbox{\boldmath $x$}   \cdot         \mbox{\boldmath $y$} &:=&	 \sum_{i=1}^n x_i y_i.
    \end{eqnarray*}
\item Die {\em skalare Multiplikation} eines Vektors \mbox{\boldmath $x$} 
    mit einer reellen Zahl $\lambda$ ist gegeben durch 
    \begin{eqnarray*}        
      \lambda \cdot \mbox{\boldmath $x$} &:=&  \bigl[ \lambda  x_1, \ldots, \lambda  x_n       \bigr].
    \end{eqnarray*}
\end{itemize}
\newpage

\noindent
Gehen Sie dabei wie folgt vor:
\begin{itemize}
\item [(a)] Erstellen Sie eine Methode \code{Einlesen} die dazu dienen soll, einen neuen Vektor zu erstellen und dessen Werte einzugeben.
      Als Argument soll die Methode die Dimension \code{n} vom Typ \code{int} erhalten und den R"uckgabetyp \code{double[]} besitzen.
\item [(b)] Erstellen Sie eine Methode \code{Ausgeben} ohne R"uckgabewert, welche als Argument ein \code{double}-Feld \code{x} erh"alt.
      Die einzelnen Eintr"age des Feldes sollen beim Aufruf dieser Methode nacheinander auf der Konsole ausgegeben werden. Die L"ange eines
      Feldes \code{Feld} k"onnen Sie im Allgemeinen durch den Ausdruck \code{Feld.length} erhalten.
\item [(c)] Erstellen Sie die Methoden \code{Addition}, \code{Multiplikation} und \code{Skalarprodukt} denen jeweils zwei Felder \code{x} und \code{y} als Argumente "ubergeben werden.
      Auch hier soll das Ergebnis nach obigen Formeln berechnet und zur"uckgegeben werden. Achten Sie darauf dass die Methoden den passenden R"uckgabetyp besitzen.
\item [(d)] Erstellen Sie die Methode \code{SkalareMultiplikation} der eine Zahl \code{skalar} vom Typ \code{double} und ein Feld \code{x} als Argumente "ubergeben wird.
      Das Ergebnis soll nach obiger Formel berechnet und zur"uckgegeben werden. Welcher R"uckgabetyp muss verwendet werden?
\end{itemize}

\end{body}

\begin{solution}
\inputcode[frame=lines,title=Vektorrechnung.java]{\filename{src/Vektorrechnung.java}}
\end{solution}
\end{exercise}
