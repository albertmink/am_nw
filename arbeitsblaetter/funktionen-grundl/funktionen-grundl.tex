\begin{exercise}{Funktionen}
\begin{body}
Funktionen stellen gekapselte Programmabschnitte dar in denen seperat Befehle ausgef\"uhrt werden.
Eine Funktion kann \"Ubergabeparameter und auch R\"uckgabewerte besitzen, womit sie sich sehr gut f\"ur sich wiederholende Programmabl\"aufe eignet.
\medskip
\begin{parts}
\item[(a)] Erstellen Sie eine Funktion vom Type \code{void}, die \code{Hello World!} auf der Konsole ausgibt.
\item[(b)] Erstellen Sie eine Funktion vom Type \code{int}, die auf eine gegebene Zahl eins addiert und das Ergebnis zur\"uck gibt.
\item[(c)] Erstellen Sie eine Funktion vom type \code{double}, die die Sume zweier Gleitkommazahlen zur"uck gibt.
\end{parts}
\end{body}


\begin{solution}
\inputcode[frame=lines,title=Funktionen.java]{\filename{Funktionen.java}}
\end{solution}

\end{exercise}
