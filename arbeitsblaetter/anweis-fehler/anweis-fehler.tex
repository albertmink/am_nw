\begin{exercise}{Kompilerfehler}
\description{Programmfehler anhand des Compilers finden}
\difficulty{einfach}
\source{unbekannt}
\utilization{SS17}
%%\tags{funktionen}
\begin{body}
Das Java-Program \code{Fehler.java} wurde kompiliert, wobei einige Fehler und Warnungen auftraten.
Finden Sie mit Hilfe der Kompilerausgaben die Fehler und notieren Sie die eine Möglichkeit den Fehler zu verbessern.
Jeder entdeckte Fehler soll in allen nachfolgenden Programmzeilen als korrigiert gelten, sodass keine
Folgefehler auftreten können.
\inputcode[frame=lines,title=Fehler.java,numbers=left]{\filename{src/Fehler.java}}

Der Kompiler hat folgende Fehler ausgegeben:

\begin{verbatim}
Fehler.java:7: error: variable j might not have been initialized
System.out.println(j);
^
Fehler.java:5: error: cannot assign a value to final variable c
c++;
^
Fehler.java:8: error: ';' expected
System.out.println(j)
^

Fehler.java:9: error: bad operand types for binary operator '+'
j = a+1;
^
first type:  int[]
second type: int

Fehler.java:10: error: '.class' expected
fo (int i = 0; i < j; i++) {
^
Fehler.java:10: error: > expected
fo (int i = 0; i < j; i++) {
^
Fehler.java:10: error: not a statement
fo (int i = 0; i < j; i++) {
^
Fehler.java:10: error: ';' expected
fo (int i = 0; i < j; i++) {
^
required: double,double
found: int
reason: actual and formal argument lists differ in length
Fehler.java:14: error: variable a is already defined in method main(String[])
int a = 0;
\end{verbatim}
\end{body}
\begin{solution}
  \inputcode[frame=lines,title=fehler-korr.java]{\filename{src/Corr.java}}
\end{solution}
\end{exercise}
