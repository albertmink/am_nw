\begin{exercise}{Literale}

\begin{body}
Nachfolgend finden Sie Beispiele von Java-\emph{Literalen}, d.h. von Zeichenfolgen die in einem Java-Quelltext konstante Werte repräsentieren. Geben Sie jeweils den Datentyp des Literals an. Mögliche Datentypen sind
\code|int|, \code|long|, \code|double|, \code|float|, \code|boolean|, \code|char| und \code|String|.
\begin{center}
\begin{minipage}{0.3\textwidth}
\begin{itemize}
\item[(a)] \code|"Hello!"|
\item[(b)] \code|-156|
\item[(c)] \code|1E6|
\item[(d)] \code|""|
\end{itemize}
\end{minipage}
\begin{minipage}{0.3\textwidth}
\begin{itemize}
\item[(e)] \code|0.5|
\item[(f)] \code|'0'|
\item[(g)] \code|1.5e-1|
\item[(h)] \code|124f|
\end{itemize}
\end{minipage}
\begin{minipage}{0.3\textwidth}
\begin{itemize}
\item[(i)] \code|"13"|
\item[(j)] \code|true|
\item[(k)] \code|'a'|
\item[(l)] \code|.2|
\end{itemize}
\end{minipage}
\end{center}
\end{body}

\begin{solution}
\begin{center}
\begin{minipage}{0.3\textwidth}
\begin{itemize}
\item[(a)] \code|String| \\ (Hello!)
\item[(b)] \code|int|    \\ ($-156$)
\item[(c)] \code|double| \\ ($1000000$)
\item[(d)] \code|String| \\ (leere Zeichenkette)
\end{itemize}
\end{minipage}
\begin{minipage}{0.3\textwidth}
\begin{itemize}
\item[(e)] \code|double| \\ ($0{,}5$)
\item[(f)] \code|char|   \\ (Das Zeichen \glqq 0\grqq)
\item[(g)] \code|double| \\ ($0{,}15$)
\item[(h)] \code|float|  \\ ($124$)
\end{itemize}
\end{minipage}
\begin{minipage}{0.3\textwidth}
\begin{itemize}
\item[(i)] \code|String| \\ (die Zeichenkette \glqq 13\grqq)
\item[(j)] \code|bool| \\ (\code|true|)
\item[(k)] \code|char|    \\ (a)
\item[(l)] \code|double|  \\ ($0{,}2$)
\end{itemize}
\end{minipage}
\end{center}
\end{solution}

\end{exercise}
