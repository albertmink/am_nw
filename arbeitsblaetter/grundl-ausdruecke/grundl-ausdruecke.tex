\begin{exercise}{Ausdrücke}

\begin{body}
Nachfolgend finden Sie Beispiele von Java-Ausdrücken. Geben Sie jeweils das Ergebnis des Ausdrucks sowie den 
Datentyp des Ergebnisses an. Mögliche Datentypen sind \code|int|, \code|double|, \code|boolean|, \code|char| und \code|String|.
\begin{center}
\begin{minipage}{0.3\textwidth}
\begin{itemize}
\item[(a)] \code|12.3 + 1|
\item[(b)] \code|0.5 * 4|
\item[(c)] \code|2.0 / 4.0|
\item[(d)] \code|(2-1.0)/5| 
\item[(e)] \code|14.0 % 5|
\end{itemize}
\end{minipage}
\begin{minipage}{0.3\textwidth}
\begin{itemize}
\item[(f)] \code|3 / 6.0|
\item[(g)] \code|(int) 1.23|
\item[(h)] \code|2 / (int) 3.14|
\item[(i)] \code|1.0 / 4|
\item[(j)] \code|15 % 2|
\end{itemize}
\end{minipage}
\begin{minipage}{0.3\textwidth}
\begin{itemize}
\item[(k)] \code|1>2|
\item[(l)] \code|1 / 2|
\item[(m)] \code|(1+1)>1|
\end{itemize}
\end{minipage}
\end{center}
\end{body}

\begin{solution}

\begin{center}
\begin{minipage}{0.3\textwidth}
\begin{itemize}
\item[(a)] $13{,}3$  \\ \code|double|
\item[(b)] $2$       \\ \code|double|
\item[(c)] $0{,}5$   \\ \code|double|
\item[(d)] $0{,}2$   \\ \code|double|
\item[(e)] $4$ ($14 = 2\cdot 5 + \boldsymbol{4}$) \\ \code|double|
\end{itemize}
\end{minipage}
\begin{minipage}{0.3\textwidth}
\begin{itemize}
\item[(f)] $0{,}5$   \\ \code|double|
\item[(g)] $1$ (ganzzahliger Anteil) \\ \code|int|
\item[(h)] $0$               \\ \code|int|
\item[(i)] $0{,}25$  \\ \code|double|
\item[(j)] $1$ ($15 = 7\cdot 2 + \boldsymbol{1}$) \\ \code|int|
\end{itemize}
\end{minipage}
\begin{minipage}{0.3\textwidth}
\begin{itemize}
\item[(k)] \code|false|          \\ \code|bool|
\item[(l)] $0$  (Ganzzahldivision) \\ \code|int|
\item[(m)] \code|true|            \\ \code|bool|
\end{itemize}
\end{minipage}
\end{center}
\end{solution}

\end{exercise}
