\begin{exercise}{Felder}
\begin{body}
Betrachten Sie die folgenden Ausschnitte eines Java-Programms:
\begin{parts}
\item 
\begin{displaycode}
    int[] feld = {1, 2, 3, 4, 5};
    for (int i = 1; i < feld.length; i++) {
         feld[i] += feld[i-1];
    }
\end{displaycode}

\item
\begin{displaycode}
    int[] feld = {1, 4, 9, 16, 25};
    for (int i = feld.length-1; i > 0; i--) {
        feld[i] -= feld[i-1];
    }
\end{displaycode}


\item
\begin{displaycode}
    int[] feld = {5, 4, 3, 2, 1};
    for (int i = 0; i < feld.length; i++) {
        feld[i] = feld[feld.length-i-1];
    }
\end{displaycode}

\item
\begin{displaycode}
    int[] feld = {1, 2, 3, 4, 5};
    for (int i = 0; i < feld.length; i++) {
        feld[i] *= (i+1);
    }
\end{displaycode}
\end{parts}
Was wird jeweils auf der Konsole ausgegeben, wenn unmittelbar nach jedem Programmausschnitt die folgenden Zeilen
ausgeführt werden?
\medskip
\begin{displaycode}
    for (int i = 0; i < feld.length; i++) {
        System.out.print(feld[i] + ", ");
    }
\end{displaycode}
\end{body}


\begin{solution}
\begin{parts}
\item
Es wird \glqq 1, 3, 6, 10, 15, \grqq\ auf der Konsole ausgegeben.

\item
Es wird \glqq 1, 3, 5, 7, 9, \grqq\ auf der Konsole ausgegeben.

\item
Es wird \glqq 1, 2, 3, 2, 1, \grqq\ auf der Konsole ausgegeben.

\item
Es wird \glqq 0, 0, 3, 3, 3, \grqq\ auf der Konsole ausgegeben.

\item
Es wird \glqq 1, 4, 9, 16, 25, \grqq\ auf der Konsole ausgegeben.

\item
Es wird \glqq 5, 4, 3, 2, 1, \grqq\ auf der Konsole ausgegeben.
\end{parts}
\end{solution}
\end{exercise}
