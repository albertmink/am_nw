\begin{exercise}{Normalisierte Darstellung}

\begin{body}
Jeder positive Binärbruch besitzt eine so genannte \emph{normalisierte Darstellung} von der Form
\[  1{,}m_{-1}m_{-2}m_{-3}\dotsm\; \cdot 10_2^e,   \]
mit \emph{Ziffern} $m_{-k} \in \{0,1\}$ und einem \emph{Exponenten} $e \in \mathbb{Z}$ zur Basis $10_2 = 2$. Ähnlich wie bei Dezimalbrüchen erhält man die normalisierte Darstellung bei Binärbrüchen durch Verschieben des Kommas und entsprechender Wahl des Exponenten. Als Beispiel betrachten wir den Binärbruch $0{,}0011_2$. Seine normalisierte Darstellung $1{,}1_2 \cdot 10_2^{-3}$ erhält man, indem man das Komma um $3$ Stellen \emph{nach rechts} verschiebt und den Exponenten $-3$ wählt. Die normalisierte Form des Binärbruchs $101{,}1_2$ ist $1{,}011_2 \cdot 10_2^{2}$. Sie entsteht, indem man das Komma um $2$ Stellen \emph{nach links} verschiebt und den Exponenten $2$ wählt.  

Geben Sie die normalisierte Darstellung der nachfolgenden Binärbrüche an.
\begin{center}
\begin{minipage}{0.3\textwidth}
\begin{itemize}
\item[(a)] $0{,}1_2$
\item[(b)] $0{,}01_2$
\item[(c)] $0{,}001_2$
\item[(d)] $0{,}0001_2$
\item[(e)] $10{,}0_2$
\end{itemize}
\end{minipage}
\begin{minipage}{0.3\textwidth}
\begin{itemize}
\item[(f)] $100{,}0_2$
\item[(g)] $1000{,}0_2$
\item[(h)] $0{,}00101_2$
\item[(i)] $10{,}01_2$
\item[(j)] $111{,}11_2$
\end{itemize}
\end{minipage}
\begin{minipage}{0.3\textwidth}
\begin{itemize}
\item[(k)] $0{,}011_2$
\item[(l)] $101{,}0_2$
\item[(m)] $1{,}011_2$
\item[(n)] $1001_2$
\item[(o)] $1_2$
\end{itemize}
\end{minipage}
\end{center}
\end{body}

\begin{solution}

\begin{center}
\begin{minipage}{0.3\textwidth}
\begin{itemize}
\item[(a)] $0{,}1_2    = 1{,}0_2 \cdot 10_2^{-1}$
\item[(b)] $0{,}01_2   = 1{,}0_2 \cdot 10_2^{-2}$
\item[(c)] $0{,}001_2  = 1{,}0_2 \cdot 10_2^{-3}$
\item[(d)] $0{,}0001_2 = 1{,}0_2 \cdot 10_2^{-4}$
\item[(e)] $10{,}0_2   = 1{,}0_2 \cdot 10_2^{1}$
\end{itemize}
\end{minipage}
\begin{minipage}{0.3\textwidth}
\begin{itemize}
\item[(f)] $100{,}0_2   = 1{,}0_2    \cdot 10_2^{2}$
\item[(g)] $1000{,}0_2  = 1{,}0_2    \cdot 10_2^{3}$
\item[(h)] $0{,}00101_2 = 1{,}01_2   \cdot 10_2^{-3}$
\item[(i)] $10{,}01_2   = 1{,}001_2  \cdot 10_2^{1}$
\item[(j)] $111{,}11_2  = 1{,}1111_2 \cdot 10_2^{2}$
\end{itemize}
\end{minipage}
\begin{minipage}{0.3\textwidth}
\begin{itemize}
\item[(k)] $0{,}011_2 = 1{,}1_2   \cdot 10_2^{-2}$
\item[(l)] $101{,}0_2 = 1{,}01_2  \cdot 10_2^{2}$
\item[(m)] $1{,}011_2 = 1{,}011_2 \cdot 10_2^{0}$
\item[(n)] $1001_2    = 1{,}001_2 \cdot 10_2^{3}$
\item[(o)] $1_2       = 1{,}0_2   \cdot 10_2^{0}$
\end{itemize}
\end{minipage}
\end{center}
\end{solution}

\end{exercise}
