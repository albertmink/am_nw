\begin{exercise}{Rundungsregel nach IEEE*}

\begin{body}
Jeder Gleitkomma-Datentyp kann nur eine bestimmte Anzahl $N$ von Nachkommastellen eines Binärbruchs (in normalisierter Darstellung) darstellen. Kann ein Binärbruch nicht mit $N$ Nachkommastellen dargestellt werden, wird dieser auf $N$ Nachkommastellen gerundet. Dies geschieht nach der \emph{IEEE-Rundungsregel} (IEEE: Institute of Electrical and Electronics Engineers). Diese ist folgendermaßen definiert:
\begin{itemize}
\item
Jeder Binärbruch wird auf den nächstgelegenen darstellbaren Binärbruch gerundet (\glqq round to nearest\grqq).

\item
Liegt der Binärbruch genau in der Mitte zwischen zwei darstellbaren Binärbrüchen, so wird er auf denjenigen darstellbaren Binärbruch gerundet, dessen letzte Stelle eine Null ist (\glqq round to nearest even\grqq).
\end{itemize}
Betrachten wir dazu das folgende Beispiel: Der Bruch $1{,}101_2 = 1{,}625$ soll nach der IEEE-Rundungsregel auf zwei darstellbare Nachkommastellen gerundet werden. Der Bruch liegt genau in der Mitte zwischen den darstellbaren Brüchen $1{,}10_2 = 1{,}5$ und $1{,}11_2 = 1{,}75$. Laut IEEE-Rundungsregel wird er auf den darstellbaren Bruch gerundet, dessen letzte Stelle eine Null ist. Dies ist der Bruch $1{,}10_2 = 1{,}5$.

Runden Sie die nachfolgenden Binärbrüche nach der IEEE-Rundungsregel auf vier Nachkommastellen. Ein Querstrich bezeichnet sich periodisch wiederholende Nachkommastellen.
\begin{center}
\begin{minipage}{0.3\textwidth}
\begin{parts}
\item[(a)] $1{,}00001_2$
\item[(b)] $1{,}00011_2$
\item[(c)] $1{,}100011_2$
\end{parts}
\end{minipage}
\begin{minipage}{0.3\textwidth}
\begin{parts}
\item[(d)] $1{,}01111_2$
\item[(e)] $1{,}0100101_2$
\item[(f)] $1{,}01101_2$
\end{parts}
\end{minipage}
\begin{minipage}{0.3\textwidth}
\begin{parts}
\item[(g)] $1{,}\overline{01}_2$
\item[(h)] $1{,}\overline{1100}_2$
\item[(i)] $1{,}0\overline{1}_2$
\end{parts}
\end{minipage}
\end{center}
\end{body}


\begin{solution}
\begin{center}
\begin{minipage}{0.3\textwidth}
\begin{parts}
\item[(a)] $1{,}0000_2 $
\item[(b)] $1{,}0010_2$
\item[(c)] $1{,}1001_2$
\end{parts}
\end{minipage}
\begin{minipage}{0.3\textwidth}
\begin{parts}
\item[(d)] $1{,}1000_2$
\item[(e)] $1{,}0101_2$
\item[(f)] $1{,}0110_2$
\end{parts}
\end{minipage}
\begin{minipage}{0.3\textwidth}
\begin{parts}
\item[(g)] $1{,}0101_2$
\item[(h)] $1{,}1101_2$
\item[(i)] $1{,}1000_2$
\end{parts}
\end{minipage}
\end{center}
\end{solution}
\end{exercise}
