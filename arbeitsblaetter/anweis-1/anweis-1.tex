\begin{exercise}{Anweisungen}
\begin{body}
Gegeben sei der folgende Ausschnitt eines Java-Programms. Dabei sei \code|n| eine Variable vom Typ~\code|int|.
\begin{displaycode}
if ( n < 3 ) {
    if ( n > 1 ) {
        System.out.println("A");
    } else {
        System.out.println("B");
    }
} else {
    if ( n <= 3 ) {
        System.out.println("C");
    } else {
        System.out.println("D");
    }
}
\end{displaycode}
\begin{parts}
\item[(a)] Welcher Buchstabe wird auf dem Bildschirm ausgegeben, falls \code|n| den Wert $1$, $2$, $3$ bzw.~$4$ besitzt?
\item[(b)] Welcher Buchstabe wird auf dem Bildschirm ausgegeben, falls \code|n| den Wert $1$, $2$, $3$ bzw.~$4$ besitzt und \textbf{jeder} Boolsche Ausdruck in dem Programmabschnitt durch seine Negation ersetzt wurde?
\end{parts}
\end{body}


\begin{solution}
\begin{parts}
\item[(a)] Für den vorliegenden Programmausschnitt erhält man
\begin{center}
\begin{tabular}{|c|c|}
\hline
$\quad$\code|n|$\quad$  & \textbf{Ausgabe} \\
\hline
$1$ & B \\
$2$ & A \\
$3$ & C \\
$4$ & D \\
\hline
\end{tabular}
\end{center}

\item[(b)] Ersetzt man die Ausdrücke \code|(n < 3)|, \code|(n > 1)| und \code|(n <= 3)| durch ihre Negationen \code|(n >= 3)|, \code|(n <= 1)| und \code|(n > 3)|, so erhält man
\begin{center} 
\begin{tabular}{|c|c|}
\hline
$\quad$\code|n|$\quad$  & \textbf{Ausgabe} \\
\hline
$1$ & D \\
$2$ & D \\
$3$ & B \\
$4$ & B \\
\hline
\end{tabular}
\end{center}
\end{parts}
\end{solution}
\end{exercise}
