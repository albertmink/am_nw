\begin{exercise}{Rekursion}
\begin{body}
Betrachten Sie die nachfolgende Definition einer Klassenmethode in der Programmiersprache Java.
\begin{displaycode}
    static void schachtelung(double x, double a, double b) {
        if ( b - a <= 0.25 ) {
            System.out.println(a + "; " + b);
        } else {
            double c = 0.5 * (a + b);
            if ( x < c ) {
                schachtelung(x,a,c);
            } else {
                schachtelung(x,c,b);
            }
        }
    }
\end{displaycode}
\begin{parts}
\item
Wie oft wird die Methode durch den Aufruf \code|schachtelung(0.6,0.0,1.0);| aufgerufen? Was wird in diesem Fall auf der Konsole ausgegeben?

\item
Geben Sie eine äquivalente Definition der Methode ohne rekursiven Aufruf an.
\end{parts}
\end{body}

\begin{solution}
\begin{parts}
\item
Die Methode wird dreimal aufgerufen. Es wird \glqq$0{,}5$; $0{,}75$\grqq\ auf der Konsole ausgegeben.

\item
Eine mögliche Klassendefinition ohne rekursiven Aufruf lautet
\begin{displaycode}
    static void schachtelung(double x, double a, double b) {
        double c;
        while ( b - a > 0.25  ) {
            c = 0.5 * (a + b);
            if ( x > c ) {
                a = c;
            } else {
                b = c;
            }
        }
        System.out.println(a + "; " + b);
    }
\end{displaycode}
\end{parts}
\end{solution}
\end{exercise}
