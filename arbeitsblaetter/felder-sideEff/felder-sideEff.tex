\begin{exercise}{Seiteneffekte}
\begin{body}
Betrachten Sie das folgende Java-Programm mit den Klassenmethoden \code{computeNorm} die sich sowohl im Datentype des \"Ubergabeparameters und des R\"uckgabeparameters unterscheiden.

\inputcode[frame=lines,title=FelderFunktionen.java,numbers=left]{\filename{FelderFunktionen.java}}

\begin{parts}
\item Die Klassenmethode \code{computeNorm} mit R\"uckgabewert \code{int} berechnet das Quadrat des \"Ubergabeparameters vom Datentyp \code{int}.
In der Klassenmethode wird eine Kopie des \"Ubergabeparameters angelegt, damit wird der Wert von \code{a} nicht ver\"andert.
Besser ist schlicht \code{return n*n;} auszuf\"uhren.
\item Nun wird ein Feld an die Klassenmethode \code{computeNorm} \"ubergeben.
Nach dem Aufruf der Klassenmethode ist auch der \"Ubergabeparameter ver\"andert.
Erkl\"aren Sie dieses Verhalten, Stichwort \emph{Seiteneffekt}.
\item Implementieren Sie eine sichere \code{computeNorm} die den \"Ubergabeparameter nicht ver\"andert.
\end{parts}
\end{body}

\begin{solution}
\begin{parts}
\item Nichts zu tun.
\item In Klassenmethode \code{computeNorm} mit \"Ubergabeparameter \code{int[]} wird keine Kopie angelegt.
Das erlaubt es wiederrum direkt in das \"ubergebene Feld \code{d} zuschrieben und den urspr\"unglichen Wert zu ver\"andern.
\item Sichere Implementierung ohne Seiteneffekte.
\inputcode[frame=lines,title=FelderFunktionen.java,numbers=left]{\filename{FelderFunktionenSol.java}}
\end{parts}
\end{solution}
\end{exercise}
