
\begin{exercise}{Stellenwertsysteme}
\begin{body}
Bei \emph{Stellenwertsystemen} wird eine Zahl durch eine Folge ${z_n z_{n-1} \dotsb z_1 z_0}$ von \emph{Ziffern} $z_k$ dargestellt. Jedes Stellenwertsystem bezieht sich dabei auf eine bestimmte \emph{Basis} $b$, die man bei Bedarf als Subskript an das Ende der Ziffernfolge schreibt. Für die Ziffern gilt $z_k \in \{0,1,\dotsc,b-1\}$. Eine Ziffernfolge
${z_n z_{n-1} \dotsb z_1 z_0}$ im Stellenwertsystem zur Basis $b$ stellt die Zahl
\[  z_n \cdot b^{n} + z_{n-1} \cdot b^{n-1} + \dotsb + z_1 \cdot b^1 + z_0 \cdot b^0  \]
dar. Wichtige Stellenwertsysteme sind das \emph{Dezimalsystem} ($b = 10$), das \emph{Binärsystem} ($b = 2$), das \emph{Oktalsystem} ($b = 8$) und das \emph{Hexadezimalsystem} ($b = 16$). Geben Sie die Darstellung der folgenden  Binär-, Oktal-, Hexadezimal- und Dezimalzahlen in den jeweils anderen Stellenwertsystemen an. Im Hexadezimalsystem bezeichnen die Buchstaben $\mathrm{A},\mathrm{B}, \dotsc, \mathrm{F}$ die Ziffern $10, 11, \dotsc, 15$.
\begin{center}
\begin{minipage}{0.22\textwidth}
\begin{parts}
\item[(a)] $10_2$
\item[(b)] $1100_2$
\item[(c)] $10101_2$
\item[(d)] $101010_2$
\end{parts}
\end{minipage}
\begin{minipage}{0.22\textwidth}
\begin{parts}
\item[(e)] $27_8$
\item[(f)] $133_8$
\item[(g)] $10_8$
\item[(h)] $77_8$
\end{parts}
\end{minipage}
\begin{minipage}{0.22\textwidth}
\begin{parts}
\item[(i)] $\mathrm{2A}_{16}$
\item[(j)] $10_{16}$
\item[(k)] $\mathrm{FF}_{16}$
\item[(l)] $\mathrm{D1}_{16}$
\end{parts}
\end{minipage}
\begin{minipage}{0.22\textwidth}
\begin{parts}
\item[(m)] $17_{10}$
\item[(n)] $33_{10}$
\item[(o)] $65_{10}$
\item[(p)] $72_{10}$
\end{parts}
\end{minipage}
\end{center}
\end{body}


\begin{solution}
\begin{center}
\begin{minipage}{0.45\textwidth}
\begin{parts}
\item[(a)] $10_2 = 2_8 = 2_{10} = 2_{16}$
\item[(b)] $1100_2 = 14_8 = 12_{10} = \mathrm{C}_{16}$
\item[(c)] $10101_2 = 25_8 = 21_{10} = 15_{16}$
\item[(d)] $101010_2 = 52_8 = 42_{10} = \mathrm{2A}_{16}$)
\item[(e)] $27_8 = 10111_2 = 23_{10} = 17_{16}$
\item[(f)] $133_8 = 1011011_2 = 91_{10} = \mathrm{5B}_{16}$
\item[(g)] $10_8 = 1000_2 = 8_{10} = 8_{16}$
\item[(h)] $77_8 = 111111_2 = 63_{10} = \mathrm{3F}_{16}$
\end{parts}
\end{minipage}
\begin{minipage}{0.45\textwidth}
\begin{parts}
\item[(i)] $\mathrm{2A}_{16} = 101010_2 = 52_8 = 42_{10}$  
\item[(j)] $10_{16} = 10000_2 = 20_8 = 16_{10}$           
\item[(k)] $\mathrm{FF}_{16} = 11111111_2 = 377_8 = 255_{10}$ 
\item[(l)] $\mathrm{D1}_{16} = 11010001_2 = 321_8 = 209_{10}$ 
\item[(m)] $17_{10} = 10001_2 = 21_8 = 11_{16}$  
\item[(n)] $33_{10} = 100001_2 = 41_8 = 21_{16}$           
\item[(o)] $65_{10} = 1000001_2 = 101_8 = 41_{16}$ 
\item[(p)] $72_{10} = 1001000_2 = 110_8 = 48_{16}$ 
\end{parts}
\end{minipage}
\end{center}
\end{solution}
\end{exercise}