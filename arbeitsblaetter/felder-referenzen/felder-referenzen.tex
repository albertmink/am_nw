\begin{exercise}{Referenzen}
\begin{body}
Betrachten Sie die folgende \code|main|-Methode eines Java-Programms.
\medskip
\begin{displaycode}
    public static void main(String[] args) {
        int[] feld1 = {1};
        int[] feld2 = methode(feld1);
        System.out.print(feld1[0] + ", " + feld2[0]);
    }
\end{displaycode}
\medskip
\noindent
Die Klassenmethode \code|methode| wird im folgenden auf drei verschiedene Weisen definiert:
\begin{parts}
\item
\begin{displaycode}
    static int[] methode(int[] feld) {
        feld[0] = feld[0] + 1;
        return feld;
    }
\end{displaycode}

\item
\begin{displaycode}
    static int[] methode(int[] feld) {
        int[] hilfsfeld = feld;
        hilfsfeld[0] = feld[0] + 1;
        return hilfsfeld;
    }
\end{displaycode}

\item
\begin{displaycode}
    static int[] methode(int[] feld) {
        int[] hilfsfeld = new int [1];
        hilfsfeld[0] = feld[0] + 1;
        return hilfsfeld;
    }
\end{displaycode}
\end{parts}
Geben Sie für jede Methodendefinition an, was beim Ausführen des Programms auf der Konsole ausgegeben wird.
\end{body}

\begin{solution}
\begin{parts}
\item
Es wird \glqq 2, 2\grqq\ auf der Konsole ausgegeben.

\item
Es wird \glqq 2, 2\grqq\ auf der Konsole ausgegeben.

\item
Es wird \glqq 1, 2\grqq\ auf der Konsole ausgegeben.

\end{parts}
\end{solution}
\end{exercise}
