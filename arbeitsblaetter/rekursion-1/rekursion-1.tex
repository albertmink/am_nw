\begin{exercise}{Rekursion}
\begin{body}
Nachfolgend ist die Definition einer Klassenmethode gegeben, die zu einer Zahl $n \in \mathbb{N}$ die Fakultät $n! := n (n-1) (n-2)  \dotsm 1$ rekursiv berechnet.
\medskip
\begin{displaycode}
    static int fakultaet(int n) {
        if (n <= 1) {
            return 1;
        } else {
            return n * (fakultaet(n-1));
        }
    }
\end{displaycode}
\medskip
\noindent
\begin{parts}
\item
Wie oft wird die Methode durch den Aufruf \code|fakultaet(4);| aufgerufen?

\item
Geben Sie eine äquivalente Definition der Methode ohne rekursiven Aufruf an.
\end{parts}
\end{body}

\begin{solution}
\begin{parts}
\item
Die Methode wird viermal aufgerufen.

\item
Eine mögliche Klassendefinition ohne rekursiven Aufruf lautet
\medskip
\begin{displaycode}
    static int fakultaet(int n) {   
        if (n <= 1) {
            return 1;
        } else {
            int fak = 1;
            for (int i = 2; i <= n; i++) {
                fak *= i;
            }
            return fak;
        }
    }
\end{displaycode}
\end{parts}
\end{solution}
\end{exercise}
