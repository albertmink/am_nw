\documentclass[9pt,german]{beamer}%
\usepackage{master/templates/beamerthemeKIT}
\usepackage{master/templates/exercises}
%\documentclass[9pt,trans]{beamer}%

%%%%%%%%%%%%%%%%%%%%%%%%%%%%%%%%%%%%%%%%%%%%%%%%%%%%%%%%%%%%%%%%%%%%%%%%
% Packages
%%%%%%%%%%%%%%%%%%%%%%%%%%%%%%%%%%%%%%%%%%%%%%%%%%%%%%%%%%%%%%%%%%%%%%%%
%\usepackage{beamerthemeKIT}
\usepackage[utf8]{inputenc}
% \usepackage[german]{babel}
\usepackage{helvet}
\usepackage[T1]{fontenc}
\usepackage{amsmath, amsthm, amssymb}
\usepackage{graphicx}
\usepackage{listings}
\usepackage{hyperref}
%%%%%%%%%%%%%%%%%%%%%%%%%%%%%%%%%%%%%%%%%%%%%%%%%%%%%%%%%%%%%%%%%%%%%%%%
% Layout
%%%%%%%%%%%%%%%%%%%%%%%%%%%%%%%%%%%%%%%%%%%%%%%%%%%%%%%%%%%%%%%%%%%%%%%%
\newcommand\EMPH[1]{\textcolor{blue}{#1}}%
\newcommand\WARN[1]{\textcolor{red}{#1}}%
\renewcommand{\rmdefault}{phv}
\newlength{\movetitlegrafics}
\setlength{\movetitlegrafics}{0.0\paperwidth}
%%%%%%%%%%%%%%%%%%%%%%%%%%%%%%%%%%%%%%%%%%%%%%%%%%%%%%%%%%%%%%%%%%%%%%%%
% Definitions
%%%%%%%%%%%%%%%%%%%%%%%%%%%%%%%%%%%%%%%%%%%%%%%%%%%%%%%%%%%%%%%%%%%%%%%%
%%% Counter
\newcounter{ctlit}%
\setcounter{ctlit}{1}%
\newcounter{kap}
%%% Colors
\definecolor{blue}{rgb}{0.1,0.1,1.0}%
\definecolor{green}{rgb}{0.0,0.58,0.49}%
\definecolor{grey}{rgb}{0.4,0.4,0.4}%
\definecolor{pink}{rgb}{0.8,0.1,0.2}%
\definecolor{red}{rgb}{1.0,0.1,0.1}%
\definecolor{steel blue}{rgb}{0.274510,0.509804,0.705882}
\definecolor{steelblue}{rgb}{0.274510,0.509804,0.705882}
\def\GREY#1{{\color{grey} #1}}%
%%% Textfarben
\newcommand\BLUE[1]{\textcolor{blue}{#1}}%
\newcommand\BLACK[1]{\textcolor{black}{#1}}%
\newcommand\GREEN[1]{\textcolor{green}{#1}}%
\newcommand\LBLUE[1]{\textcolor{LightBlue}{#1}}%
\newcommand\RED[1]{\textcolor{red}{#1}}%
%%% Layout
\def\cmt#1{\BLUE{#1}}%
\def\code#1{\texttt{#1}}%
\def\codeblock#1{\GREEN{\texttt{#1}}}%
\def\ftext#1{\textbf{{#1}}}%
\def\headline#1{\large\textbf{\GREEN{#1}}}%
\def\KITitem{\KITBullet[2mm]\ }%
\def\q{\quad}%
\def\qq{\qquad}%
\def\subtitle#1{{\small #1}}%
\def\tab{\hspace*{0.5cm}}%
\def\3{{\ss}}%
\def\IM{\mathbb{M}}%
\def\IN{\mathbb{N}}%
\def\IR{\mathbb{R}}%
\def\IZ{\mathbb{Z}}%
\def\Id{Id}%
\def\ie{{\rm i}}%
\newcommand{\entspr}{$\stackrel{\wedge}{=}$}%
\renewcommand{\Phi}{\varPhi}%
\renewcommand{\Psi}{\varPsi}%
%%% Standards
%\input{D:/TeX/formate/mystyle/defs}%
%\input{/home/ae01/tex/formate/mystyle/defs}%
%%% Text
\def\3{{\ss}}%
%%% Symbols
\def\bs{$\backslash$}%
\def\hpm{\hphantom{$-$}}%
\def\meps{\epsilon}%
%%% code environment
\lstset{
  showstringspaces=false,
  numbers=none,
  keywordstyle=\color{green},  % coloring and formatting of keywords as public, class, import 
  commentstyle=\color{blue}\small\ttfamily, % Color of comments
  stringstyle=\color{pink},
  breaklines=true
}

\lstdefinestyle{JAVA}
{
  language=Java,
  basicstyle=\ttfamily, % defines text formatting
  frame=tb  % print top and bottom lines, frame=single, L
}

\lstdefinestyle{JAVAsmall}
{
  language=Java,
  basicstyle=\small\ttfamily, % defines text formatting
  frame=tb  % print top and bottom lines, frame=single, L
}

\lstdefinestyle{JAVAlines}
{
  language=Java,
  numbers=left,
  numberstyle=\color{grey}\ttfamily,
  numbersep=4pt, % distance between numbers and code1_1
  xleftmargin=4pt, % distance from frame to listing
  xrightmargin=4pt, % distance from frame to listing
  basicstyle=\ttfamily, % defines text formatting
  keywordstyle=\color{green},  % coloring and formatting of keywords as public, class, import 
  commentstyle=\color{blue}\ttfamily, % Color of comments
  frame=tb  % print top and bottom lines, frame=single, L
}

\lstdefinestyle{JAVAsmalllines}
{
  language=Java,
  numbers=left,
  numberstyle=\color{grey}\ttfamily,
  numbersep=4pt, % distance between numbers and code1_1
  xleftmargin=4pt, % distance from frame to listing
  xrightmargin=4pt, % distance from frame to listing
  basicstyle=\small\ttfamily, % defines text formatting
  keywordstyle=\color{green},  % coloring and formatting of keywords as public, class, import 
  commentstyle=\color{blue}\ttfamily, % Color of comments
  frame=tb  % print top and bottom lines, frame=single, L
}


\lstdefinestyle{BASH}{
  basicstyle=\Large\ttfamily, % defines text formatting
  frame=none,
  language=bash
}


%%%%%%%%%%%%%%%%%%%%%%%%%%%%%%%%%%%%%%%%%%%%%%%%%%%%%%%%%%%%%%%%%%%%%%%%
% Titlepage
%%%%%%%%%%%%%%%%%%%%%%%%%%%%%%%%%%%%%%%%%%%%%%%%%%%%%%%%%%%%%%%%%%%%%%%%
\title[Institut f\"ur Angewandte und Numerische Mathematik]%
 {\fontsize{15}{15}\selectfont{}
  \"Ubung:
  \textit{Einstieg in die Informatik}\\[1.5mm]
  \textit{\phantom{\"Ubung:}\; und algorithmische Mathematik}\\[1.5mm]
  }%\hspace*{3cm}\normalsize{f\"ur Mathematiker}}
\author{\fontsize{9}{9}\selectfont{}
 Albert Mink\
  }
\institute[Institut f\"ur Angewandte und Numerische Mathematik]
 {\fontsize{6}{6}\selectfont{}%
  Institut f\"ur Angewandte und Numerische Mathematik}
\date{Wintersemester 2018/19}%
\subject{}%
%\beamerdefaultoverlayspecification{<+->}
%%%%%%%%%%%%%%%%%%%%%%%%%%%%%%%%%%%%%%%%%%%%%%%%%%%%%%%%%%%%%%%%%%%%%%%%



\makeatletter
\def\input@path{{uebungsfolien/}}
\graphicspath{{uebungsfolien/}}
\makeatother


%%%%%%%%%%%%%%%%%%%%%%%%%%%%%%%%%%%%%%%%%%%%%%%%%%%%%%%%%%%%%%%%%%%%%%%%
% Document
%%%%%%%%%%%%%%%%%%%%%%%%%%%%%%%%%%%%%%%%%%%%%%%%%%%%%%%%%%%%%%%%%%%%%%%%
\begin{document}
\maketitle%
\addtocounter{framenumber}{-1}%
%%%%%%%%%%%%%%%%%%%%%%%%%%%%%%%%%%%%%%%%%%%%%%%%%%%%%%%%%%%%%%%%%%%%%%%%

\begin{frame}
  \frametitle{Arbeitsblatt 0}%
\tableofcontents[hideallsubsections]
\end{frame}

\def\kap{1}%
%\AtBeginSection{}
%%%%%%%%%%%%%%%%%%%%%%%%%%%%%%%%%%%%%%%%%%%%%%%%%%%%%%%%%%%%%%%%%%%%%%%%
\section{Einführung}
\begin{frame}
  \frametitle{\kap. Einführung}%
\tableofcontents[current]
\end{frame}


%%%%%%%%%%%%%%%%%%%%%%%%%%%%%%%%%%%%%%%%%%%%%%%%%%%%%%%%%%%%%%%%%%%%%%%%
\def\stitle{Die \"Ubung}%
\subsection{\stitle}\label{S:Uebersicht}
\begin{frame}[fragile]%
  \frametitle{\kap.\ref{S:Uebersicht} \stitle}%
\medskip

\begin{itemize}
%  \item Klausur: 01.02.18 und 12.07.18
%  \item Vorlesung: Montags 11:30 -- 13:00
  \item Dienstags 11:30 -- 13:00
  \item Inhalte der \"Ubung
  \begin{itemize}
    \item Besprechung der Arbeitsbl\"atter (etwa 2 bis 3 Aufgaben)
    \item Rechnerdemos
    \item Beantwortung von Fragen zur Vorlesung und Praktikum
  \end{itemize}
  \item Arbeitsbl\"atter werden immer eine Woche vor der \"Ubung hochgeladen
\end{itemize}
\medskip

\begin{description}[leftmargin=*,style=nextline]
  \item[\textcolor{black}{\textbf{Ansprechpartner}}]
  \item[\"Ubung] Albert.Mink@kit.edu, Raum 210 Geb. 30.70,\\ Sprechstunde: nach Vereinbarung
  \item[\"Ubung und Praktikum] Zoltan.Veszelka@kit.edu, Raum 3.014 Geb. 20.30,\\ Sprechstunde: Di, 14:00 -- 15:00
  \item[Praktikum] Ihre Tutoren im Praktikum
\end{description}
\end{frame}


%%%%%%%%%%%%%%%%%%%%%%%%%%%%%%%%%%%%%%%%%%%%%%%%%%%%%%%%%%%%%%%%%%%%%%%%
\def\stitle{Bearbeitung der Praktikumsaufgaben}%
\subsection{\stitle}\label{S:Praktikum}
\begin{frame}[fragile]%
  \frametitle{\kap.\ref{S:Praktikum} \stitle}%
\medskip

\begin{itemize}
  \item Praktikumsaufgaben k\"onnen zu Hause oder im Praktikum bearbeitet werden
  \item Vorbereitung \textbf{vor} dem Praktikum ist obligatorisch
  \item Tutoren \textbf{unterst\"utzen} Sie bei Fragen und auftretenden Schwierigkeiten
  \item Sie erhalten Testate f\"ur Pflichtaufgaben die alle nachfolgenden Bedingungen erf\"ullen:
  \begin{itemize}
    \item Der Bearbeitungszeitraum ist nicht \"uberschritten (Ausnahmen sind \textbf{nur} in begr\"undeten F\"allen m\"oglich, z.B. Krankheit mit Attest)
    \item Ihr Programm enth\"alt keine \emph{syntaktischen} Fehler (fehlerfrei kompilier- und ausf\"uhrbar)
    \item Ihr Programm enth\"alt keine \emph{semantischen} Fehler (es liefert die korrekten Ergebnisse)
    \item Sie k\"onnen Ihr Programm \emph{live} kompilieren, ausf\"uhren \textbf{und} erkl\"aren
  \end{itemize}
\end{itemize}
\medskip

\begin{center}
 \LARGE{Voraussetzung f\"ur die Zulassung zur Klausur:\\ Erwerb \textbf{aller} Testate}
\end{center}
\medskip

Siehe auch: Merkblatt zum Praktikum (auf ILIAS) \url{https://ilias.studium.kit.edu}
\end{frame}


%%%%%%%%%%%%%%%%%%%%%%%%%%%%%%%%%%%%%%%%%%%%%%%%%%%%%%%%%%%%%%%%%%%%%%%%
\def\stitle{Einordnung und Zusammenfassung}%
\subsection{\stitle}\label{S:Einordnung}
\begin{frame}[fragile]%
  \frametitle{\kap.\ref{S:Einordnung} \stitle}%
  \medskip

  \small
  \centering
  \begin{tabular}{p{2cm}|p{2.0cm}|p{2.7cm}|p{3.0cm}}
  & {\centering Vorlesung} & {\centering \"Ubung} & {\centering Praktikum}\\
  \hline
  \hline & & &  \\
  Inhalte & Vermittlung des Vorlesungsstoffes & Vertiefung und Wiederholung des Vorlesungsstoffes, Besprechung der Arbeitsbl\"atter, Kl\"arung offener Fragen & Selbst\"andiges Programmieren und Abgabe von Testaten (Unterst\"utzt durch Ihre Tutoren)\\
  \hline & & & \\
  Materialien & Folien & Arbeitsbl\"atter, Folien & Aufgabenbl\"atter\\
  \hline & & & \\
  Be\-ar\-bei\-tungs\-zeit\-raum & & 1 Woche\newline (ab Dienstag) & 2 Wochen\newline  (ab Montag)\\
  \hline & & & \\
  Klau\-sur\-vo\-r\-aus\-set\-zung & & & Fristgerechte und erfolgreiche Abgabe \textbf{aller} Pflichtaufgaben\\
  %  \hline & & & \\
  %  aktive Mitarbeit & wenig & mittel & viel \\
  \end{tabular}
  \medskip

  \textbf{Achtung:} Die ersten Tutorien finden bereits ab Montag, den 22. Oktober statt!
\end{frame}

\def\kap{2}%
\AtBeginSection{}
%%%%%%%%%%%%%%%%%%%%%%%%%%%%%%%%%%%%%%%%%%%%%%%%%%%%%%%%%%%%%%%%%%%%%%%%
\section{Praktikum}
\begin{frame}
  \frametitle{\kap. Praktikum}%
\tableofcontents[currentsection]
\end{frame}
% \setcounter{section}{0}


%%%%%%%%%%%%%%%%%%%%%%%%%%%%%%%%%%%%%%%%%%%%%%%%%%%%%%%%%%%%%%%%%%%%%%%%
\def\stitle{Bearbeitung der Praktikumsaufgaben}%
\subsection{\stitle}\label{S:PraktikumSCC}
\begin{frame}[t]%
  \frametitle{\kap.\ref{S:PraktikumSCC} \stitle}%
\medskip

Praktikumsrechner des SCC
\begin{itemize}
  \item Linux Betriebssystem
  \item Software
  \begin{itemize}
    \item Eingabefenster (Shell, Terminal oder Konsole)
    \item Editor mit Syntax-Hervorhebung (z.B. Kate oder Gedit)
    \item Internet-Browser
    \item pdf-Betrachter
  \end{itemize}
  \item Kurze Einf\"uhrung: \emph{Anleitung und Informationen zum Praktikum mit den Sprachen C++ und Java} (auf ILIAS)
\end{itemize}
\end{frame}


%%%%%%%%%%%%%%%%%%%%%%%%%%%%%%%%%%%%%%%%%%%%%%%%%%%%%%%%%%%%%%%%%%%%%%%%
\def\stitle{Grundlagen der Linux Konsole}%
\subsection{\stitle}\label{S:Anleitung}
\begin{frame}[t]%
\frametitle{\kap.\ref{S:Anleitung} \stitle\ (1)}%

\heading{Generelle Bemerkungen zur Konsole}
\begin{itemize}
  \item Laufende Programme bzw. Befehlsausf\"uhrungen k\"onnen durch die Tastenkombination \textbf{Strg+c} abgebrochen werden
  \item F\"ur Autovervollst\"andigung von Dateinamen und Pfaden in der Konsole zwei Mal \textbf{Tab}
  \item F\"ur Hilfestellungen und Dokumentation \textbf{man} bzw. \textbf{-{}-help} und \textbf{google}
\end{itemize}

\end{frame}


%%%%%%%%%%%%%%%%%%%%%%%%%%%%%%%%%%%%%%%%%%%%%%%%%%%%%%%%%%%%%%%%%%%%%%%%
\begin{frame}[t]%
\frametitle{\kap.\ref{S:Anleitung} \stitle\ (2)}%
\medskip

\textbf{Die wichtigsten UNIX-Kommandos zum navigieren}
\begin{itemize}
  \setlength{\itemsep}{4pt}
  \item Aktuelles Verzeichnis ausgeben: \textbf{pwd (print working directory)}
  \item Verzeichnis wechseln: \textbf{cd (change directory)}
  \begin{itemize}
    \setlength{\itemsep}{2pt}
    \item \textbf{cd /tmp} Wechsel in das Verzeichnis /tmp (absoluter Pfad)
    \item \textbf{cd work/dat} Wechsel in das Unterverzeichnis dat von work (relativer Pfad)
    \item \textbf{cd} Wechsel in Ihr Home-Verzeichnis
    \item \textbf{cd ..} Wechsel in das \"ubergeordnete Verzeichnis
  \end{itemize}
  \item Inhalt des aktuellen Verzeichnisses auflisten: \textbf{ls (list)}
  \begin{itemize}
    \setlength{\itemsep}{2pt}
    \item \textbf{ls -{}- help} anzeigen der Dokumenation, alt. \textbf{man ls}
    \item \textbf{ls -l} zeigt Inhalt als Liste an. Verzeichnisse, Archive und ausf\"uhrbare Dateien werden eingef\"arbt
    \item \textbf{ls -lh} zeigt Inhalt zus\"atzlich als human readable an
    \item \textbf{ls -lhS} zeigt Inhalt zus\"atzlich der Gr\"o\ss e nach geordnet an
    \item $\ldots$
  \end{itemize}
\end{itemize}

\end{frame}


%%%%%%%%%%%%%%%%%%%%%%%%%%%%%%%%%%%%%%%%%%%%%%%%%%%%%%%%%%%%%%%%%%%%%%%%
\begin{frame}[t]%
\frametitle{\kap.\ref{S:Anleitung} \stitle\ (3)}%

\heading{Die wichtigsten UNIX-Kommandos zum erstellen und l\"oschen von Verzeichnissen}
\begin{itemize}
  \setlength{\itemsep}{4pt}
  \item Neues Verzeichnis anlegen: \textbf{mkdir (make directory)}
  \begin{itemize}
    \item \textbf{mkdir neu} legt das Verzeichnis neu im aktuellen Verzeichnis an
  \end{itemize}
  \item Neue Datei erstellen: \textbf{touch}
  \begin{itemize}
    \item \textbf{touch helloworld.java} erstellt die Datei helloworld.java
  \end{itemize}
  \item L\"oschen einer Datei: \textbf{rm (remove)}
  \begin{itemize}
    \setlength{\itemsep}{2pt}
    \item \textbf{rm helloworld.java} l\"oscht die Datei helloworld.java
    \item \textbf{rm -r helloworld.java} l\"oscht rekursiv, also auch gesamte Verzeichnisse
    \item \textbf{rm -f helloworld.java} erzwingt das L\"oschen
    \item \textbf{rm -v helloworld.java} aktiviert die Erkl\"arung was der Befehl bewirkt
  \end{itemize}
  \item Abrufen der Dokumentation: \textbf{man}
  \begin{itemize}
    \item \textbf{man rm} ruft die Dokumentation des Befehls \textbf{rm} auf. Mit \textbf{q (quit)} gelangt man zur\"uck.
  \end{itemize}
\end{itemize}

\end{frame}


%%%%%%%%%%%%%%%%%%%%%%%%%%%%%%%%%%%%%%%%%%%%%%%%%%%%%%%%%%%%%%%%%%%%%%%%
\begin{frame}[t]%
\frametitle{\kap.\ref{S:Anleitung} \stitle\ (4)}%

\heading{Die wichtigsten UNIX-Kommandos zum kopieren}
\begin{itemize}
  \setlength{\itemsep}{4pt}
  \item Befehl \textbf{cp (copy) [OPTION] <SOURCE> <DESTINATION>}
  \begin{itemize}
    \setlength{\itemsep}{2pt}
    \item \textbf{cp work.tex final.tex} erstellt eine Kopie von work.tex die final.tex hei\ss t
    \item \textbf{cp -r folderSource folderDest} kopiert rekursiv und damit auch Verzeichnisse
    \item Optionen: \textbf{-v, --verbose; -r, --recursive; -f, --force}
  \end{itemize}
\end{itemize}
\end{frame}


%%%%%%%%%%%%%%%%%%%%%%%%%%%%%%%%%%%%%%%%%%%%%%%%%%%%%%%%%%%%%%%%%%%%%%%%
\def\stitle{Programmentwicklung}%
\subsection{\stitle}\label{S:Progentw}
\begin{frame}[fragile]%
\frametitle{\kap.\ref{S:Progentw} \stitle}%

\heading{\"Ubersetzen des Programms}
\begin{itemize}
  \item \textbf{Wichtig:} Java Programme m\"ussen mit .java enden
  \item Java-Quelltext wird durch Aufruf des Java-Compilers in Java-Bytecode übersetzt \code{javac Dateiname.java}
  \item Zur Programm Ausf\"uhrung muss der Java-Interpreter aufgerufen werden \code{java Dateiname}
\end{itemize}

\heading{Beispiel: Kugelvolumen}
\begin{itemize}
  \item Der Java-Quelltext befindet sich in der Datei \code{KugelVolumen.java}
  \item Mit \code{javac KugelVolumen.java} wird der Quelltext in Bytecode übersetzt
  \item Der Aufruf \code{java KugelVolumen} startet das Programm in der Java Virtual Machine
  \item Auf dem Bildschirm erscheint folgende Ausgabe:
  \begin{lstlisting}[style=bash]
  Bitte Kugelradius eingeben:
  > 1
  Das Volumen betreagt v = 4.1887902047863905
  \end{lstlisting}
\end{itemize}
\end{frame}

\def\kap{3}%
\AtBeginSection{}
%%%%%%%%%%%%%%%%%%%%%%%%%%%%%%%%%%%%%%%%%%%%%%%%%%%%%%%%%%%%%%%%%%%%%%%%
\section{Installation von Java und Text Editoren}
\begin{frame}
  \frametitle{\kap. Installation von Java und Text Editoren}%
\tableofcontents[current]
\end{frame}
% \setcounter{section}{0}


%%%%%%%%%%%%%%%%%%%%%%%%%%%%%%%%%%%%%%%%%%%%%%%%%%%%%%%%%%%%%%%%%%%%%%%%
\def\stitle{Installation von Java SE}%
\subsection{\stitle}\label{S:Compiler}
\begin{frame}[t]%
  \frametitle{\kap.\ref{S:Compiler} \stitle}%

\heading{Abhängig vom Betriebssystem variiert die Installation}
\begin{description}
  \item [Linux] \"Uber den Paketmanager, hier Ubuntu 18.04, mittels \\
  \code{\$ sudo apt install openjdk-11-jdk}
  \item[Windows] Installiere Java SE (beinhaltet JDK). Download unter \textcolor{KITblue}{\url{https://www.oracle.com/technetwork/java/javase/downloads/index.html}}
  \item[Alternativ] Ab Windows 10 v.1607 "{}Anniversary Update"{} kann das \emph{Windows Subsystem for Linux} (WSL) installiert werden.
    Damit steht unter Windows das Linux Terminal zur Verfügung.
    F\"ur die Installation siehe \textcolor{KITblue}{\url{https://docs.microsoft.com/en-us/windows/wsl/install-win10}}
    \textbf{Idee:} Editiere die Quell-Dateien in Windows, und übersetzte und führe das Programm in der Linux Konsole aus.
\end{description}

\vfill
In der Übung werden die Programme mit WSL/WSL2 entwickelt.
\end{frame}


%%%%%%%%%%%%%%%%%%%%%%%%%%%%%%%%%%%%%%%%%%%%%%%%%%%%%%%%%%%%%%%%%%%%%%%%
\def\stitle{Editoren}%
\subsection{\stitle}\label{S:Editor}
\begin{frame}[t]%
  \frametitle{\kap.\ref{S:Editor} \stitle}%

Den Quelltext eines Java-Programms k\"onnen Sie mit jedem Texteditor erstellen.
Achten Sie aber darauf, dass nur der reine Text und keine Formatierungen gespeichert wird (\textbf{keine} Textverarbeitungssoftware wie MS Word, LibreOffice).
\vfill

\heading{Geeigneter Editor unter Windos}
\begin{description}
  \item[Notepad++] Open-source Texteditor f\"ur Windows mit Syntax-Highlighting.
\end{description}

\heading{Geeignete Editoren unter Linux}
\begin{description}
  \item[gedit] Dieser Editor ist auf GNOME Desktop Umgebungen vorinstalliert, somit auf den Linux Distributionen Fedora und Ubuntu.
  \item[vim] Vim ist ein Konsolen-basierter Editor.
\end{description}

\vfill
Für die Programm Entwicklung mit WSL eignet sich Notepad++.
\end{frame}


%%%%%%%%%%%%%%%%%%%%%%%%%%%%%%%%%%%%%%%%%%%%%%%%%%%%%%%%%%%%%%%%%%%%%%%%
\def\stitle{Entwicklungsumgebungen}%
\subsection{\stitle}\label{S:IDE}
\begin{frame}[t]%
  \frametitle{\kap.\ref{S:IDE} \stitle}%

\heading{\textcolor{KIT-Rot}{Erfahrene} Programmierer arbeiten oft mit Entwicklungsumgebung (IDE: Integrated Development Environment).}

\begin{description}
  \item[Eclipse] Eine weitverbreitete open-source Entwicklungsumgebung die große Unterstützung aus der Industrie erhält, unter anderen von IBM, Bosch, CA Technologies, SAP, Oracle, siehe \textcolor{KITblue}{\url{www.eclipse.org}}.
  \item[NetBeans] Eine plattformunabh\"angige Entwicklungsumgebung, die im Rahmen eines von der Firma Sun Microsystems gef\"orderten Projekts entwickelt wird, siehe \textcolor{KITblue}{\url{www.netbeans.org}}.
  \item[VS Code] siehe \textcolor{KITblue}{\url{https://code.visualstudio.com/docs/languages/java}}
  \item[YCM] vim-plugin mit Autovervollständigung und vielem mehr, siehe \textcolor{KITblue}{\url{https://github.com/ycm-core/YouCompleteMe}}
\end{description}
\end{frame}

\def\kap{4}%
\def\stitle{Hello World}
%%%%%%%%%%%%%%%%%%%%%%%%%%%%%%%%%%%%%%%%%%%%%%%%%%%%%%%%%%%%%%%%%%%%%%%%
\section{\stitle}
\begin{frame}
  \frametitle{\kap. \stitle}%
\tableofcontents[current]
\end{frame}


%%%%%%%%%%%%%%%%%%%%%%%%%%%%%%%%%%%%%%%%%%%%%%%%%%%%%%%%%%%%%%%%%%%%%%%%
\begin{frame}[fragile]%
  \frametitle{\kap. \stitle\ - Quelltext}%

\lstinputlisting[style=JAVA,title=Diese kleine minimal Beispiel hei\ss t HelloWorld und gibt "Hello World!"{} auf der Konsole aus.]
{helloWorld/HelloWorld.java}
\end{frame}


%%%%%%%%%%%%%%%%%%%%%%%%%%%%%%%%%%%%%%%%%%%%%%%%%%%%%%%%%%%%%%%%%%%%%%%%
\begin{frame}[fragile]%
  \frametitle{\kap. \stitle\ - \"Ubersetzen und Ausf\"uhren}%

\begin{lstlisting}[title={Um das Programm HelloWorld auszuf\"uhren werden folgende Schritte auf dem Terminal durchgef\"uhrt.},style=BASH]
$ javac HelloWorld.java
$ java HelloWorld
Hello World!
\end{lstlisting}
\end{frame}

%% TODO kompilieren, mit Fehlern

\def\kap{5}%
\AtBeginSection{}
%%%%%%%%%%%%%%%%%%%%%%%%%%%%%%%%%%%%%%%%%%%%%%%%%%%%%%%%%%%%%%%%%%%%%%%%
\section{Beispiel Kugelvolumen}
\begin{frame}
  \frametitle{\kap. Beispiel Kugelvolumen}%
\tableofcontents[current]
\end{frame}


%%%%%%%%%%%%%%%%%%%%%%%%%%%%%%%%%%%%%%%%%%%%%%%%%%%%%%%%%%%%%%%%%%%%%%%%
\def\stitle{Definition Kugelvolumen}%
\subsection{\stitle}\label{S:BeispielKugelvolumen}
\begin{frame}[t]%
  \frametitle{\kap.\ref{S:BeispielKugelvolumen} \stitle}%
\medskip

Das Kugelvolumen $V$ ist der Rauminhalt einer Kugel und abh"angig vom Kugelradius $r>0$ und ist beschrieben durch
$$ V(r) := \frac{4}{3} \pi r^3. $$
\begin{itemize}
  \item Schreiben Sie ein Java-Programm, welches das Kugelvolumen einer beliebigen Kugel berechnet.
\end{itemize}
\medskip

Vorgehen:
\begin{itemize}
\item Lese Variable $r$ ein
\item W\"ahle geeigneten Datentyp f"ur Volumen $V$
\item Lade Wert von $\pi$ aus Bibliothek
\item Berechne Volumen
\item Gebe berechneten Wert auf Konsole aus
\end{itemize}
\end{frame}


%%%%%%%%%%%%%%%%%%%%%%%%%%%%%%%%%%%%%%%%%%%%%%%%%%%%%%%%%%%%%%%%%%%%%%%%
\def\stitle{Beispiel Programm}%
\subsection{\stitle}\label{S:BeispielProgramm}
\begin{frame}[t]%
  \frametitle{\kap.\ref{S:BeispielProgramm} \stitle}%
\heading{Lese Kugelradius $r$ ein}

\lstinputlisting[style=JAVAlines,frame=single,linerange={1-10, 13-14}]
{\getexercisefolder/KugelVolumen.java}
\end{frame}


%%%%%%%%%%%%%%%%%%%%%%%%%%%%%%%%%%%%%%%%%%%%%%%%%%%%%%%%%%%%%%%%%%%%%%%%
\def\stitle{Schritt f\"ur Schritt}%
\subsection{\stitle}\label{S:SchrittSchritt}
\begin{frame}[t]%
  \frametitle{\kap.\ref{S:SchrittSchritt} \stitle}%

\heading{Lade Wert $\pi$ aus Bilbliothek und berechne das Volumen}

\lstinputlisting[style=JAVAlines,frame=single,linerange={1-11, 13-14}]
{\getexercisefolder/KugelVolumen.java}

\end{frame}


%%%%%%%%%%%%%%%%%%%%%%%%%%%%%%%%%%%%%%%%%%%%%%%%%%%%%%%%%%%%%%%%%%%%%%%%
\def\stitle{Schritt f\"ur Schritt}%
\begin{frame}[t]%
  \frametitle{\kap.\ref{S:SchrittSchritt} \stitle}%

\heading{Gebe das Volumen auf Konsole aus}
\lstinputlisting[style=JAVA,linerange={1-14}]
{\getexercisefolder/KugelVolumen.java}

\end{frame}


%%%%%%%%%%%%%%%%%%%%%%%%%%%%%%%%%%%%%%%%%%%%%%%%%%%%%%%%%%%%%%%%%%%%%%%%
\def\stitle{Schritt f\"ur Schritt}%
\begin{frame}[fragile]%
  \frametitle{\kap.\ref{S:SchrittSchritt} \stitle}%
\medskip

\begin{lstlisting}[title={Um das Programm \code{KugelVolumen} auszuf\"uhren werden folgende Schritte auf dem Terminal durchgef\"uhrt.},style=BASH]
$ javac KugelVolumen.java
$ java KugelVolumen
Bitte Kugelradius eingeben: 1
Das Volumen betraegt v = 4.1887902047863905
\end{lstlisting}

\end{frame}

\def\kap{6}%
\AtBeginSection{}
%%%%%%%%%%%%%%%%%%%%%%%%%%%%%%%%%%%%%%%%%%%%%%%%%%%%%%%%%%%%%%%%%%%%%%%%
\section{Dokumentation}
\begin{frame}
  \frametitle{\kap. Dokumentation}%
\tableofcontents[current]
\end{frame}
% \setcounter{section}{0}


%%%%%%%%%%%%%%%%%%%%%%%%%%%%%%%%%%%%%%%%%%%%%%%%%%%%%%%%%%%%%%%%%%%%%%%%
\def\stitle{Java SE Dokumentation}%
\subsection*{\stitle}\label{S:Java SE 8 Dokumentation}
\begin{frame}[t]%
  \frametitle{\kap.\ref{S:Java SE 8 Dokumentation} \stitle}%
\medskip

\begin{itemize}
\item Vollständige Dokumentation \textcolor{KITblue}{\url{http://docs.oracle.com/javase/8/docs/api/}}
\item Umfassende Internet Tutorien \textcolor{KITblue}{\url{https://www.tutorialspoint.com/java/}} oder \textcolor{KITblue}{\url{https://www.javatpoint.com/java-tutorial}}
\end{itemize}
\medskip

\includegraphics[width=0.8\textwidth]{\getexercisefolder/java_small.png}
\end{frame}


%%%%%%%%%%%%%%%%%%%%%%%%%%%%%%%%%%%%%%%%%%%%%%%%%%%%%%%%%%%%%%%%%%%%%%%%
\def\stitle{Online Compiler}%
\subsection*{\stitle}\label{S:Online Compiler}
\begin{frame}[t]%
  \frametitle{\kap.\ref{S:Online Compiler} \stitle}%
\medskip

Auf der Seite \textcolor{KITblue}{\url{https://www.onlinegdb.com/online_java_compiler}} können Sie sehr bequem kleinere Programme entwickeln, ohne auf Ihrem Rechner Java SE installiert zu haben.
Einzige Voraussetzung ist dabei eine bestehende Internet Verbindung.
\end{frame}



\begin{frame}
  \frametitle{Zusammenfassung}%
\tableofcontents[hideallsubsections]
\end{frame}

\begin{frame}
\centering
\Huge\GREEN{Fragen?}
\vspace{2cm}

{\LARGE
N\"achste \"Ubung: 23. Oktober\\
Besprechung Arbeitsblatt 1
}
\end{frame}


%%%%%%%%%%%%%%%%%%%%%%%%%%%%%%%%%%%%%%%%%%%%%%%%%%%%%%%%%%%%%%%%%%%%%%%%
\end{document}
