\documentclass[c,18pt]{beamer}
\listfiles

\mode<presentation>
{
  \usetheme[deutsch,titlepage0]{KIT}
  \setbeamercovered{transparent}
  \setbeamertemplate{enumerate items}[ball]
}
\usepackage[utf8]{inputenc}
\date{23.10.2018}

\newlength{\Ku}
\setlength{\Ku}{1.43375pt}

\usepackage{templateSlide/exercises}
\usepackage[java]{code}

\exercisespath{uebungsfolien/}

%% title slide
\title[Übung 01: Einstieg in die Informatik und algorithmische Mathematik]
  {Übung 01: Einstieg in die Informatik und algorithmische Mathematik}
\subtitle{Albert Mink | Wintersemester 2018/2019}

\author[Albert Mink, ]{KIT}

\AuthorTitleSep{\relax}

\institute[Institut für Angewandte und Numerische Mathematik (IANM)]{Institut für Angewandte und Numerische Mathematik}

\TitleImage[width=\titleimagewd]{logos/KIT-Titel}
\logo{\includegraphics{logos/lbrg_logo}}

%%%%%%%%%%%%%%%%%%%%%%%%%%%%%%%%%%%%%%%%%%%%%%%%%%%%%%%%%%%%%%%%%%%%%%%%
% Document
%%%%%%%%%%%%%%%%%%%%%%%%%%%%%%%%%%%%%%%%%%%%%%%%%%%%%%%%%%%%%%%%%%%%%%%%
\begin{document}
\begin{frame}
  \maketitle
\end{frame}
%%%%%%%%%%%%%%%%%%%%%%%%%%%%%%%%%%%%%%%%%%%%%%%%%%%%%%%%%%%%%%%%%%%%%%%%

\begin{frame}
  \frametitle{Übung zum Arbeitsblatt 01}%
\tableofcontents[hideallsubsections]
\end{frame}
%TODO listings add linerange

%%%%%%%%%%%%%%%%%%%%%%%%%%%%%%%%%%%%%%%%%%%%%%%%%%%%%%%%%%%%%%%%%%%%%%%%
%%%%%%%%%%%%%%%%%%%%%%%%%%%%%%%%%%%%%%%%%%%%%%%%%%%%%%%%%%%%%%%%%%%%%%%%
\section{Wiederholung}\label{K:wdh}
\begin{frame}
  \frametitle{\ref{K:wdh} Wiederholung}%
\tableofcontents[current]
\end{frame}


%%%%%%%%%%%%%%%%%%%%%%%%%%%%%%%%%%%%%%%%%%%%%%%%%%%%%%%%%%%%%%%%%%%%%%%%
\def\stitle{Grundlagen in Java}
\subsection{\stitle}\label{S:GrundlageninJava}
\begin{frame}[fragile]%
  \frametitle{\ref{K:wdh}.\ref{S:GrundlageninJava} \stitle}%

\begin{description}[leftmargin=*,style=nextline]
\item[\textcolor{KITgreen}{\textbf{Datentypen}}]
\item[Ganzzahlige Typen]  \code{byte, short, int, long}
\item[Gleitkomma Typen]   \code{float, double}
\item[Zeichen]            \code{char, String}
\item[Boolscher Typen]    \code{boolean}
\end{description}
\medskip

\begin{description}[leftmargin=*,style=nextline]
\item[\textcolor{KITgreen}{\textbf{Variablen}}]
\item[Deklaration] \code{double x;}
\item[Definition] \code{int n = -1;}
\item[Wertzuweisung] \code{int a = n;}
\end{description}

\end{frame}


%%%%%%%%%%%%%%%%%%%%%%%%%%%%%%%%%%%%%%%%%%%%%%%%%%%%%%%%%%%%%%%%%%%%%%%%
\begin{frame}[fragile]%
  \frametitle{\ref{K:wdh}.\ref{S:GrundlageninJava} \stitle}%

\textcolor{KITgreen}{\heading{Variablen}} \code{int, for, const, double, else, final, import, class, this, short, while, ...}

\end{frame}


%%%%%%%%%%%%%%%%%%%%%%%%%%%%%%%%%%%%%%%%%%%%%%%%%%%%%%%%%%%%%%%%%%%%%%%%
\def\stitle{Kompilieren und Ausführen}
\subsection{\stitle}\label{S:CompilierenUexec}
\begin{frame}[fragile]%
  \frametitle{\ref{K:wdh}.\ref{S:CompilierenUexec} \stitle}%

Um das Programm Hello World auszuführen werden folgende Schritte auf dem Terminal durchgeführt.

\begin{lstlisting}[style=BASH]
$ javac HelloWorld.java
$ java HelloWorld
  Hello World!
\end{lstlisting}

\end{frame}


%%%%%%%%%%%%%%%%%%%%%%%%%%%%%%%%%%%%%%%%%%%%%%%%%%%%%%%%%%%%%%%%%%%%%%%%
\setcounter{exercise}{1}
\inputexercise{grundl-java}

%%%%%%%%%%%%%%%%%%%%%%%%%%%%%%%%%%%%%%%%%%%%%%%%%%%%%%%%%%%%%%%%%%%%%%%%
\setcounter{exercise}{2}
\inputexercise{grundl-literale}

%%%%%%%%%%%%%%%%%%%%%%%%%%%%%%%%%%%%%%%%%%%%%%%%%%%%%%%%%%%%%%%%%%%%%%%%
\setcounter{exercise}{3}
\inputexercise{grundl-ausdruecke} %% Java Bsp Code

%%%%%%%%%%%%%%%%%%%%%%%%%%%%%%%%%%%%%%%%%%%%%%%%%%%%%%%%%%%%%%%%%%%%%%%%
\def\kap{2}%
\inputexercise{guterStartInsPraktikum}

%%%%%%%%%%%%%%%%%%%%%%%%%%%%%%%%%%%%%%%%%%%%%%%%%%%%%%%%%%%%%%%%%%%%%%%%
\section{Zusammenfassung}
\begin{frame}
  \frametitle{Zusammenfassung}%
\tableofcontents[hideallsubsections]
\end{frame}

%%%%%%%%%%%%%%%%%%%%%%%%%%%%%%%%%%%%%%%%%%%%%%%%%%%%%%%%%%%%%%%%%%%%%%%%
\begin{frame}
\centering
\Huge\textcolor{KITgreen}{Fragen?}
\vspace{2cm}

{\LARGE
Nächste Übung: 30. Oktober\\
Besprechung Arbeitsblatt 2
}
\end{frame}


%%%%%%%%%%%%%%%%%%%%%%%%%%%%%%%%%%%%%%%%%%%%%%%%%%%%%%%%%%%%%%%%%%%%%%%%
\end{document}
