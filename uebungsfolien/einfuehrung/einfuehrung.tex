%\AtBeginSection{}
%%%%%%%%%%%%%%%%%%%%%%%%%%%%%%%%%%%%%%%%%%%%%%%%%%%%%%%%%%%%%%%%%%%%%%%%
\section{Einführung}
\begin{frame}
  \frametitle{\kap. Einführung}%
\tableofcontents[current]
\end{frame}


%%%%%%%%%%%%%%%%%%%%%%%%%%%%%%%%%%%%%%%%%%%%%%%%%%%%%%%%%%%%%%%%%%%%%%%%
\def\stitle{Die \"Ubung}%
\subsection{\stitle}\label{S:Uebersicht}
\begin{frame}[fragile]%
  \frametitle{\kap.\ref{S:Uebersicht} \stitle}%
\medskip

\begin{itemize}
%  \item Klausur: 01.02.18 und 12.07.18
%  \item Vorlesung: Montags 11:30 -- 13:00
  \item Dienstags 11:30 -- 13:00
  \item Inhalte der \"Ubung
  \begin{itemize}
    \item Besprechung der Arbeitsbl\"atter (etwa 2 bis 3 Aufgaben)
    \item Rechnerdemos
    \item Beantwortung von Fragen zur Vorlesung und Praktikum
  \end{itemize}
  \item Arbeitsbl\"atter werden immer eine Woche vor der \"Ubung hochgeladen
\end{itemize}
\medskip

\begin{description}[leftmargin=*,style=nextline]
  \item[\textcolor{black}{\textbf{Ansprechpartner}}]
  \item[\"Ubung] Albert.Mink@kit.edu, Raum 210 Geb. 30.70,\\ Sprechstunde: nach Vereinbarung
  \item[\"Ubung und Praktikum] Zoltan.Veszelka@kit.edu, Raum 3.014 Geb. 20.30,\\ Sprechstunde: Di, 14:00 -- 15:00
  \item[Praktikum] Ihre Tutoren im Praktikum
\end{description}
\end{frame}


%%%%%%%%%%%%%%%%%%%%%%%%%%%%%%%%%%%%%%%%%%%%%%%%%%%%%%%%%%%%%%%%%%%%%%%%
\def\stitle{Bearbeitung der Praktikumsaufgaben}%
\subsection{\stitle}\label{S:Praktikum}
\begin{frame}[fragile]%
  \frametitle{\kap.\ref{S:Praktikum} \stitle}%
\medskip

\begin{itemize}
  \item Praktikumsaufgaben k\"onnen zu Hause oder im Praktikum bearbeitet werden
  \item Vorbereitung \textbf{vor} dem Praktikum ist obligatorisch
  \item Tutoren \textbf{unterst\"utzen} Sie bei Fragen und auftretenden Schwierigkeiten
  \item Sie erhalten Testate f\"ur Pflichtaufgaben die alle nachfolgenden Bedingungen erf\"ullen:
  \begin{itemize}
    \item Der Bearbeitungszeitraum ist nicht \"uberschritten (Ausnahmen sind \textbf{nur} in begr\"undeten F\"allen m\"oglich, z.B. Krankheit mit Attest)
    \item Ihr Programm enth\"alt keine \emph{syntaktischen} Fehler (fehlerfrei kompilier- und ausf\"uhrbar)
    \item Ihr Programm enth\"alt keine \emph{semantischen} Fehler (es liefert die korrekten Ergebnisse)
    \item Sie k\"onnen Ihr Programm \emph{live} kompilieren, ausf\"uhren \textbf{und} erkl\"aren
  \end{itemize}
\end{itemize}
\medskip

\begin{center}
 \LARGE{Voraussetzung f\"ur die Zulassung zur Klausur:\\ Erwerb \textbf{aller} Testate}
\end{center}
\medskip

Siehe auch: Merkblatt zum Praktikum (auf ILIAS) \url{https://ilias.studium.kit.edu}
\end{frame}


%%%%%%%%%%%%%%%%%%%%%%%%%%%%%%%%%%%%%%%%%%%%%%%%%%%%%%%%%%%%%%%%%%%%%%%%
\def\stitle{Einordnung und Zusammenfassung}%
\subsection{\stitle}\label{S:Einordnung}
\begin{frame}[fragile]%
  \frametitle{\kap.\ref{S:Einordnung} \stitle}%
  \medskip

  \small
  \centering
  \begin{tabular}{p{2cm}|p{2.0cm}|p{2.7cm}|p{3.0cm}}
  & {\centering Vorlesung} & {\centering \"Ubung} & {\centering Praktikum}\\
  \hline
  \hline & & &  \\
  Inhalte & Vermittlung des Vorlesungsstoffes & Vertiefung und Wiederholung des Vorlesungsstoffes, Besprechung der Arbeitsbl\"atter, Kl\"arung offener Fragen & Selbst\"andiges Programmieren und Abgabe von Testaten (Unterst\"utzt durch Ihre Tutoren)\\
  \hline & & & \\
  Materialien & Folien & Arbeitsbl\"atter, Folien & Aufgabenbl\"atter\\
  \hline & & & \\
  Be\-ar\-bei\-tungs\-zeit\-raum & & 1 Woche\newline (ab Dienstag) & 2 Wochen\newline  (ab Montag)\\
  \hline & & & \\
  Klau\-sur\-vo\-r\-aus\-set\-zung & & & Fristgerechte und erfolgreiche Abgabe \textbf{aller} Pflichtaufgaben\\
  %  \hline & & & \\
  %  aktive Mitarbeit & wenig & mittel & viel \\
  \end{tabular}
  \medskip

  \textbf{Achtung:} Die ersten Tutorien finden bereits ab Montag, den 22. Oktober statt!
\end{frame}
