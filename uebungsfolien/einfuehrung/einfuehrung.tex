%\AtBeginSection{}
%%%%%%%%%%%%%%%%%%%%%%%%%%%%%%%%%%%%%%%%%%%%%%%%%%%%%%%%%%%%%%%%%%%%%%%%
\section{Einführung}
\begin{frame}
  \frametitle{\kap. Einführung}%
\tableofcontents[current]
\end{frame}


%%%%%%%%%%%%%%%%%%%%%%%%%%%%%%%%%%%%%%%%%%%%%%%%%%%%%%%%%%%%%%%%%%%%%%%%
\def\stitle{Die Übung}%
\subsection{\stitle}\label{S:Uebersicht}
\begin{frame}[fragile]%
  \frametitle{\kap.\ref{S:Uebersicht} \stitle}%

\heading{Inhalte der Übung}
\begin{itemize}
  \item Live programmieren von Rechnerdemos
  \item Besprechung der Arbeitsblätter
  \item Beantwortung von Fragen zur Vorlesung
  %\item Arbeitsblätter werden eine Woche vor der Übung hochgeladen
\end{itemize}
\hfill

\begin{description}[leftmargin=*,style=nextline]
  \item[\textcolor{black}{\textbf{Ansprechpartner}}]
  \item[Übung] Albert.Mink@kit.edu, Raum 210 Geb. 30.70,\\ Sprechstunde: nach Vereinbarung
  \item[Übung und Praktikum] Zoltan.Veszelka@kit.edu, Raum 3.014 Geb. 20.30,\\ Sprechstunde: Di, 14:00 -- 15:00
  \item[Praktikum] Ihre Tutoren im Praktikum
\end{description}
\end{frame}


%%%%%%%%%%%%%%%%%%%%%%%%%%%%%%%%%%%%%%%%%%%%%%%%%%%%%%%%%%%%%%%%%%%%%%%%
\def\stitle{Bearbeitung der Praktikumsaufgaben}%
\subsection{\stitle}\label{S:Praktikum}
\begin{frame}[fragile]%
  \frametitle{\kap.\ref{S:Praktikum} \stitle}%

\begin{itemize}
  \item Praktikumsaufgaben können zu Hause oder im Praktikum bearbeitet werden
  \item Vorbereitung \textbf{vor} dem Praktikum ist obligatorisch
  \item Tutoren \textbf{unterstützen} Sie bei Fragen und auftretenden Schwierigkeiten
  \item Sie erhalten Testate für Pflichtaufgaben die alle nachfolgenden Bedingungen erfüllen:
  \begin{itemize}
    \item Bearbeitungszeitraum ist nicht überschritten (Ausnahmen sind \textbf{nur} in begründeten Fällen möglich, z.B. Krankheit mit Attest)
    \item Programm enthält keine \emph{syntaktischen} Fehler (fehlerfrei kompilier- und ausführbar)
    \item Programm enthält keine \emph{semantischen} Fehler (es liefert die korrekten Ergebnisse)
    \item Sie können Ihr Programm \emph{live} kompilieren, ausführen \textbf{und} erklären
  \end{itemize}
\end{itemize}
\hfill

\begin{center}
 \textcolor{KITred}{Voraussetzung für die Zulassung zur Klausur:\\ Erwerb \textbf{aller} Testate}
\end{center}
\hfill

Siehe auch: Merkblatt zum Praktikum (auf ILIAS) \url{https://ilias.studium.kit.edu}
\end{frame}


%%%%%%%%%%%%%%%%%%%%%%%%%%%%%%%%%%%%%%%%%%%%%%%%%%%%%%%%%%%%%%%%%%%%%%%%
\def\stitle{Einordnung und Zusammenfassung}%
\subsection{\stitle}\label{S:Einordnung}
\begin{frame}[fragile]%
  \frametitle{\kap.\ref{S:Einordnung} \stitle}%

  \begin{tabular}{p{2cm}|p{2.0cm}|p{2.7cm}|p{3.0cm}}
  & {\centering Vorlesung} & {\centering Übung} & {\centering Praktikum}\\
  \hline
  \hline & & &  \\
  Inhalte & Vermittlung des Vorlesungs\-stoffes & Vertiefung und Wiederholung des Vorlesungsstoffes, Besprechung der Arbeitsblätter & Selbständiges Programmieren und Abgabe von Testaten (Unterstützt durch Ihre Tutoren)\\
  \hline & & & \\
  Materialien & Folien & Folien & Aufgabenblätter\\
  \hline & & & \\
  Be\-ar\-bei\-tungs\-zeit\-raum & & 1 Woche\newline (ab Dienstag) & 2 Wochen\newline  (ab Montag)\\
  %\hline & & & \\
  %Klau\-sur\-vo\-r\-aus\-set\-zung & & & Fristgerechte und erfolgreiche Abgabe \textbf{aller} Pflichtaufgaben\\
  \hline & & & \\
  Mitarbeit & wenig & mittel & viel
  \end{tabular}
  \vfill

  \textbf{Achtung:} Die ersten Tutorien finden bereits ab Montag, den 21. Oktober statt!
\end{frame}


%%%%%%%%%%%%%%%%%%%%%%%%%%%%%%%%%%%%%%%%%%%%%%%%%%%%%%%%%%%%%%%%%%%%%%%%
\def\stitle{Wichtige Termine}%
\subsection{\stitle}\label{S:Termine}
\begin{frame}[fragile]%
  \frametitle{\kap.\ref{S:Termine} \stitle}%

\begin{itemize}
  \item Vorlesung: Montags 11:30 -- 13:00
  \item Übung: Dienstags 11:30 -- 13:00
  \item Terminvergabe Praktikum: 15.Okt um 15:30 -- 17.Okt um 12:00
  \item (Voraussichtlich) Klausur: 30.01.20
\end{itemize}
\end{frame}
