\def\stitle{Wiederholung Mehrdimensionale Felder}
\section{\stitle}\label{K:wdh}
\begin{frame}
  \frametitle{\kap. \stitle}%
\tableofcontents[current]
\end{frame}
%%%%%%%%%%%%%%%%%%%%%%%%%%%%%%%%%%%%%%%%%%%%%%%%%%%%%%%%%%%%%%%%%%%%%%%%
%%%%%%%%%%%%%%%%%%%%%%%%%%%%%%%%%%%%%%%%%%%%%%%%%%%%%%%%%%%%%%%%%%%%%%%%
%%%%%%%%%%%%%%%%%%%%%%%%%%%%%%%%%%%%%%%%%%%%%%%%%%%%%%%%%%%%%%%%%%%%%%%%
\def\stitle{Erzeugen von mehrdimensionalen Feldern}
\subsection{\stitle}\label{S:Erzeugen}
\begin{frame}[t]%
  \frametitle{\ref{K:wdh}.\ref{S:Erzeugen} \stitle}
\medskip

Ein Feld kann mehrere Variablen vom selben Datentyp enthalten, hier \code{int} und \code{char}.
\lstinputlisting[style=JAVA]{\getexercisefolder/FelderErzeugen.java}

\end{frame}
%%%%%%%%%%%%%%%%%%%%%%%%%%%%%%%%%%%%%%%%%%%%%%%%%%%%%%%%%%%%%%%%%%%%%%%%


%%%%%%%%%%%%%%%%%%%%%%%%%%%%%%%%%%%%%%%%%%%%%%%%%%%%%%%%%%%%%%%%%%%%%%%%
%%%%%%%%%%%%%%%%%%%%%%%%%%%%%%%%%%%%%%%%%%%%%%%%%%%%%%%%%%%%%%%%%%%%%%%%
\def\stitle{Zugreifen auf mehrdimensionale Felder}
\subsection{\stitle}\label{S:Zugreifen}
\begin{frame}[t]%
  \frametitle{\ref{K:wdh}.\ref{S:Zugreifen} \stitle}

Mit dem Operator \code{[]} wird auf bestimmte Feldkomponenten zugegriffen.
\lstinputlisting[style=JAVA]{\getexercisefolder/FelderZugreifen.java}

\end{frame}
