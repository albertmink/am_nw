\def\stitle{\theexercise\ - Ausdruck}
\section{\stitle}
\begin{frame}
  \frametitle{\stitle}%
\tableofcontents[current]
\end{frame}

\begin{frame}[t]%
    \frametitle{\stitle}
Geben Sie das Ergebnis und den Datentyp der nachfolgenden Ausdr\"ucke an:
\begin{enumerate}
\item \code{(int) 2.0/8.0}
\item[] {\bf Ergebnis: } \dotfill {\bf Datentyp: } \dotfill

\item \code{1/3}
\item[] {\bf Ergebnis: } \dotfill {\bf Datentyp: } \dotfill

\item \code|3.1*2!=6|
\item[] {\bf Ergebnis: } \dotfill {\bf Datentyp: } \dotfill

\item \code{( 2==2 \&& 1 > 5/2.0 || false )}
\item[] {\bf Ergebnis: } \dotfill {\bf Datentyp: } \dotfill

\item \code{120\%50/20}
\item[] {\bf Ergebnis: } \dotfill {\bf Datentyp: } \dotfill

\item \code{5/2.0}
\item[] {\bf Ergebnis: } \dotfill {\bf Datentyp: } \dotfill

\item \code{( true || (false && 3 > 2) )}
\item[] {\bf Ergebnis: } \dotfill {\bf Datentyp: } \dotfill

\item \code{(char)('a'+(int)1)}
\item[] {\bf Ergebnis: } \dotfill {\bf Datentyp: } \dotfill
\end{enumerate}
\end{frame}


\begin{frame}[t]%
    \frametitle{\stitle - Lsg}
Geben Sie das Ergebnis und den Datentyp der nachfolgenden Ausdr\"ucke an:
\begin{enumerate}
\item \code{(int) 2.0/8.0}
\item[] {\bf Ergebnis: } 0.25 \hfill {\bf Datentyp: } \code{double}

\item \code{1/3}
\item[] {\bf Ergebnis: } 0 \hfill {\bf Datentyp: } \code{int}

\item \code|3.1*2!=6|
\item[] {\bf Ergebnis: } true \hfill {\bf Datentyp: } \code{boolean}

\item \code{( 2==2 \&& 1 > 5/2.0 || false )}
\item[] {\bf Ergebnis: } false \hfill {\bf Datentyp: } \code{boolean}

\item \code{120\%50/20}
\item[] {\bf Ergebnis: } 1 \hfill {\bf Datentyp: } \code{int}

\item \code{5/2.0}
\item[] {\bf Ergebnis: } 2.5 \hfill {\bf Datentyp: } \code{double}

\item \code{( true || (false && 3 > 2) )}
\item[] {\bf Ergebnis: } true \hfill {\bf Datentyp: } \code{boolean}

\item \code{(char)('a'+(int)1)}
\item[] {\bf Ergebnis: } b \hfill {\bf Datentyp: } \code{char}
\end{enumerate}
\end{frame}
