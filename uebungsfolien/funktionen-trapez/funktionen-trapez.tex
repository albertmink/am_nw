\def\stitle{\theexercise\ - Trapezregel}
\section{\stitle}
\begin{frame}
  \frametitle{\stitle}%
\tableofcontents[current]
\end{frame}


\begin{frame}[fragile]%
    \frametitle{\stitle}

Berechnen Sie numerisch das Integral der Funktion $f$ definiert als $f(x) = -x^2+4$ in den Grenzen $0$ und $3$.
\begin{enumerate}
\item Verwenden Sie die Trapezregel mit zwei St\"utzstellen $a=0$ und $b=3$
\[
I_1
= \int_a^b f(x)dx
\simeq (b-a)\frac{f(a)+f(b)}{2}
.\]
\item Verwenden Sie die zusammengesetzte Trapezregel für $N$ St\"utzstellen
\[
I_2
= \int_a^b f(x)dx
\simeq h \left[\frac{1}{2}f(a)+\frac{1}{2}f(b)+\sum_{n=1}^{N-1}f\left(a+nh \right)\right]
,\]
mit Gewicht $h = \frac{b-a}{N}$ und Integrationsgrenzen $a=0$ und $b=3$.

\emph{Implementieren Sie eine Java-Funktion}
\lstinputlisting[style=JAVAsmall,linerange={4-6}]{\getexercisefolder/Trapez.java}
\end{enumerate}

\end{frame}


\begin{frame}[t]%
  \frametitle{\theexercise\ - Graph}%
\heading{Trapezregel} für N=3 und $f(x)=-x^2+4$

\pgfplotsset{
    integral axis/.style={
        axis lines=middle,
        enlarge y limits=upper,
        xlabel=$x$, ylabel=$y$,
        ytick=\empty,
        xticklabel style={font=\small, text height=1.5ex, anchor=north},
        samples=100
    }
}
\begin{tikzpicture}
[
    % The function that is used for all the plots
    declare function={f=-x*x+4;}
]
\begin{axis}
[
    integral axis,
    xmin=-1, xmax=4,
    domain=-0.5:3.5,
    xtick={0,1,2,3},
    xticklabels={$a=0$,$a+h$,$a+2h$,$b=3$},
]
% The function
\addplot [very thick, KITgreen] {f} node [anchor=south] {$f(x)$};

% The filled area under the approximate integral
\addplot [domain=0:3,samples=4,draw=none, on layer=axis background, fill=KITgreen30] {f} \closedcycle;

% The vertical lines between the segments
\addplot [domain=0:3,samples=4, ycomb] {f};
\end{axis}
\end{tikzpicture}


\end{frame}


\begin{frame}[t]%
  \frametitle{\theexercise\ - Beispiel Programm}%

\lstinputlisting[style=JAVAsmall]{\getexercisefolder/Trapez.java}
\end{frame}


\begin{frame}[t]%
  \frametitle{\theexercise\ - Konvergenz des Verfahrens}%
\heading{Frage:}
Um welchen Faktor wird die numerische Berechnung genauer, wenn Anzahl der Stützstellen verdoppelt wird?

\begin{equation*}
\int_a^b f(x)dx
\simeq
h \left[\frac{1}{2}f(a)+\frac{1}{2}f(b)+\sum_{n=1}^{N-1}f\left(a+nh \right)\right]
\end{equation*}

\begin{table}
\caption{Fehleranalyse}
\begin{tabular}{r|r}
N   & absoluter Fehler \\ \hline \hline
10  &  0.0450\\
20  &  0.01125\\
40  &  0.00281249\\
80  &  7.031249E-4\\
160 &  1.7578125E-4\\
320 &  4.39453125E-5\\
\end{tabular}
\end{table}
\hfill

Quadratische Konvergenz  (verdopple Auflösung - verringere Fehler um Faktor 4)
\end{frame}
