\def\stitle{\theexercise\ - Funktionen - Trapezregel}
\section{\stitle}
\begin{frame}[fragile]%
    \frametitle{\stitle}

Berechnen Sie numerisch das Integral der Funktion $f$ definiert als $f(x) = -x^2+4$ in den Grenzen $0$ und $3$.
\begin{itemize}
\item[(a)] Verwenden Sie die Trapezregel mit zwei St\"utzstellen
\[
I_1
= \int_a^b f(x)dx
\simeq (b-a)\frac{f(a)+f(b)}{2}
.\]
Hier soll $a=0$ und $b=3$ sein.
\item[(b)] Verwenden Sie die zusammengesetzte Trapezregel für $N$ St\"utzstellen
\[
I_2
= \int_a^b f(x)dx
\simeq h \left[\frac{1}{2}f(a)+\frac{1}{2}f(b)+\sum_{n=1}^{N-1}f\left(a+nh \right)\right]
,\]
mit Gewicht $h = \frac{b-a}{N}$ und Integrationsgrenzen $a=0$ und $b=3$.
\emph{Implementieren Sie eine Java-Funktion}

\begin{lstlisting}[style=Java]
public static function(double x) {
  return -x*x + 4;
}
\end{lstlisting}
\end{itemize}


\end{frame}

\begin{frame}[t]%
  \frametitle{Beispiel Programm}%

\lstinputlisting[style=JAVAsmalllines]{\getexercisefolder/Trapez.java}
\end{frame}
