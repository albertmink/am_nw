\def\stitle{\theexercise\ - Trapezregel}
\section{\stitle}
\begin{frame}
  \frametitle{\stitle}%
\tableofcontents[current]
\end{frame}


\begin{frame}[fragile]%
    \frametitle{\stitle}

Berechnen Sie numerisch das Integral der Funktion $f$ definiert als $f(x) = -x^2+4$ in den Grenzen $0$ und $3$.
\begin{enumerate}
\item Verwenden Sie die Trapezregel mit zwei St\"utzstellen $a=0$ und $b=3$
\[
I_1
= \int_a^b f(x)dx
\simeq (b-a)\frac{f(a)+f(b)}{2}
.\]
\item Verwenden Sie die zusammengesetzte Trapezregel für $N$ St\"utzstellen
\[
I_2
= \int_a^b f(x)dx
\simeq h \left[\frac{1}{2}f(a)+\frac{1}{2}f(b)+\sum_{n=1}^{N-1}f\left(a+nh \right)\right]
,\]
mit Gewicht $h = \frac{b-a}{N}$ und Integrationsgrenzen $a=0$ und $b=3$.

\emph{Implementieren Sie eine Java-Funktion}
\lstinputlisting[style=JAVAsmall,linerange={4-6}]{\getexercisefolder/Trapez.java}
\end{enumerate}

\end{frame}


\begin{frame}[t]%
  \frametitle{\theexercise\ - Beispiel Programm}%

\lstinputlisting[style=JAVAsmall]{\getexercisefolder/Trapez.java}
\end{frame}
