%%%%%%%%%%%%%%%%%%%%%%%%%%%%%%%%%%%%%%%%%%%%%%%%%%%%%%%%%%%%%%%%%%%%%%%%
\def\stitle{\theexercise\ - Effizienz}
\def\sAtitle{\theexercise\ - Effizienz Teil A}
\def\sBtitle{\theexercise\ - Effizienz Teil B}

\section{\stitle}
\begin{frame}
  \frametitle{\stitle}%
\tableofcontents[current]
\end{frame}

\subsection{\sAtitle}
\begin{frame}%
  \frametitle{\sAtitle}%

\begin{enumerate}
\item Beschreiben Sie das Ergebnis der Rechnung.
\item Charakterisieren Sie den Rechenaufwand des Ausschnittes mit Hilfe eines Ausdrucks in der Landau-Notation.
\end{enumerate}

\lstinputlisting[style=JAVAsmall]{\getexercisefolder/Effizienz.java}
\end{frame}

%%%%%%%%%%%%%%%%%%%%%%%%%%%%%%%%%%%%%%%%%%%%%%%%%%%%%%%%%%%%%%%%%%%%%%%%
\begin{frame}%
  \frametitle{\sAtitle\ - L\"osungsvorschlag}%

\begin{enumerate}
  \item Beschreiben Sie das Ergebnis der Rechnung.
  \begin{itemize}
    \item Mit Hilfe von einer \code{for}-Schleife werden die Elemente des neuen Vektors \code{res} berechnet.
    \item Dabei wird mit der Hilfsvariablen \code{skp} und einer weiteren \code{for}-Schleife das Skalarprodukt der \code{mu}-ten Spalte der Matrix \code{a} und des Vektors \code{b} berechnet.
    \item Zum Schluss wird noch das jeweilige Element des Vektors \code{c} addiert.
    \item Das entspricht der Berechnung \code{d = a * b + c},
  \end{itemize}
  \item Charakterisieren Sie den Rechenaufwand des Ausschnittes mit Hilfe eines Ausdrucks in der Landau-Notation.
  \begin{itemize}
  \item Die \"au\ss ere Schleife wird \code{n}-mal durchgelaufen, das entspricht $n$-mal die Anzahl der Operationen in der inneren Schleife.
  \item Die innere Schleife wird auch \code{n}-mal durchgelaufen.
  Dort wird eine Addition und eine  Multiplikation durchgef\"uhrt.
  Dazu kommt noch eine Addition.
  Das ergibt $2n+1$ Operationen.
  \item Insgesamt sind  das wegen der Schachtelung der Schleifen $n(2n+1)=2n^2+n$ Operationen.
  Das Entspricht einer Komplexit\"at von $\mathcal{O}(n^2)$.
  \end{itemize}
\end{enumerate}
\end{frame}


%%%%%%%%%%%%%%%%%%%%%%%%%%%%%%%%%%%%%%%%%%%%%%%%%%%%%%%%%%%%%%%%%%%%%%%%
\subsection{\sBtitle}
\begin{frame}%
  \frametitle{\sBtitle}%

Vereinfachen Sie die folgenden Ausdr\"ucke zur Komplexit\"at in der Landau-Notation, z.B.~$\mathcal{O}(n^3+n) = \mathcal{O}(n^3)$:
\begin{enumerate}
  \item $\mathcal{O}(10n^3/(2n^2)+n^2)$
  \item $\mathcal{O}(n(n^3+n)-n^2)$
  \item $\mathcal{O}(e^{1000}+ln (n))$
  \item $\mathcal{O}(-1+e+(ln(n))*n +1/n )$
\end{enumerate}
\end{frame}

%%%%%%%%%%%%%%%%%%%%%%%%%%%%%%%%%%%%%%%%%%%%%%%%%%%%%%%%%%%%%%%%%%%%%%%%
\begin{frame}%
  \frametitle{\sBtitle\ - L\"osungsvorschlag}%

\begin{enumerate}
  \item $\mathcal{O}(10n^3/(2n^2)+n^2) = \mathcal{O}(5n + n^2) = \mathcal{O}(n^2)$
  \item $\mathcal{O}(n(n^3+n)-n^2) = \mathcal{O}(n^4 + n^2 - n^2) = \mathcal{O}(n^4)$
  \item $\mathcal{O}(e^{1000}+ln (n)) = \mathcal{O}(ln (n))$
  \item $\mathcal{O}(-1+e+(ln(n))*n +1/n) = \mathcal{O}((ln(n)) * n + 1/n) = \mathcal{O}((ln(n)) * n)$
\end{enumerate}
\end{frame}
