\AtBeginSection{}
%%%%%%%%%%%%%%%%%%%%%%%%%%%%%%%%%%%%%%%%%%%%%%%%%%%%%%%%%%%%%%%%%%%%%%%%
\section{Installation von Java und Text Editoren}
\begin{frame}
  \frametitle{\kap. Installation von Java und Text Editoren}%
\tableofcontents[current]
\end{frame}
% \setcounter{section}{0}


%%%%%%%%%%%%%%%%%%%%%%%%%%%%%%%%%%%%%%%%%%%%%%%%%%%%%%%%%%%%%%%%%%%%%%%%
\def\stitle{Installation von Java SE}%
\subsection{\stitle}\label{S:Compiler}
\begin{frame}[t]%
  \frametitle{\kap.\ref{S:Compiler} \stitle}%

\heading{Abhängig vom Betriebssystem variiert die Installation}
\begin{description}
  \item [Linux] \"Uber den Paketmanager, hier Ubuntu 18.04, mittels \\
  \code{\$ sudo apt install openjdk-11-jdk}
  \item[Windows] Installiere Java SE (beinhaltet JDK). Download unter \textcolor{KITblue}{\url{https://www.oracle.com/technetwork/java/javase/downloads/index.html}}
  \item[Alternativ] Ab Windows 10 v.1607 "{}Anniversary Update"{} kann das \emph{Windows Subsystem for Linux} (WSL) installiert werden.
    Damit steht unter Windows das Linux Terminal zur Verfügung.
    F\"ur die Installation siehe \textcolor{KITblue}{\url{https://docs.microsoft.com/en-us/windows/wsl/install-win10}}
    \textbf{Idee:} Editiere die Quell-Dateien in Windows, und übersetzte und führe das Programm in der Linux Konsole aus.
\end{description}

\vfill
In der Übung werden die Programme mit WSL/WSL2 entwickelt.
\end{frame}


%%%%%%%%%%%%%%%%%%%%%%%%%%%%%%%%%%%%%%%%%%%%%%%%%%%%%%%%%%%%%%%%%%%%%%%%
\def\stitle{Editoren}%
\subsection{\stitle}\label{S:Editor}
\begin{frame}[t]%
  \frametitle{\kap.\ref{S:Editor} \stitle}%

Den Quelltext eines Java-Programms k\"onnen Sie mit jedem Texteditor erstellen.
Achten Sie aber darauf, dass nur der reine Text und keine Formatierungen gespeichert wird (\textbf{keine} Textverarbeitungssoftware wie MS Word, LibreOffice).
\vfill

\heading{Geeigneter Editor unter Windos}
\begin{description}
  \item[Notepad++] Open-source Texteditor f\"ur Windows mit Syntax-Highlighting.
\end{description}

\heading{Geeignete Editoren unter Linux}
\begin{description}
  \item[gedit] Dieser Editor ist auf GNOME Desktop Umgebungen vorinstalliert, somit auf den Linux Distributionen Fedora und Ubuntu.
  \item[vim] Vim ist ein Konsolen-basierter Editor.
\end{description}

\vfill
Für die Programm Entwicklung mit WSL eignet sich Notepad++.
\end{frame}


%%%%%%%%%%%%%%%%%%%%%%%%%%%%%%%%%%%%%%%%%%%%%%%%%%%%%%%%%%%%%%%%%%%%%%%%
\def\stitle{Entwicklungsumgebungen}%
\subsection{\stitle}\label{S:IDE}
\begin{frame}[t]%
  \frametitle{\kap.\ref{S:IDE} \stitle}%

\heading{\textcolor{KIT-Rot}{Erfahrene} Programmierer arbeiten oft mit Entwicklungsumgebung (IDE: Integrated Development Environment).}

\begin{description}
  \item[Eclipse] Eine weitverbreitete open-source Entwicklungsumgebung die große Unterstützung aus der Industrie erhält, unter anderen von IBM, Bosch, CA Technologies, SAP, Oracle, siehe \textcolor{KITblue}{\url{www.eclipse.org}}.
  \item[NetBeans] Eine plattformunabh\"angige Entwicklungsumgebung, die im Rahmen eines von der Firma Sun Microsystems gef\"orderten Projekts entwickelt wird, siehe \textcolor{KITblue}{\url{www.netbeans.org}}.
  \item[VS Code] siehe \textcolor{KITblue}{\url{https://code.visualstudio.com/docs/languages/java}}
  \item[YCM] vim-plugin mit Autovervollständigung und vielem mehr, siehe \textcolor{KITblue}{\url{https://github.com/ycm-core/YouCompleteMe}}
\end{description}
\end{frame}
