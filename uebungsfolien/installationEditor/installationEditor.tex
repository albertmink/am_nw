\AtBeginSection{}
%%%%%%%%%%%%%%%%%%%%%%%%%%%%%%%%%%%%%%%%%%%%%%%%%%%%%%%%%%%%%%%%%%%%%%%%
\section{Installation von Java und Editoren}
\begin{frame}
  \frametitle{\kap. Installation von Java und Editoren}%
\tableofcontents[current]
\end{frame}
% \setcounter{section}{0}


%%%%%%%%%%%%%%%%%%%%%%%%%%%%%%%%%%%%%%%%%%%%%%%%%%%%%%%%%%%%%%%%%%%%%%%%
\def\stitle{Installation von Java}%
\subsection{\stitle}\label{S:Compiler}
\begin{frame}[t]%
  \frametitle{\kap.\ref{S:Compiler} \stitle}%
\medskip

Falls Sie zuhause an Ihrem PC Programme aus der Vorlesung ausf\"uhren und \"Ubungsaufgaben f\"ur das Praktikum entwickeln wollen, dann ben\"otigen Sie zun\"achst einen Compiler.
\medskip

\textbf{Installation eines Compilers}
\begin{description}
  \item [Linux] \"Uber den Paketmanager, hier Ubuntu 18.04, mittels \\
  \code{\$ sudo apt install openjdk-11-jdk}
  \item[Windows] Download Java SE (JDK, Java Development Kit) -- \underline{nicht} Java (JRE, Java Runtime Environment) -- via Internet Browser
  \item[Alternativ] Ab Windows 10 v.1607 "{}Anniversary Update"{} kann unter Windows ein Linux Subsytem installiert werden.
    Das ist besonders f\"ur Windows Nutzer attraktiv die nicht auf die Vorz\"uge von Linux verzichten wollen.
    F\"ur die Installation der \textbf{Bash on Ubuntu on Windows} siehe \textcolor{KITblue}{\url{https://docs.microsoft.com/en-us/windows/wsl/install-win10}}.
\end{description}
\end{frame}


%%%%%%%%%%%%%%%%%%%%%%%%%%%%%%%%%%%%%%%%%%%%%%%%%%%%%%%%%%%%%%%%%%%%%%%%
\def\stitle{Editoren}%
\subsection{\stitle}\label{S:Editor}
\begin{frame}[t]%
  \frametitle{\kap.\ref{S:Editor} \stitle\ (1)}%
\medskip

Java-Programme k\"onnen Sie grunds\"atzlich mit jedem Texteditor erstellen.
Achten Sie aber darauf, dass nur der reine Text ohne Formatierungen gespeichert wird (\textbf{keine} Textverabeitungssoftware wie MS Office).
Editoren die die Syntax von Java erkennt und farbig hervorheben (Syntax-Highlighting) sind vorteilhaft.
\medskip

\begin{description}
  \item[\textcolor{black}{\textbf{Windows}}]
  \item[Notepad++] Open-source Texteditor f\"ur Windows mit Syntax-Highlighting.
\end{description}
\end{frame}


%%%%%%%%%%%%%%%%%%%%%%%%%%%%%%%%%%%%%%%%%%%%%%%%%%%%%%%%%%%%%%%%%%%%%%%%
\begin{frame}[t]%
  \frametitle{\kap.\ref{S:Editor} \stitle\ (2)}%

\begin{description}
  \item[\textcolor{black}{\textbf{Linux}}]
  \item[gedit] Dieser Editor ist auf s\"amtlichen GNOME Desktop Umgebungen vorinstalliert, somit auf der Linux Distribution Fedora.
    Auch Ubuntu, das allerdings auf eine andere Desktop Umgebung ausgeliegert wird, bietet standartm\"assig gedit an.
    Gedit bietet umfangreiche Plugins an und unterst\"utzt nat\"urlich Syntax-Highlighting.
  \item[vim bzw. nano] Vim ist eine Weiterentwicklung des Texteditors vi und ist wie nano ein rein konsolen-basierter Editor.
    Unterst\"utzt Syntax-Highlighting und sehr vieles dar\"uber hinaus.
  \item[Kate (KDE)] Standard Texteditor auf Linux Distributionen mit KDE Desktop Umgebung.
      Das Syntax-Highlighting f\"ur Java ist automatisch aktiviert wenn die ge\"offnete Datei als .java-Datei abgespeichert wird.
  \item[EMacs] Einer der bekanntesten Linux Texteditoren (unterst\"utzt Syntaxhervorhebung).
      EMacs unterscheidet sich von der Bedienung her grundlegend von anderen Texteditoren.
\end{description}
\end{frame}


%%%%%%%%%%%%%%%%%%%%%%%%%%%%%%%%%%%%%%%%%%%%%%%%%%%%%%%%%%%%%%%%%%%%%%%%
\def\stitle{Integrierte Entwicklungsumgebungen}%
\subsection{\stitle}\label{S:IDE}
\begin{frame}[t]%
  \frametitle{\kap.\ref{S:IDE} \stitle}%
\medskip

{\color{red}Erfahrene Java-Programmierer} benutzen anstelle eines Texteditors eine integrierte Entwicklungsumgebung (IDE: Integrated Development Environment) zum Erstellen von Java-Programmen.
IDE-s bieten umfangreiche Funktionalit\"aten die den Anwender bei der Programmentwicklung unterst\"utzen, so wird in der Regel ein Compiler mit installiert.
\medskip

\begin{description}
  \item[Eclipse] Eine weitverbreitete open-source Entwicklungsumgebung f\"ur Windows und Linux, die gro\ss e Unterst\"utzung aus der Industrie erh\"alt, IBM, Bosch, CA Technologies, SAP, Oracle, RedHat.
    Aktuell wird kein Compiler mit ausgeliefert, bzw. muss von Hand installiert werden, siehe \textcolor{KITblue}{\url{www.eclipse.org}}.
  \item[NetBeans] Eine plattformunabh\"angige Entwicklungsumgebung, die im Rahmen eines von der Firma Sun Microsystems gef\"orderten Projekts entwickelt wird.
    Kann von der Webseite \textcolor{KITblue}{\url{www.netbeans.org}} herunter geladen werden und beinhaltet einen Java Compiler.
\end{description}
\end{frame}
