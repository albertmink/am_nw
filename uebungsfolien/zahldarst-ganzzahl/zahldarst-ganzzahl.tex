\def\stitle{\theexercise\ - Ganzzahl-Datentypen}
\section{\stitle}
\begin{frame}%
  \frametitle{\stitle}
\tableofcontents[current]
\end{frame}

\begin{frame}[fragile]%
    \frametitle{\stitle}%

In Java besitzen die Ganzzahl-Datentypen die folgenden Größen
\begin{center}
\begin{tabular}{|c|c|}
\hline
\textbf{Datentyp} & \textbf{Größe} \\
\hline
byte  &  8 Bit \\
short & 16 Bit \\
int   & 32 Bit \\
long  & 64 Bit \\
\hline
\end{tabular}
\end{center}
Das führende Bit jedes Datentypen gibt das Vorzeichen der Zahl an.
Ein 0-Bit zeigt eine nichtnegative Zahl an, ein 1-Bit eine negative Zahl.
Zum Beispiel ist $0111 1111_2=127$ und $1000 0000_2 = -128$ die kleinste und größte darstellbare Zahl des Datentyps \code{byte}.
\begin{itemize}
\item Wie viele verschiedene ganze Zahlen können mit den Datentypen \code{byte}, \code{short}, \code{int} und \code{long} dargestellt werden?

\item[Lsg]
jedes Bitmuster eines Datentyps entspricht genau einer ganzen Zahl.
Jedes Bit kann genau zwei Werte annehmen.
Daher können mit den Datentypen \code{byte}, \code{short}, \code{int} und \code{long} jeweils $2^8$, $2^{16}$, $2^{32}$, bzw. $2^{64}$ verschiedene ganze Zahlen dargestellt werden.

\end{itemize}

\end{frame}

\begin{frame}[fragile]%
  \frametitle{Fortsetzung}%
\centering
\medskip

\begin{itemize}
\item
Welches sind jeweils die größten positiven, ganzen Zahlen, die mit den Datentypen \code{byte}, \code{short}, \code{int} und \code{long} dargestellt werden können?
\item[Lsg]
Der Datentyp \code{byte} mit 8 Bit kann maximal sieben-stellige, positive Binärzahlen darstellen, das führende Bit bei solchen Zahlen ein 0-Bit sein muss.
Die größte positive, ganze Zahl lautet daher $1111111_2 = 127 = 128 - 1 = 2^7 - 1$.
Für die Datentypen \code{short}, \code{int} und \code{long} erhält man in gleicher Weise $2^{15} - 1$, $2^{31} - 1$ bzw. $2^{63} - 1$ als größte darstellbare positive, ganze Zahl.
\end{itemize}

\end{frame}

\begin{frame}[fragile]%
  \frametitle{Fortsetzung}%
\centering
\medskip

\begin{itemize}
\item
Betrachten Sie den folgenden Quelltextausschnitt:
\begin{verbatim}
byte a = 100;
a += 100;
\end{verbatim}
Stellt der Wert der Variable \code{a} nach Ausführung dieser Zeilen die Zahl $200$ dar?
Begründen Sie Ihre Antwort.

\item[Lsg]
Nein, der Wert der Variable a stellt nach Ausführung der Quelltextzeilen die Zahl $200$ nicht dar.
Der Datentyp der Variable ist \code{byte}.
Die größte positive, ganze Zahl, die von diesem Datentyp dargestellt werden kann, ist $2^7 - 1 = 127$. Die Zahl $200$ ist somit nicht darstellbar.
Man kann die Antwort noch weiter erläutern: Die Binärdarstellung der Zahl $100 = 64 + 32 + 4$ lautet $1100100_2$.
Der Datentyp \code{byte} stellt sie als das Bitmuster $01100100$ dar.
Das führende 0-Bit zeigt an, dass die Zahl nichtnegativ ist.
Addiert man nun $100$ zu der Zahl hinzu, so erhält man $200 = 2 \cdot 100 = 128 + 64 + 8$, in Binärdarstellung $11001000_2$.
Dem entspricht das Bitmuster $11001000$.
Da das führende Bit ein 1-Bit ist, stellt dieses Bitmuster im Datentyp \code{byte} eine negative Zahl dar.
\end{itemize}
\end{frame}