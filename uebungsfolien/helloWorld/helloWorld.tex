\def\stitle{Hello World}
%%%%%%%%%%%%%%%%%%%%%%%%%%%%%%%%%%%%%%%%%%%%%%%%%%%%%%%%%%%%%%%%%%%%%%%%
\section{\stitle}
\begin{frame}
  \frametitle{\kap. \stitle}%
\tableofcontents[current]
\end{frame}


%%%%%%%%%%%%%%%%%%%%%%%%%%%%%%%%%%%%%%%%%%%%%%%%%%%%%%%%%%%%%%%%%%%%%%%%
\subsection{Hello World}\label{S:helloWorld}
\begin{frame}[fragile]%
  \frametitle{\kap.\ref{S:helloWorld} \stitle\ - Quelltext}%

\lstinputlisting[style=JAVA,title=Diese kleine minimal Beispiel hei\ss t HelloWorld und gibt "Hello World!"{} auf der Konsole aus.]
{\getexercisefolder /HelloWorld.java}
\end{frame}


%%%%%%%%%%%%%%%%%%%%%%%%%%%%%%%%%%%%%%%%%%%%%%%%%%%%%%%%%%%%%%%%%%%%%%%%
\begin{frame}[fragile]%
  \frametitle{\kap.\ref{S:helloWorld} \stitle\ - \"Ubersetzen und Ausf\"uhren}%

\begin{lstlisting}[title={Um das Programm HelloWorld auszuf\"uhren werden folgende Schritte auf dem Terminal durchgef\"uhrt.},style=BASH]
$ javac HelloWorld.java
$ java HelloWorld
Hello World!
\end{lstlisting}
\end{frame}


%%%%%%%%%%%%%%%%%%%%%%%%%%%%%%%%%%%%%%%%%%%%%%%%%%%%%%%%%%%%%%%%%%%%%%%%
\def\stitle{Operatoren (Beispiele)}%
\subsection{\stitle}\label{S:operator}
\begin{frame}[t]%
  \frametitle{\kap.\ref{S:operator} \stitle}%
\medskip
\heading{Rechenoperatoren:}
\begin{itemize}
\item \code{1 + 1}
\item \code{1-99}
\item \code{3*3}
\item \code{10/2}
\item \code{3.14159*2}
\end{itemize}
\medskip
\heading{Wertzuweisung:}
\begin{itemize}
\item \code{int x = 27;}
\item \code{int y = 3*x+2;}
\item \code{double u = 3.14159*2;}
\end{itemize}
\end{frame}


%%%%%%%%%%%%%%%%%%%%%%%%%%%%%%%%%%%%%%%%%%%%%%%%%%%%%%%%%%%%%%%%%%%%%%%%
\def\stitle{Datentypen}%
\subsection{\stitle}\label{S:type}
\begin{frame}[t]%
  \frametitle{\kap.\ref{S:type} \stitle }%
\medskip
Ganze Zahlen (Integer): \code{int}\\
\begin{itemize}
  \item von $-2147483648$ bis $2147483647$
  \item Bei verlassen des Bereichs: Overflow
\end{itemize}
\medskip
Gleitkommazahl (double): \code{double}\\
\begin{itemize}
 \item au\ss erhalb des darstellbaren Bereichs: Gibt \lstinline|NaN| oder \lstinline|Inf| zur\"uck
 \item Rechengenauigkeit ca. $10^{-16}$
 \item Vorsicht bei "`Integer-Division"'\\
 \lstinline|double r = 5/2;| $\rightarrow$ \lstinline|r = 2;|
\end{itemize}
\medskip
Verwendung von Konstanten (\code{const})
\begin{itemize}
  \item Der Wert einer mit \lstinline|const| erstellten Variable kann nicht ver\"andert werden.
  \item Eine Deklaration (z.B. \lstinline|const int i;|) ist nicht ausreichend, da sp\"ater kein Wert zugewiesen werden kann.
  \item Beispiel: \lstinline|const int pi=3.14;|
\end{itemize}
\end{frame}
