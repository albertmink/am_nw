\def\stitle{\theexercise\ - Felder: Berechnungen}
\section{\stitle}
\begin{frame}
  \frametitle{\stitle}%
\tableofcontents[current]
\end{frame}



\begin{frame}[t]% 
    \frametitle{\stitle ~a)}
Betrachten Sie den folgenden Ausschnitt eines Java-Programms.
\lstinputlisting[style=JAVAlines]{\getexercisefolder/Felderb1.java}
\begin{itemize}
    \item Was wird auf der Konsole ausgegeben?
    \item Was wird in diesem Programm berechnet?
    \pause
    \item 3, 4, Fertig mit 4.
    \item Der maximalen Eintrag des Feldes.
\end{itemize}
\end{frame}

\begin{frame}[t]% 
    \frametitle{\stitle ~b)}
Betrachten Sie den folgenden Ausschnitt eines Java-Programms.
\lstinputlisting[style=JAVAlines]{\getexercisefolder/Felderb2.java}
\begin{itemize}
    \item Was wird auf der Konsole ausgegeben?
    \item Was wird in diesem Programm berechnet?
    \pause
    \item 1, 4, 8, 10, 13
    \item Die Summe aller Eintr"age des Feldes, bzw. die $L1$-Norm.
\end{itemize}
\end{frame}