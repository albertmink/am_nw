\def\stitle{\theexercise\ - Komplexe Zahlen}
\section{\stitle}
\begin{frame}%
  \frametitle{\stitle}%
\tableofcontents[current]
\end{frame}
%%%%%%%%%%%%%%%%%%%%%%%%%%%%%%%%%%%%%%%%%%%%%%%%%%%%%%%%%%%%%%%%%%%%%%%%
\begin{frame}%
  \frametitle{\stitle}%

Unter einer \emph{komplexen Zahl} $z$ versteht man eine Zahl der Form $z = a + b\mathrm{i}$, wobei $a$ und $b$ zwei reelle Zahlen und $\mathrm{i}$ die sogenannte \emph{imagin\"are Einheit} ist.
Die imagin\"are Einheit ist durch $\mathrm{i}^2 = -1$ definiert.
Die Zahl $a$ hei\ss t \emph{Realteil}, die Zahl $b$ \emph{Imagin\"arteil} von $z$.
Die Menge der komplexen Zahlen wird mit $\mathbb{C}$ bezeichnet und kann als algebraische Erweiterung der reellen Zahlen $\mathbb{R}$ verstanden werden.
Der \emph{Betrag} $\lvert z\rvert$ einer komplexen Zahl $z = a + b\mathrm{i}$ ist wie folgt definiert:
\begin{equation*}
\lvert z \rvert = \sqrt{a^2 + b^2}.
\end{equation*}
F\"ur die \emph{Addition} zweier komplexer Zahlen $z = a + b\mathrm{i}$ und $w = c + d\mathrm{i}$ gilt:
\begin{equation*}
z + w =(a+c) + (b+d)\mathrm{i}.
\end{equation*}
F\"ur die \emph{Multiplikation} gilt entsprechend
\begin{equation*}
z w =(ac - bd) + (ad + bc)\mathrm{i}.
\end{equation*}
Diese Formel folgt nach Ausmultiplizieren direkt aus der Definition $\mathrm{i}^2 = -1$.
\end{frame}


%%%%%%%%%%%%%%%%%%%%%%%%%%%%%%%%%%%%%%%%%%%%%%%%%%%%%%%%%%%%%%%%%%%%%%%%
\begin{frame}%
  \frametitle{\stitle}%
Schreiben Sie ein Java--Programm, welches das Rechnen mit komplexen Zahlen erm\"oglicht.
\begin{enumerate}
\item[1.]
  Erstellen Sie eine \"offentliche Klasse namens \code{Komplex} die eine komplexe Zahl repr\"asentiert.
  Definieren Sie f\"ur diese Klasse zwei private Instanzvariablen namens \code{real} und \code{imag} vom Typ \code{double}.
\item[2.]
  Definieren Sie einen \"offentlichen Konstruktor mit zwei formalen Parametern namens \code{real} und \code{imag} vom Typ \code{double}.
\end{enumerate}
\pause
\lstinputlisting[style=JAVAsmall,firstline=1,lastline=12]{\getexercisefolder/Komplex.java}
\end{frame}


%%%%%%%%%%%%%%%%%%%%%%%%%%%%%%%%%%%%%%%%%%%%%%%%%%%%%%%%%%%%%%%%%%%%%%%%
\begin{frame}%
  \frametitle{\stitle}%
\begin{enumerate}
\item[3.]
  Definieren Sie f\"ur die Klasse \code{Komplex} eine \"offentliche Instanzmethode namens \code{toString}, ohne formale Parameter.
  Die Methode soll die Zeichenkette \glqq $a$ + i$b$\grqq\ als Wert vom Typ \code{String} zur\"uck geben, wobei $a$ und $b$ durch die Werte der Instanzvariablen \code{real} und \code{imag} zu ersetzen sind.
  Wenn Sie nun den Befehl \code{System.out.println(z);} f\"ur eine Instanz \code{z} der Klasse \code{Komplex} aufrufen, wird die von der Methode \code{toString} zur\"uckgegebene Zeichenkette auf der Konsole ausgegeben.
\end{enumerate}
\pause
\lstinputlisting[style=JAVA,firstline=14,lastline=17]{\getexercisefolder/Komplex.java}
\end{frame}


%%%%%%%%%%%%%%%%%%%%%%%%%%%%%%%%%%%%%%%%%%%%%%%%%%%%%%%%%%%%%%%%%%%%%%%%
\begin{frame}%
  \frametitle{\stitle}%
\begin{enumerate}
\item[4.]
  Definieren Sie f\"ur die Klasse \code{Komplex} eine \"offentliche Instanzmethode namens \code{betrag} ohne formale Parameter, welche den Betrag der komplexen Zahl berechnet und als Wert vom Typ \code{double} zur\"uckgibt.
\end{enumerate}
\pause
\lstinputlisting[style=JAVAsmall,firstline=19,lastline=22]{\getexercisefolder/Komplex.java}
\end{frame}


%%%%%%%%%%%%%%%%%%%%%%%%%%%%%%%%%%%%%%%%%%%%%%%%%%%%%%%%%%%%%%%%%%%%%%%%
\begin{frame}%
  \frametitle{\stitle}%
\begin{enumerate}
\item[5.]
  Definieren Sie f\"ur die Klasse \code{Komplex} zwei \"offentliche Klassenmethoden mit den Namen \code{addieren} und \code{multiplizieren}.
  Versehen Sie beide Methoden mit zwei formalen Parametern vom Typ \code{Komplex}.
  Berechnen Sie in den Methoden die Summe bzw{.} das Produkt der \"ubergebenen komplexen Zahlen und geben Sie das Ergebnis jeweils als Wert vom Typ \code{Komplex} zur\"uck.
\end{enumerate}
\pause
\lstinputlisting[style=JAVAsmall,firstline=24,lastline=36]{\getexercisefolder/Komplex.java}
\end{frame}


%%%%%%%%%%%%%%%%%%%%%%%%%%%%%%%%%%%%%%%%%%%%%%%%%%%%%%%%%%%%%%%%%%%%%%%%
\begin{frame}%
  \frametitle{\stitle}%
\begin{enumerate}
\item[6.]
  Erstellen Sie eine \"offentliche Klasse namens \code{KomplexeArithm} mit der \code{main}-Methode des Programms.
  Lesen Sie in der \code{main}-Methode den Real- und Imagin\"arteil zweier komplexer Zahlen von der Konsole ein und erzeugen Sie zwei entsprechende Instanzen der Klasse \code{Komplex}.
  Geben Sie die Betr\"age beider komplexer Zahlen, sowie deren Summe und Produkt auf der Konsole aus.
\end{enumerate}

\end{frame}

%%%%%%%%%%%%%%%%%%%%%%%%%%%%%%%%%%%%%%%%%%%%%%%%%%%%%%%%%%%%%%%%%%%%%%%%
\begin{frame}%
  \frametitle{\stitle}%

\lstinputlisting[style=JAVAfootnote]{\getexercisefolder/KomplexeArithm.java}
\end{frame}



%%%%%%%%%%%%%%%%%%%%%%%%%%%%%%%%%%%%%%%%%%%%%%%%%%%%%%%%%%%%%%%%%%%%%%%%
\begin{frame}%
  \frametitle{\stitle}%
\begin{enumerate}
\item[7.]
  Testen Sie Ihr Programm mit den folgenden Daten:
\begin{align*}
  z &= 1 + 1\mathrm{i}, 
& w &= 2 + 2\mathrm{i}, 
& \lvert z \rvert &= \sqrt{2}, 
& \lvert w \rvert &= \sqrt{8},
& z + w &= 3 + 3\mathrm{i},
& zw    &= 4\mathrm{i}; \\
  u &= 2\mathrm{i}, 
& v &= 4\mathrm{i}, 
& \lvert u \rvert &= 2, 
& \lvert v \rvert &= 4,
& u + v &= 6\mathrm{i},
& uv    &= -8.
\end{align*}
\end{enumerate}
\end{frame}
