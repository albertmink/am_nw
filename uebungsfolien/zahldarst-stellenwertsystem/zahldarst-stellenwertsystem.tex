\def\stitle{\theexercise\ - Stellenwertsysteme}
\section{\stitle}
\begin{frame}%
  \frametitle{\stitle}%
\tableofcontents[current]
\end{frame}


\begin{frame}[t]%
  \frametitle{\stitle}%


Bei \emph{Stellenwertsystemen} wird eine Zahl durch eine Folge ${z_n z_{n-1} \dotsb z_1 z_0}$ von \emph{Ziffern} $z_k$ dargestellt.
Jedes Stellenwertsystem bezieht sich dabei auf eine bestimmte \emph{Basis} $b$, die man bei Bedarf als Subskript an das Ende der Ziffernfolge schreibt.
Für die Ziffern gilt $z_k \in \{0,1,\dotsc,b-1\}$.
Eine Ziffernfolge ${z_n z_{n-1} \dotsb z_1 z_0}$ im Stellenwertsystem zur Basis $b$ stellt die Zahl
\[  z_n \cdot b^{n} + z_{n-1} \cdot b^{n-1} + \dotsb + z_1 \cdot b^1 + z_0 \cdot b^0  \]
dar.
Wichtige Stellenwertsysteme sind das \emph{Dezimalsystem} ($b = 10$), das \emph{Binärsystem} ($b = 2$), das \emph{Oktalsystem} ($b = 8$) und das \emph{Hexadezimalsystem} ($b = 16$).
Geben Sie die Darstellung der folgenden  Binär-, Oktal-, Hexadezimal- und Dezimalzahlen in den jeweils anderen Stellenwertsystemen an.
Im Hexadezimalsystem bezeichnen die Buchstaben $\mathrm{A},\mathrm{B}, \dotsc, \mathrm{F}$ die Ziffern $10, 11, \dotsc, 15$.
\begin{center}
\begin{minipage}{0.22\textwidth}
\begin{enumerate}
\item[(a)] $10_2$
\item[(b)] $1100_2$
\item[(c)] $10101_2$
\item[(d)] $101010_2$
\end{enumerate}
\end{minipage}
\begin{minipage}{0.22\textwidth}
\begin{enumerate}
\item[(e)] $27_8$
\item[(f)] $133_8$
\item[(g)] $10_8$
\item[(h)] $77_8$
\end{enumerate}
\end{minipage}
\begin{minipage}{0.22\textwidth}
\begin{enumerate}
\item[(i)] $\mathrm{2A}_{16}$
\item[(j)] $10_{16}$
\item[(k)] $\mathrm{FF}_{16}$
\item[(l)] $\mathrm{D1}_{16}$
\end{enumerate}
\end{minipage}
\begin{minipage}{0.22\textwidth}
\begin{enumerate}
\item[(m)] $17_{10}$
\item[(n)] $33_{10}$
\item[(o)] $65_{10}$
\item[(p)] $72_{10}$
\end{enumerate}
\end{minipage}
\end{center}


\end{frame}


\begin{frame}[fragile]
  \frametitle{\stitle\ Lsg}

\begin{table}
\caption{Binär}
\begin{tabular}{l|l|l|l|l|l|l}
$2^6$ & $2^5$ & $2^4$ & $2^3$ & $2^2$ & $2^1$ & $2^0$ \\ \hline
$64$  & $32$  & $16$  & $8$   & $4$   &$1$    & $2$
\end{tabular}
\end{table}
\medskip

Siehe Teil $(f)$\\
enkodiere in dezimal: \\
$\textcolor{KITred}{133}_8
=
\textcolor{KITred}{1}*8^2 + \textcolor{KITred}{3}*8^1 + \textcolor{KITred}{3}*8^0 = 91_{10} = 91$
\medskip

dekodiere in binär: \\
$ 91_{10}
=
\textcolor{KITred}{1}*2^6 + \textcolor{KITred}{0}*2^5 + \textcolor{KITred}{1}*2^4 + \textcolor{KITred}{1}*2^3 + \textcolor{KITred}{0}*2^2 + \textcolor{KITred}{1}*2^1 + \textcolor{KITred}{1}*2^0 = \textcolor{KITred}{1011011}_2$

dekodiere in oktal: \\
$91_{10}
=
\textcolor{KITred}{5}*16^1 + \textcolor{KITred}{11}*16^0 = \textcolor{KITred}{5B}_{16}
$

\end{frame}