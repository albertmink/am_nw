\def\stitle{\theexercise\ - Seiteneffekte}
\section{\stitle}
%%%%%%%%%%%%%%%%%%%%%%%%%%%%%%%%%%%%%%%%%%%%%%%%%%%%%%%%%%%%%%%%%%%%%%%%
\begin{frame}
  \frametitle{\stitle}%
\tableofcontents[current]
\end{frame}

%%%%%%%%%%%%%%%%%%%%%%%%%%%%%%%%%%%%%%%%%%%%%%%%%%%%%%%%%%%%%%%%%%%%%%%%
\begin{frame}[t]%
    \frametitle{\stitle}

  \lstinputlisting[style=JAVAfootnotelines]{\getexercisefolder/FelderFunktionen.java}

\end{frame}

%%%%%%%%%%%%%%%%%%%%%%%%%%%%%%%%%%%%%%%%%%%%%%%%%%%%%%%%%%%%%%%%%%%%%%%%
\begin{frame}[t]%
\frametitle{\theexercise - Pass by value}

Primitive Datentypen wie \code{int, double, char, ...} werden als Kopie an Funktionen übergeben.
Das bedeutet hier, dass die Variable \code{a} durch den Funktionsaufruf nicht geändert wird.

\lstinputlisting[style=JAVAfootnote,frame=single,linerange={2-6,12-16,20}]{\getexercisefolder/FelderFunktionen.java}
Liste der primitiven Datentypen in Java \textcolor{KITblue}{\url{https://docs.oracle.com/javase/tutorial/java/nutsandbolts/datatypes.html}}
\end{frame}


%%%%%%%%%%%%%%%%%%%%%%%%%%%%%%%%%%%%%%%%%%%%%%%%%%%%%%%%%%%%%%%%%%%%%%%%
\begin{frame}[t]%
\frametitle{\theexercise - Pass by value}

Ein array ist kein primitiver Datentyp sondern ein (Kontainer) Objekt.
Der ursprüngliche Werte kann bei Funktionsaufrufen verändert werden.
\emph{Achtung!}
\lstinputlisting[style=JAVAfootnote,frame=single,linerange={2-2,8-11,12-12,17-20}]{\getexercisefolder/FelderFunktionen.java}

Dokumentation des Datentyps array \textcolor{KITblue}{\url{https://docs.oracle.com/javase/tutorial/java/nutsandbolts/arrays.html}}

\end{frame}
