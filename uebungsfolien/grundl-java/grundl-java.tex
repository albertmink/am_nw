\def\stitle{\theexercise\ - Programmentwicklung}
\section{\stitle}
\begin{frame}%
  \frametitle{\stitle}%
\tableofcontents[current]
\end{frame}


\begin{frame}%
  \frametitle{\stitle}%

\heading{Wozu dient der Java-Compiler, wozu der Java-Interpreter?}

\begin{itemize}
\item Der Java-Compiler übersetzt den Quelltext in einen sogenannten \emph{Bytecode}.
\item Der Java-Interpreter auch Laufzeitumgebung genannt, lädt den Bytecode und führt in in der Java Virtual Machine (JVM) aus.
\end{itemize}
\includegraphics[width=\textwidth]{\getexercisefolder funktion}

\end{frame}


\begin{frame}[t]%
  \frametitle{\stitle\ Fortsetzung}%

\heading{Erläutern Sie die Aussage \glqq Java ist plattformunabhängig\grqq.}

\begin{itemize}
\item Der Bytecode kann von jedem Java-Interpreter (JVM) ausgeführt werden.
\item Also ist der Bytecode nur an die JVM gebunden aber nicht an ein bestimmtes Betriebssystem oder bestimmte Hardware.
\end{itemize}
\centering
\includegraphics[width=0.6\textwidth]{\getexercisefolder compiler}

\end{frame}