\AtBeginSection{}
%%%%%%%%%%%%%%%%%%%%%%%%%%%%%%%%%%%%%%%%%%%%%%%%%%%%%%%%%%%%%%%%%%%%%%%%
\section{Praktikum}
\begin{frame}
  \frametitle{\kap. Praktikum}%
\tableofcontents[currentsection]
\end{frame}
% \setcounter{section}{0}


%%%%%%%%%%%%%%%%%%%%%%%%%%%%%%%%%%%%%%%%%%%%%%%%%%%%%%%%%%%%%%%%%%%%%%%%
\def\stitle{Bearbeitung der Praktikumsaufgaben}%
\subsection{\stitle}\label{S:PraktikumSCC}
\begin{frame}[t]%
  \frametitle{\kap.\ref{S:PraktikumSCC} \stitle}%

\heading{Praktikumsrechner am SCC}
\begin{itemize}
  \item Linux Betriebssystem
  \item Software
  \begin{itemize}
    \item Eingabefenster (Shell, Terminal oder Konsole)
    \item Editor mit Syntax-Hervorhebung z.B. Gedit)
    \item Internet-Browser
    \item pdf-Betrachter
  \end{itemize}
  \item Kurze Einführung: \emph{Anleitung und Informationen zum Praktikum mit den Sprachen C++ und Java} (auf ILIAS)
\end{itemize}
\end{frame}


%%%%%%%%%%%%%%%%%%%%%%%%%%%%%%%%%%%%%%%%%%%%%%%%%%%%%%%%%%%%%%%%%%%%%%%%
\def\stitle{Grundlagen der Linux Konsole}%
\subsection{\stitle}\label{S:Anleitung}
\begin{frame}[t]%
\frametitle{\kap.\ref{S:Anleitung} \stitle\ (1)}%

\heading{Generelle Bemerkungen zur Konsole}
\begin{itemize}
  \item Laufende Programme bzw. Befehlsausführungen können durch die Tastenkombination \textbf{Strg+c} abgebrochen werden
  \item Für Autovervollständigung von Dateinamen und Pfaden in der Konsole zwei Mal \textbf{Tab}
  \item Für Hilfestellungen und Dokumentation \textbf{man} bzw. \textbf{-{}-help} und \textbf{google}
\end{itemize}
\end{frame}


%%%%%%%%%%%%%%%%%%%%%%%%%%%%%%%%%%%%%%%%%%%%%%%%%%%%%%%%%%%%%%%%%%%%%%%%
\begin{frame}[t]%
\frametitle{\kap.\ref{S:Anleitung} \stitle\ (2)}%

\heading{Die wichtigsten UNIX-Kommandos zum navigieren}
\begin{itemize}
  \setlength{\itemsep}{4pt}
  \item Aktuelles Verzeichnis ausgeben: \textbf{pwd (print working directory)}
  \item Verzeichnis wechseln: \textbf{cd (change directory)}
  \begin{itemize}
    \setlength{\itemsep}{2pt}
    \item \textbf{cd /tmp} Wechsel in das Verzeichnis /tmp (absoluter Pfad)
    \item \textbf{cd work/dat} Wechsel in das Unterverzeichnis dat von work (relativer Pfad)
    \item \textbf{cd} Wechsel in Ihr Home-Verzeichnis
    \item \textbf{cd ..} Wechsel in das übergeordnete Verzeichnis
  \end{itemize}
  \item Inhalt des aktuellen Verzeichnisses auflisten: \textbf{ls (list)}
  \begin{itemize}
    \setlength{\itemsep}{2pt}
    \item \textbf{ls -{}- help} anzeigen der Dokumenation, alt. \textbf{man ls}
    \item \textbf{ls -l} zeigt Inhalt als Liste an. Verzeichnisse, Archive und ausführbare Dateien werden eingefärbt
    \item \textbf{ls -lh} zeigt Inhalt zusätzlich als human readable an
    \item \textbf{ls -lhS} zeigt Inhalt zusätzlich der Grö\ss e nach geordnet an
    \item $\ldots$
  \end{itemize}
\end{itemize}
\end{frame}


%%%%%%%%%%%%%%%%%%%%%%%%%%%%%%%%%%%%%%%%%%%%%%%%%%%%%%%%%%%%%%%%%%%%%%%%
\begin{frame}[t]%
\frametitle{\kap.\ref{S:Anleitung} \stitle\ (3)}%

\heading{Die wichtigsten UNIX-Kommandos zum erstellen und löschen von Verzeichnissen}
\begin{itemize}
  \setlength{\itemsep}{4pt}
  \item Neues Verzeichnis anlegen: \textbf{mkdir (make directory)}
  \begin{itemize}
    \item \textbf{mkdir neu} legt das Verzeichnis neu im aktuellen Verzeichnis an
  \end{itemize}
  \item Neue Datei erstellen: \textbf{touch}
  \begin{itemize}
    \item \textbf{touch helloworld.java} erstellt die Datei helloworld.java
  \end{itemize}
  \item Löschen einer Datei: \textbf{rm (remove)}
  \begin{itemize}
    \setlength{\itemsep}{2pt}
    \item \textbf{rm helloworld.java} löscht die Datei helloworld.java
    \item \textbf{rm -r helloworld.java} löscht rekursiv, also auch gesamte Verzeichnisse
    \item \textbf{rm -f helloworld.java} erzwingt das Löschen
    \item \textbf{rm -v helloworld.java} aktiviert die Erklärung was der Befehl bewirkt
  \end{itemize}
  \item Abrufen der Dokumentation: \textbf{man}
  \begin{itemize}
    \item \textbf{man rm} ruft die Dokumentation des Befehls \textbf{rm} auf. Mit \textbf{q (quit)} gelangt man zurück.
  \end{itemize}
\end{itemize}
\end{frame}


%%%%%%%%%%%%%%%%%%%%%%%%%%%%%%%%%%%%%%%%%%%%%%%%%%%%%%%%%%%%%%%%%%%%%%%%
\begin{frame}[t]%
\frametitle{\kap.\ref{S:Anleitung} \stitle\ (4)}%

\heading{Die wichtigsten UNIX-Kommandos zum kopieren}
\begin{itemize}
  \setlength{\itemsep}{4pt}
  \item Befehl \textbf{cp (copy) [OPTION] <SOURCE> <DESTINATION>}
  \begin{itemize}
    \setlength{\itemsep}{2pt}
    \item \textbf{cp work.tex final.tex} erstellt eine Kopie von work.tex die final.tex hei\ss t
    \item \textbf{cp -r folderSource folderDest} kopiert rekursiv und damit auch Verzeichnisse
    \item Optionen: \textbf{-v, --verbose; -r, --recursive; -f, --force}
  \end{itemize}
\end{itemize}
\end{frame}


%%%%%%%%%%%%%%%%%%%%%%%%%%%%%%%%%%%%%%%%%%%%%%%%%%%%%%%%%%%%%%%%%%%%%%%%
\def\stitle{Programmentwicklung}%
\subsection{\stitle}\label{S:Progentw}
\begin{frame}[fragile]%
\frametitle{\kap.\ref{S:Progentw} \stitle}%

\heading{Übersetzen des Programms}
\begin{itemize}
  \item \textbf{Wichtig:} Java Programme müssen mit .java enden
  \item Java-Quelltext wird durch Aufruf des Java-Compilers in Java-Bytecode übersetzt \code{javac Dateiname.java}
  \item Zur Programm Ausführung muss der Java-Interpreter aufgerufen werden \code{java Dateiname}
\end{itemize}

\heading{Beispiel: Kugelvolumen}
\begin{itemize}
  \item Der Java-Quelltext befindet sich in der Datei \code{KugelVolumen.java}
  \item Mit \code{javac KugelVolumen.java} wird der Quelltext in Bytecode übersetzt
  \item Der Aufruf \code{java KugelVolumen} startet das Programm in der Java Virtual Machine
  \item Auf dem Bildschirm erscheint folgende Ausgabe:
  \begin{lstlisting}[style=bash]
  Bitte Kugelradius eingeben:
  > 1
  Das Volumen betraegt v = 4.1887902047863905
  \end{lstlisting}
\end{itemize}
\end{frame}
