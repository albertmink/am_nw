\AtBeginSection{}
%%%%%%%%%%%%%%%%%%%%%%%%%%%%%%%%%%%%%%%%%%%%%%%%%%%%%%%%%%%%%%%%%%%%%%%%
\section{Praktikum}
\begin{frame}
  \frametitle{\kap. Praktikum}%
\tableofcontents[currentsection]
\end{frame}
% \setcounter{section}{0}


%%%%%%%%%%%%%%%%%%%%%%%%%%%%%%%%%%%%%%%%%%%%%%%%%%%%%%%%%%%%%%%%%%%%%%%%
\def\stitle{Bearbeitung der Praktikumsaufgaben}%
\subsection{\stitle}\label{S:PraktikumSCC}
\begin{frame}[t]%
  \frametitle{\kap.\ref{S:PraktikumSCC} \stitle}%
\medskip

Praktikumsrechner des SCC
\begin{itemize}
  \item Verwendetes Betriebssystem: Linux
  \item Ben\"otigte Software:
  \begin{itemize}
    \item Eingabefenster (Shell, Terminal oder Konsole)
    \item Editor mit Syntax-Hervorhebung (z.B. Kate oder GEdit)
    \item Internet-Browser (Firefox, Iceweasel oder Konquerer)
    \item pdf-Betrachter (Adobe-Reader, Xpdf oder Okular)
  \end{itemize}
  \item Kurze Einf\"uhrung: \emph{Anleitung und Informationen zum Praktikum mit den Sprachen C++ und Java} (auf ILIAS)
\end{itemize}
\end{frame}


%%%%%%%%%%%%%%%%%%%%%%%%%%%%%%%%%%%%%%%%%%%%%%%%%%%%%%%%%%%%%%%%%%%%%%%%
\def\stitle{Grundlagen der Linux Konsole}%
\subsection{\stitle}\label{S:Anleitung}
\begin{frame}[t]%
\frametitle{\kap.\ref{S:Anleitung} \stitle\ (1)}%
\medskip

\textbf{Generelle Bemerkungen zur Konsole}
\begin{itemize}
  \item Laufende Programme bzw. Befehlsausf\"uhrungen k\"onnen durch die Tastenkombination \textbf{Strg+c} abgebrochen werden
  \item F\"ur Autovervollst\"andigung von Dateinamen und Pfaden in der Konsole zwei Mal \textbf{Tab}
  \item F\"ur Hilfestellungen und Dokumentation \textbf{man} bzw. \textbf{-{}-help} und \textbf{google}
\end{itemize}

\end{frame}


%%%%%%%%%%%%%%%%%%%%%%%%%%%%%%%%%%%%%%%%%%%%%%%%%%%%%%%%%%%%%%%%%%%%%%%%
\begin{frame}[t]%
\frametitle{\kap.\ref{S:Anleitung} \stitle\ (2)}%
\medskip

\textbf{Die wichtigsten UNIX-Kommandos zum navigieren}
\begin{itemize}
  \setlength{\itemsep}{4pt}
  \item Aktuelles Verzeichnis ausgeben: \textbf{pwd (print working directory)}
  \item Verzeichnis wechseln: \textbf{cd (change directory)}
  \begin{itemize}
    \setlength{\itemsep}{2pt}
    \item \textbf{cd /tmp} Wechsel in das Verzeichnis /tmp (absoluter Pfad)
    \item \textbf{cd work/dat} Wechsel in das Unterverzeichnis dat von work (relativer Pfad)
    \item \textbf{cd} Wechsel in Ihr Home-Verzeichnis
    \item \textbf{cd ..} Wechsel in das \"ubergeordnete Verzeichnis
  \end{itemize}
  \item Inhalt des aktuellen Verzeichnisses auflisten: \textbf{ls (list)}
  \begin{itemize}
    \setlength{\itemsep}{2pt}
    \item \textbf{ls -{}- help} anzeigen der Dokumenation, alt. \textbf{man ls}
    \item \textbf{ls -l} zeigt Inhalt als Liste an. Verzeichnisse, Archive und ausf\"uhrbare Dateien werden eingef\"arbt
    \item \textbf{ls -lh} zeigt Inhalt zus\"atzlich als human readable an
    \item \textbf{ls -lhS} zeigt Inhalt zus\"atzlich der Gr\"o\ss e nach geordnet an
    \item $\ldots$
  \end{itemize}
\end{itemize}

\end{frame}


%%%%%%%%%%%%%%%%%%%%%%%%%%%%%%%%%%%%%%%%%%%%%%%%%%%%%%%%%%%%%%%%%%%%%%%%
\begin{frame}[t]%
\frametitle{\kap.\ref{S:Anleitung} \stitle\ (3)}%
\medskip

\textbf{Die wichtigsten UNIX-Kommandos zum erstellen und l\"oschen von Verzeichnissen}
\begin{itemize}
  \setlength{\itemsep}{4pt}
  \item Neues Verzeichnis anlegen: \textbf{mkdir (make directory)}
  \begin{itemize}
    \item \textbf{mkdir neu} legt das Verzeichnis neu im aktuellen Verzeichnis an
  \end{itemize}
  \item Neue Datei erstellen: \textbf{touch}
  \begin{itemize}
    \item \textbf{touch helloworld.java} erstellt die Datei helloworld.java
  \end{itemize}
  \item L\"oschen einer Datei: \textbf{rm (remove)}
  \begin{itemize}
    \setlength{\itemsep}{2pt}
    \item \textbf{rm helloworld.java} l\"oscht die Datei helloworld.java
    \item \textbf{rm -r helloworld.java} l\"oscht rekursiv, also auch gesamte Verzeichnisse
    \item \textbf{rm -f helloworld.java} erzwingt das L\"oschen
    \item \textbf{rm -v helloworld.java} aktiviert die Erkl\"arung was der Befehl bewirkt
  \end{itemize}
  \item Abrufen der Dokumentation: \textbf{man}
  \begin{itemize}
    \item \textbf{man rm} ruft die Dokumentation des Befehls \textbf{rm} auf. Mit \textbf{q (quit)} gelangt man zur\"uck.
  \end{itemize}
\end{itemize}

\end{frame}


%%%%%%%%%%%%%%%%%%%%%%%%%%%%%%%%%%%%%%%%%%%%%%%%%%%%%%%%%%%%%%%%%%%%%%%%
\begin{frame}[t]%
\frametitle{\kap.\ref{S:Anleitung} \stitle\ (4)}%
\medskip

\textbf{Die wichtigsten UNIX-Kommandos zum kopieren}
\begin{itemize}
  \setlength{\itemsep}{4pt}
  \item Befehl \textbf{cp (copy) [OPTION] <SOURCE> <DESTINATION>}
  \begin{itemize}
    \setlength{\itemsep}{2pt}
    \item \textbf{cp work.tex final.tex} erstellt eine Kopie von work.tex die final.tex hei\ss t
    \item \textbf{cp -r folderSource folderDest} kopiert rekursiv und damit auch Verzeichnisse
    \item Optionen: \textbf{-v, --verbose; -r, --recursive; -f, --force}
  \end{itemize}
\end{itemize}
\end{frame}


%%%%%%%%%%%%%%%%%%%%%%%%%%%%%%%%%%%%%%%%%%%%%%%%%%%%%%%%%%%%%%%%%%%%%%%%
\def\stitle{Programmentwicklung}%
\subsection{\stitle}\label{S:Progentw}
\begin{frame}[t]%
\frametitle{\kap.\ref{S:Progentw} \stitle}%
\medskip

\textbf{\"Ubersetzen des Programms}
\begin{itemize}
  \item \textbf{Wichtig!:} Java Programme m\"ussen mit .java enden
  \item Java-Quelltext wird durch Aufruf des Java-Compilers in Java-Bytecode \textbf{javac <Dateiname>.java} \"ubersetzt
  \item Zur Ausf\"uhrung muss der Java-Interpreter ohne Angabe einer Extension aufgerufen werden
  \item[] \textbf{java <Dateiname>}
\end{itemize}
\medskip

\textbf{Beispiel: Kugelvolumen}
\begin{itemize}
  \item Das Programm wurde unter \textbf{KugelVolumen.java} abgespeichert
  \item Befehl \textbf{javac KugelVolumen.java} kompiliert das Programm und \textbf{java KugelVolumen} f\"uhrt es aus
  \item Auf dem Bildschirm erscheint folgende Ausgabe: \\
  \code{Bitte Kugelradius eingeben:\\
  > 1 \\
  Das Volumen betr\"agt v = 4.1887902047863905}
\end{itemize}
\end{frame}
