\def\stitle{Wiederholung Rundungsfehler}
\section{\stitle}\label{K:wdh}
\begin{frame}
  \frametitle{\kap. \stitle}%
\tableofcontents[current]
\end{frame}
%%%%%%%%%%%%%%%%%%%%%%%%%%%%%%%%%%%%%%%%%%%%%%%%%%%%%%%%%%%%%%%%%%%%%%%%
%%%%%%%%%%%%%%%%%%%%%%%%%%%%%%%%%%%%%%%%%%%%%%%%%%%%%%%%%%%%%%%%%%%%%%%%
%%%%%%%%%%%%%%%%%%%%%%%%%%%%%%%%%%%%%%%%%%%%%%%%%%%%%%%%%%%%%%%%%%%%%%%%
\subsection{\stitle}\label{S:rund}
\begin{frame}[t]%
 \frametitle{\ref{K:wdh}.\ref{S:rund} \stitle}

Der Fehler bei Subtraktion fast gleich gro\ss er Gleitkomma Zahlen wird \emph{Ausl\"oschung} genannt.
\medskip

\textbf{Wdh.: } Eine Gleitkommazahl $x$ wird wie folgt dargestellt
$$x = 0.x_1 x_2 x_3 ... x_m \cdot B^e.$$
F\"ur die einzelnen Ziffern $x_i$ gilt dabei $0 \leq x_i < B$ zu einer Basis $B$ mit Exponent $e$ und maximalen Mantissen-L\"ange $m$.
\heading{Beispiel f\"ur Ausl\"oschung}
Betrachtet man eine Gleitkommazahl im Dezimalsystem ($B=10$) f\"ur welche $6$ Nachkommastellen dargestellt werden k\"onnen ($m=6$), so gilt:\\
\begin{eqnarray*}
0.239859 \cdot 10^0 - 0.239841 \cdot 10^0 &=& 0.000018 \cdot 10^0\\
 &=& 0.180000 \cdot 10^{-4}
\end{eqnarray*}
Die letzten $4$ Nullen sind nur korrekt, falls $0.239859$ und $0.239841$ exakte Zahlen waren.
Dabei k\"onnen Abweichungen in den letzten Nachkommastellen durch vorangegangene Rundungsfehler entstanden sein.
Der Fehler ist hier ca. $40\%$!
\end{frame}
%%%%%%%%%%%%%%%%%%%%%%%%%%%%%%%%%%%%%%%%%%%%%%%%%%%%%%%%%%%%%%%%%%%%%%%%


%%%%%%%%%%%%%%%%%%%%%%%%%%%%%%%%%%%%%%%%%%%%%%%%%%%%%%%%%%%%%%%%%%%%%%%%
%%%%%%%%%%%%%%%%%%%%%%%%%%%%%%%%%%%%%%%%%%%%%%%%%%%%%%%%%%%%%%%%%%%%%%%%
\def\stitle{Rundungsfehler: Maschinengenauigkeit}
\subsection{\stitle}\label{S:Maschinengenauigkeit}
\begin{frame}[t]%
 \frametitle{\ref{K:wdh}.\ref{S:Maschinengenauigkeit} \stitle}

\heading{Kleines Beispiel Programm}
\lstinputlisting[style=JAVAlines]{\getexercisefolder/MaschinenGenauigkeit.java}

\end{frame}

%%%%%%%%%%%%%%%%%%%%%%%%%%%%%%%%%%%%%%%%%%%%%%%%%%%%%%%%%%%%%%%%%%%%%%%%
%%%%%%%%%%%%%%%%%%%%%%%%%%%%%%%%%%%%%%%%%%%%%%%%%%%%%%%%%%%%%%%%%%%%%%%%
\def\stitle{Rundungsfehler: Float}
\subsection{\stitle}\label{S:Float}
\begin{frame}[t]%
 \frametitle{\ref{K:wdh}.\ref{S:Float} \stitle}

\heading{Kleines Beispiel Programm}
\lstinputlisting[style=JAVAlines]{\getexercisefolder/FloatLimits.java}

\end{frame}
