\def\stitle{\theexercise\ - Felder}
\section{\stitle}
\begin{frame}
  \frametitle{\stitle}%
\tableofcontents[current]
\end{frame}



\begin{frame}[t]% 
    \frametitle{\stitle \: a)}

  Betrachten Sie den folgenden Ausschnitt eines Java-Programms.
  \lstinputlisting[style=JAVAlines]{\getexercisefolder/Feldera.java}
  \begin{itemize}
  \item Was wird auf der Konsole ausgegeben?
  \pause
  \item \code{[1, 3, 6, 10, 15]}
  \end{itemize}
\end{frame}


\begin{frame}[t]%
    \frametitle{\stitle \: b)}

  Betrachten Sie den folgenden Ausschnitt eines Java-Programms.
  \lstinputlisting[style=JAVAlines]{\getexercisefolder/Felderb.java}
  \begin{itemize}
  \item Was wird auf der Konsole ausgegeben?
  \pause
  \item \code{[1, 3, 5, 7, 9]}
  \end{itemize}
\end{frame}

\begin{frame}[t]%
    \frametitle{\stitle \: c)}

  Betrachten Sie den folgenden Ausschnitt eines Java-Programms.
  \lstinputlisting[style=JAVAlines]{\getexercisefolder/Felderc.java}
  \begin{itemize}
  \item Was wird auf der Konsole ausgegeben?
  \pause
  \item \code{[1, 2, 3, 2, 1]}
  \end{itemize}
\end{frame}


\begin{frame}[t]%
    \frametitle{\stitle \: d)}

  Betrachten Sie den folgenden Ausschnitt eines Java-Programms.
  \lstinputlisting[style=JAVAlines]{\getexercisefolder/Felderd.java}
  \begin{itemize}
  \item Was wird auf der Konsole ausgegeben?
  \pause
  \item \code{[0, 0, 3, 3, 3]}
  \end{itemize}
\end{frame}

\begin{frame}[t]%
    \frametitle{\stitle \: e)}

  Betrachten Sie den folgenden Ausschnitt eines Java-Programms.
  \lstinputlisting[style=JAVAlines]{\getexercisefolder/Feldere.java}
  \begin{itemize}
  \item Was wird auf der Konsole ausgegeben?
  \pause
  \item \code{[1, 4, 9, 16, 25]}
  \end{itemize}
\end{frame}

\begin{frame}[t]%
    \frametitle{\stitle \: f)}

  Betrachten Sie den folgenden Ausschnitt eines Java-Programms.
  \lstinputlisting[style=JAVAlines]{\getexercisefolder/Felderf.java}
  \begin{itemize}
  \item Was wird auf der Konsole ausgegeben?
  \pause
  \item \code{[5, 4, 3, 2, 1]}
  \end{itemize}
\end{frame}
