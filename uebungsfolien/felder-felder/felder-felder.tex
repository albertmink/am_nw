\begin{frame}[t]%
  \setcounter{exercise}{22} % TODO make exercise number dynamic
  \medskip
  \begin{exercise}{Felder (a)}
  \begin{body}
  Betrachten Sie den folgenden Ausschnitt eines Java-Programms.
  \lstinputlisting[style=JAVAlines]{felder-felder/Feldera.java}
  \begin{parts}
  \item Was wird auf der Konsole ausgegeben?
  \pause
  \item \code{[1, 3, 6, 10, 15]}
  \end{parts}
  \end{body}
  \end{exercise}
\end{frame}


\begin{frame}[t]%
  \setcounter{exercise}{22} % TODO make exercise number dynamic
  \medskip
  \begin{exercise}{Felder (b)}
  \begin{body}
  Betrachten Sie den folgenden Ausschnitt eines Java-Programms.
  \lstinputlisting[style=JAVAlines]{felder-felder/Felderb.java}
  \begin{parts}
  \item Was wird auf der Konsole ausgegeben?
  \pause
  \item \code{[1, 3, 5, 7, 9]}
  \end{parts}
  \end{body}
  \end{exercise}
\end{frame}

\begin{frame}[t]%
  \setcounter{exercise}{22} % TODO make exercise number dynamic
  \medskip
  \begin{exercise}{Felder (c)}
  \begin{body}
  Betrachten Sie den folgenden Ausschnitt eines Java-Programms.
  \lstinputlisting[style=JAVAlines]{felder-felder/Felderc.java}
  \begin{parts}
  \item Was wird auf der Konsole ausgegeben?
  \pause
  \item \code{[1, 2, 3, 2, 1]}
  \end{parts}
  \end{body}
  \end{exercise}
\end{frame}


\begin{frame}[t]%
  \setcounter{exercise}{22} % TODO make exercise number dynamic
  \medskip
  \begin{exercise}{Felder (d)}
  \begin{body}
  Betrachten Sie den folgenden Ausschnitt eines Java-Programms.
  \lstinputlisting[style=JAVAlines]{felder-felder/Felderd.java}
  \begin{parts}
  \item Was wird auf der Konsole ausgegeben?
  \pause
  \item \code{[0, 0, 3, 3, 3]}
  \end{parts}
  \end{body}
  \end{exercise}
\end{frame}

\begin{frame}[t]%
  \setcounter{exercise}{22} % TODO make exercise number dynamic
  \medskip
  \begin{exercise}{Felder (e)}
  \begin{body}
  Betrachten Sie den folgenden Ausschnitt eines Java-Programms.
  \lstinputlisting[style=JAVAlines]{felder-felder/Feldere.java}
  \begin{parts}
  \item Was wird auf der Konsole ausgegeben?
  \pause
  \item \code{[1, 4, 9, 16, 25]}
  \end{parts}
  \end{body}
  \end{exercise}
\end{frame}

\begin{frame}[t]%
\setcounter{exercise}{22} % TODO make exercise number dynamic
  \medskip
  \begin{exercise}{Felder (f)}
  \begin{body}
  Betrachten Sie den folgenden Ausschnitt eines Java-Programms.
  \lstinputlisting[style=JAVAlines]{felder-felder/Felderf.java}
  \begin{parts}
  \item Was wird auf der Konsole ausgegeben?
  \pause
  \item \code{[5, 4, 3, 2, 1]}
  \end{parts}
  \end{body}
  \end{exercise}
\end{frame}
