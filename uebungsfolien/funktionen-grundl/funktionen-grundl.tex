\def\stitle{\theexercise\ - Funktionen}
\section{\stitle}
\begin{frame}
  \frametitle{\stitle}%
\tableofcontents[current]
\end{frame}

\begin{frame}[t]%
    \frametitle{\stitle}

\begin{enumerate}
\item Erstellen Sie eine Funktion vom Type \code{void}, die \code{Hello World!} auf der Konsole ausgibt.
\item Erstellen Sie eine Funktion vom Type \code{int}, die auf eine gegebene Zahl eins addiert und das Ergebnis zur\"uck gibt.
\item Erstellen Sie eine Funktion vom Typ \code{double}, die die Summe zweier Gleitkommazahlen zur"uck gibt.
\end{enumerate}
\end{frame}

\begin{frame}[fragile]%
  \frametitle{\theexercise\ - Rechnerprogramm}%

\lstinputlisting[style=JAVAlines]{\getexercisefolder/Funktionen.java}
\end{frame}


\begin{frame}[fragile]%
  \frametitle{\theexercise\ - Bemerkung}%

\heading{Primitive Datentypen} werden als lokale Kopie an Funktionen übergeben.

\lstinputlisting[style=JAVA]{\getexercisefolder/Funktionen_bem.java}
\heading{Achtung} bei alle anderen Datentypen insbesondere Arrays.


\end{frame}
