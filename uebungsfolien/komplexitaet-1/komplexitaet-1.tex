%%%%%%%%%%%%%%%%%%%%%%%%%%%%%%%%%%%%%%%%%%%%%%%%%%%%%%%%%%%%%%%%%%%%%%%%
\def\stitle{\theexercise\ - Komplexit\"at}

\section{\stitle}
\begin{frame}
  \frametitle{\stitle}%
\tableofcontents[current]
\end{frame}


%%%%%%%%%%%%%%%%%%%%%%%%%%%%%%%%%%%%%%%%%%%%%%%%%%%%%%%%%%%%%%%%%%%%%%%%
\begin{frame}%
  \frametitle{\stitle}%

\heading{Zähle die Rechenoperationen in Abhängigkeit von $n$}

Geben Sie jeweils den \emph{asymptotischen Rechenaufwand} der Methoden in Abhängigkeit von $n$ als \emph{Landau-Symbol} an.
Mögliche Landau-Symbole sind $\mathcal{O}(1)$, $\mathcal{O}(\log n)$, $\mathcal{O}(n)$, $\mathcal{O}(n \log n)$, $\mathcal{O}(n^2)$, $\mathcal{O}(n^3)$, $\mathcal{O}(b^n)$, $\mathcal{O}(n!)$.

\lstinputlisting[style=JAVA,frame=l]{\getexercisefolder/a.java}

\pause
Einfache For-Schleife die n-mal durchlaufen wird.
Pro Schleifendurchlauf findet eine Multiplikation statt.
\begin{itemize}
    \item $\mathcal{O}(n)$
\end{itemize}
\end{frame}


%%%%%%%%%%%%%%%%%%%%%%%%%%%%%%%%%%%%%%%%%%%%%%%%%%%%%%%%%%%%%%%%%%%%%%%%
\begin{frame}%
  \frametitle{\stitle\ - Teilaufgaben}%
\lstinputlisting[style=JAVA,frame=l]{\getexercisefolder/b.java}
\pause
Pro Methodenaufruf ($n>0$) wird eine Multiplikation durchgeführt und zwei rekursive Methodenaufrufe.
\begin{itemize}
    \item $\mathcal{O}(2^n -1) = \mathcal{O}(2^n)$
\end{itemize}
\end{frame}


%%%%%%%%%%%%%%%%%%%%%%%%%%%%%%%%%%%%%%%%%%%%%%%%%%%%%%%%%%%%%%%%%%%%%%%%
\begin{frame}%
  \frametitle{\stitle\ - Teilaufgaben}%
\lstinputlisting[style=JAVA,frame=l]{\getexercisefolder/c.java}
\pause
Zwei geschachtelte For-Schleifen die jeweils n-mal durchlaufen werden mit einer Multiplikation.
\begin{itemize}
    \item $\mathcal{O}(n*n)$
\end{itemize}
\end{frame}


%%%%%%%%%%%%%%%%%%%%%%%%%%%%%%%%%%%%%%%%%%%%%%%%%%%%%%%%%%%%%%%%%%%%%%%%
\begin{frame}%
  \frametitle{\stitle\ - Teilaufgaben}%
\lstinputlisting[style=JAVA,frame=l]{\getexercisefolder/d.java}
\pause
Pro Methodenaufruf wird eine Multiplikation und ein weiterer Methodenaufruf mit~$n/2$ durchgeführt.
Da Integerdivision $n/2 = \lfloor n/2 \rfloor$.
\begin{itemize}
    \item $\mathcal{O}( \lfloor \ln(n) \rfloor+2)= \mathcal{O}(\ln(n))$
\end{itemize}
Wir haben die Berechnung $n/2$ nicht mitgezählt.
\end{frame}

%%%%%%%%%%%%%%%%%%%%%%%%%%%%%%%%%%%%%%%%%%%%%%%%%%%%%%%%%%%%%%%%%%%%%%%%
\begin{frame}%
  \frametitle{\stitle\ - Teilaufgaben}%
\lstinputlisting[style=JAVA,frame=l]{\getexercisefolder/e.java}
\pause
Pro Methodenaufruf mit $n$ wird die Schleife n-mal durchlaufen.
In jedem Schleifendurchlauf findet eine Multiplikation statt und ein Methodenaufruf mit~$n-1$.
\begin{itemize}
    \item $\mathcal{O}(n!)$
\end{itemize}
\end{frame}

%%%%%%%%%%%%%%%%%%%%%%%%%%%%%%%%%%%%%%%%%%%%%%%%%%%%%%%%%%%%%%%%%%%%%%%%
\begin{frame}%
  \frametitle{\stitle\ - Teilaufgaben}%
\lstinputlisting[style=JAVA,frame=l]{\getexercisefolder/f.java}
\pause
Pro Methodenaufruf findet eine Multiplikation und ein rekursiver Aufruf mit $n-1$ statt.
\begin{itemize}
    \item $\mathcal{O}(n)$
\end{itemize}
\end{frame}
