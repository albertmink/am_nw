%%%%%%%%%%%%%%%%%%%%%%%%%%%%%%%%%%%%%%%%%%%%%%%%%%%%%%%%%%%%%%%%%%%%%%%%
\section{Wiederholung}
\begin{frame}
  \frametitle{\kap. Wiederholung}%
\tableofcontents[current]
\end{frame}

\def\stitle{For Schleife}
\begin{frame}[fragile]%
  \frametitle{\kap. \stitle}%

\begin{lstlisting}[style=java]
for(Initialisierung; Abbruch; Update) {
    Anweisung_1;
    ...
    Anweisung_n;
}
\end{lstlisting}
\begin{itemize}
\item Der \textbf{\textcolor{KITblue}{Initialisierungsteil}}  wird einmal ausgef\"uhrt und dient zur
    Deklaration von (lokalen) Variablen. \\ Bsp.: \lstinline|int i=0|
\item Der \textbf{\textcolor{KITblue}{Abbruchsteil}}  ist ein Ausdruck vom Typ
    \textcolor{KITgreen}{\lstinline|bool|}. Solange sich der Wert
    \textcolor{KITgreen}{\lstinline|true|} ergibt, wird die Verbundanweisung und
    anschlie\ss{}end der Updateteil wiederholt. Die Schleife wird beendet, wenn
    der Ausdruck \textcolor{KITgreen}{\lstinline|false|} ist.
\item Der \textbf{\textcolor{KITblue}{Updateteil}}  enth\"alt in der Regel einen Inkrement Operator, welcher einen Wert f\"ur den n\"achsten Schleifendurchlauf ver\"andert. \\ Bsp.: \lstinline|i++| oder \lstinline|i+=5|
\end{itemize}
\end{frame}
