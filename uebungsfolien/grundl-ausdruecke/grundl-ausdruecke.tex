\def\stitle{\theexercise\ - Ausdrücke}
\section{\stitle}
\begin{frame}%
  \frametitle{\stitle}%
\tableofcontents[current]
\end{frame}

%%%%%%%%%%%%%%%%%%%%%%%%%%%%%%%%%%%%%%%%%%%%%%%%%%%%%%%%%%%%%%%%%%%%%%%%
\begin{frame}[t]
  \frametitle{\stitle}

Geben Sie das Ergebnis und den Datentyp der nachfolgenden Ausdrücke an
\begin{center}
\begin{tabular}{ |c|c|c| }
\hline
Ausdruck              & Ergebnis & Datentyp \\
\hline
\hline
\code{12.3 + 1}       &     &   \\
\code{0.5 * 4}        &     &   \\
\code{1 / 2}          &     &   \\
\code{(int) 1.23}     &     &   \\
\code{2 / (int) 3.14} &     &   \\
\code{2.0 / 4.0}      &     &   \\
\code{3 / 6.0}        &     &   \\
\code{1.0 / 4}        &     &   \\
\code{(2-1.0)/5}      &     &   \\
\code{14.0 \% 5}      &     &   \\
\code{15 \% 2}        &     &   \\
\code{1>2}            &     &   \\
\code{(1+1)>1}        &     &   \\
\hline
\end{tabular}
\end{center}

\end{frame}

%%%%%%%%%%%%%%%%%%%%%%%%%%%%%%%%%%%%%%%%%%%%%%%%%%%%%%%%%%%%%%%%%%%%%%%%
\begin{frame}[t]
  \frametitle{\stitle}

Geben Sie das Ergebnis und den Datentyp der nachfolgenden Ausdrücke an
\begin{center}
\begin{tabular}{ |c|c|c|c| }
\hline
Ausdruck              & Ergebnis & Datentyp       &  Kommentar\\
\hline
\hline
\code{12.3 + 1}       & 13.3     & \code{double}  &  \\
\code{0.5 * 4}        & 2.0      & \code{double}  &  \\
\code{1 / 2}          & 0        & \code{int}     &  integer Divison\\
\code{(int) 1.23}     & 1        & \code{int}     &  Typumwandlung VL 6.6\\
\code{2 / (int) 3.14} & 0        & \code{int}     &  \\
\code{2.0 / 4.0}      & 0.5      & \code{double}  &  \\
\code{3 / 6.0}        & 0.5      & \code{double}  &  \\
\code{1.0 / 4}        & 0.25     & \code{double}  &  \\
\code{(2-1.0)/5}      & 0.2      & \code{double}  &  \\
\code{14.0 \% 5}      & 4.0      & \code{double}  &  modulo Operator\\
\code{15 \% 2}        & 1        & \code{int}     &  \\
\code{1>2}            & false    & \code{boolean} &  \\
\code{(1+1)>1}        & true     & \code{boolean} &  \\
\hline
\end{tabular}
\end{center}

\end{frame}

%%%%%%%%%%%%%%%%%%%%%%%%%%%%%%%%%%%%%%%%%%%%%%%%%%%%%%%%%%%%%%%%%%%%%%%%
\begin{frame}[t]%
  \frametitle{\stitle\ Bsp Programm}

\lstinputlisting[style=JAVAsmall]
{\getexercisefolder Ausdruecke1.java}
\end{frame}


%%%%%%%%%%%%%%%%%%%%%%%%%%%%%%%%%%%%%%%%%%%%%%%%%%%%%%%%%%%%%%%%%%%%%%%%
\begin{frame}[t]%
  \frametitle{\stitle\ Bsp Programm}

\lstinputlisting[style=JAVAsmall]
{\getexercisefolder Ausdruecke2.java}
\end{frame}
