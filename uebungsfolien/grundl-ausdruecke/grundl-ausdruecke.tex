\def\stitle{\theexercise\ - Ausdrücke}
\section{\stitle}
\begin{frame}%
  \frametitle{\stitle}%
\tableofcontents[current]
\end{frame}


\begin{frame}[t]
  \frametitle{\stitle}

Nachfolgend finden Sie Beispiele von Java-Ausdrücken.
Geben Sie jeweils das Ergebnis des Ausdrucks sowie den Datentyp des Ergebnisses an.
Mögliche Datentypen sind \code{int}, \code{double}, \code{boolean}, \code{char} und \code{String}.
\begin{center}
\begin{minipage}{0.3\textwidth}
\begin{enumerate}
\item \code{12.3 + 1}
\item \code{0.5 * 4}
\item \code{2.0 / 4.0}
\item \code{(2-1.0)/5}
\item \code{14.0 \% 5}
\end{enumerate}
\end{minipage}
\begin{minipage}{0.3\textwidth}
\begin{enumerate}
\setcounter{enumi}{5}
\item \code{3 / 6.0}
\item \code{(int) 1.23}
\item \code{2 / (int) 3.14}
\item \code{1.0 / 4}
\item \code{15 \% 2}
\end{enumerate}
\end{minipage}
\begin{minipage}{0.3\textwidth}
\begin{enumerate}
\setcounter{enumi}{10}
\item \code{1>2}
\item \code{'a' + 1}
\item \code{1 / 2}
\item \code{1 == 2}
\item \code{(1+1)>1}
\end{enumerate}
\end{minipage}
\end{center}

\end{frame}

%%%%%%%%%%%%%%%%%%%%%%%%%%%%%%%%%%%%%%%%%%%%%%%%%%%%%%%%%%%%%%%%%%%%%%%%
\begin{frame}[t]%
  \frametitle{\stitle\ Bsp Programm}

\lstinputlisting[style=JAVAsmall]
{grundl-ausdruecke/Ausdruecke1.java}
\end{frame}


%%%%%%%%%%%%%%%%%%%%%%%%%%%%%%%%%%%%%%%%%%%%%%%%%%%%%%%%%%%%%%%%%%%%%%%%
\begin{frame}[t]%
  \frametitle{\stitle\ Bsp Programm}

\lstinputlisting[style=JAVAsmall]
{grundl-ausdruecke/Ausdruecke2.java}
\end{frame}