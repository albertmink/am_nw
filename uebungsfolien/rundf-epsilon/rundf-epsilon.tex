\def\stitle{\theexercise\ - Maschinengenauigkeit}
\section{\stitle}

\begin{frame}[t]%
    \frametitle{\stitle}
\medskip
%\end{frame}

Ist $fl(x)$ die Gleitkomma-Darstellung einer reellen Zahl $x$, dann ist der relative Rundungsfehler durch \[ e(x) := \frac{\lvert x - fl(x) \rvert }{\lvert x \rvert}  \] definiert.
Die \emph{Maschinengenauigkeit} $\epsilon$ ist die kleinste, positive Zahl, für die \[ e(x) \leq \epsilon \quad\text{für alle } x \in \mathbb{R} \text{ mit } x_{Min} \leq x \leq x_{Max} \] gilt.
Hierbei bezeichnen $x_{Min}$ und $x_{Max}$ die kleinste und die größte Maschinenzahl des betreffenden Datentyps.
Die Maschinengenauigkeit kann mit folgendem Algorithmus berechnet werden:
\begin{center}
\begin{minipage}{0.8\textwidth}
\begin{enumerate}
\item[(1)] Initialisiere $x$ mit $1$, und gehe zu (2).
\item[(2)] Solange $x + 1 \neq 1$ gilt, halbiere $x$. Ansonsten gehe zu (3).
\item[(3)] Die Maschinengenauigkeit $\epsilon$ ist $2x$.
\end{enumerate}
\end{minipage}
\end{center}
\bigskip
Bearbeiten Sie die folgenden Aufgaben:
\begin{itemize}
\item Schreiben Sie ein Java-Programm welches die Maschinengenauigkeit für den Datentyp \code{double} berechnet.
\item Schreiben Sie ein Java-Programm welches die Maschinengenauigkeit für den Datentyp \code{int} berechnet.
\end{itemize}

\end{frame}

\begin{frame}[fragile]%
 \frametitle{a) L\"osungsweg}%

\begin{center}
\lstinputlisting[style=JAVAlines]{\getexercisefolder/EpsilonDouble.java}
\end{center}
\end{frame}

\begin{frame}[fragile]%
 \frametitle{b) L\"osungsweg}%

\begin{center}
\lstinputlisting[style=JAVAlines]{\getexercisefolder/EpsilonInt.java}
\end{center}
\end{frame}
