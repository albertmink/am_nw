\def\stitle{Wiederholung - Klassen}
\section{\stitle}
\begin{frame}%
  \frametitle{\stitle}%
\tableofcontents[current]
\end{frame}


\def\sstitle{Definiere neue Datenstruktur}
\subsection{\sstitle}
\begin{frame}%
  \frametitle{\sstitle}%
Eine Klasse \code{Person} mit Klassenvariablen \code{name, year, gender}.
\lstinputlisting[style=JAVA]{\getexercisefolder/Person.java}
\begin{description}
\item[Motivation]
Erstelle speziellen Datentyp \code{Person} und b\"undle damit primitive Datentypen \code{String, int, char} in einer neuen Datenstruktur.
\item[Nomenklatur] Instanzen der Klasse Person hei\ss en Objekte.
\end{description}
\end{frame}


\def\sstitle{Erzeuge Objekte}
\subsection{\sstitle}
\begin{frame}%
  \frametitle{\sstitle}%
\lstinputlisting[style=JAVA]{\getexercisefolder/MinimalExample.java}
\end{frame}


\def\sstitle{Greife auf Objekte zu}
\subsection{\sstitle}
\begin{frame}%
  \frametitle{\sstitle}%
\lstinputlisting[style=JAVA]{\getexercisefolder/MinimalExampleZu.java}
\begin{description}
\item[Klassenvariablen] k\"onnen anstelle von \code{public} auch als \code{private} angelegt werden.
\end{description}
\end{frame}


\def\sstitle{Der Konstruktor}
\subsection{\sstitle}
\begin{frame}%
  \frametitle{\sstitle}%
Konstruiere Klassen nach einem vorgegeben Muster, definiert als Klassenmethode.
Der Konstruktor tr\"agt den Namen der Klasse -- hier: \code{PersonK} -- und dient zur Instanziierung von Klassen.
\lstinputlisting[style=JAVA]{\getexercisefolder/PersonK.java}
\end{frame}


\def\sstitle{Felder von Objekten}
\subsection{\sstitle}
\begin{frame}%
  \frametitle{\sstitle}%
Selbst definierte Datentypen k\"onnen wiederum als Felder bespeichert werden.
Der Zugriff auf Feldeintr\"age erfolgt analog -- beachte allerdings den Datentyp.
\lstinputlisting[style=JAVA]{\getexercisefolder/MinimalExampleFelder.java}
\end{frame}
