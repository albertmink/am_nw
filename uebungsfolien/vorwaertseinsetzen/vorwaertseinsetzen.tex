\def\stitle{\theexercise\ - Vorw\"artseinsetzen}
\section{\stitle}
\begin{frame}%
  \frametitle{\stitle}%
Sei folgendes Gleichungssystem gegeben
\[
\begin{pmatrix}
A_{11} & 0  &  0 \\
A_{21} & A_{22}  &  0 \\
A_{31} & A_{32}  &  A_{33} \\
\end{pmatrix}
\begin{pmatrix}
y_{1} \\
y_{2} \\
y_{3} \\
\end{pmatrix}
=
\begin{pmatrix}
b_{1} \\
b_{2} \\
b_{3} \\
\end{pmatrix}
,
\]
für eine gegebene untere Dreiecksmatrix $ A \in \mathbb{R}^{3\times 3} $ und Vektor $ b \in \mathbb{R}^3 $.
\medskip

Durch Vorw\"artseinsetzen k\"onnen nun der Reihe nach die Komponenten des unbekannten Vektors $y$ berechnet werden.
Durch sukzessives Einsetzen folgt
\[
y_i=\frac{1}{A_{ii}}\left(b_i-\sum_{k=1}^{i-1}A_{ik}y_k\right).
\]
\begin{enumerate}
\item Erstellen Sie eine Funktion \code{vorwaertseinsetzen}, die eine untere Dreiecksmatrix $ A $ und einen Vektor $ b $ erh\"alt und den L\"osungsvektor $ y $ zur\"uck gibt.
\end{enumerate}
\end{frame}


\begin{frame}%
  \frametitle{\stitle\ - L\"osungsvorschlag}%
\lstinputlisting[style=JAVA]{\getexercisefolder/Vorwaerts.java}
\end{frame}
