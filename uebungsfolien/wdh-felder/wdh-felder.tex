\def\stitle{Wiederholung Felder}
\section{\stitle}\label{K:wdh}
\begin{frame}
  \frametitle{\kap. \stitle}%
\tableofcontents[current]
\end{frame}
%%%%%%%%%%%%%%%%%%%%%%%%%%%%%%%%%%%%%%%%%%%%%%%%%%%%%%%%%%%%%%%%%%%%%%%%

%%%%%%%%%%%%%%%%%%%%%%%%%%%%%%%%%%%%%%%%%%%%%%%%%%%%%%%%%%%%%%%%%%%%%%%%
%%%%%%%%%%%%%%%%%%%%%%%%%%%%%%%%%%%%%%%%%%%%%%%%%%%%%%%%%%%%%%%%%%%%%%%%
\def\stitle{Erzeugen von Feldern}
\subsection{\stitle}\label{S:Erzeugen}
\begin{frame}[t]%
  \frametitle{\ref{K:wdh}.\ref{S:Erzeugen} \stitle}


Ein Feld kann mehrere Variablen vom selben Datentyp enthalten, hier \code{int}.
\lstinputlisting[style=JAVAlines]{\getexercisefolder/FelderErzeugen.java}

\end{frame}
%%%%%%%%%%%%%%%%%%%%%%%%%%%%%%%%%%%%%%%%%%%%%%%%%%%%%%%%%%%%%%%%%%%%%%%%


%%%%%%%%%%%%%%%%%%%%%%%%%%%%%%%%%%%%%%%%%%%%%%%%%%%%%%%%%%%%%%%%%%%%%%%%
%%%%%%%%%%%%%%%%%%%%%%%%%%%%%%%%%%%%%%%%%%%%%%%%%%%%%%%%%%%%%%%%%%%%%%%%
\def\stitle{Zugreifen auf Felder}
\subsection{\stitle}\label{S:Zugreifen}
\begin{frame}[t]%
  \frametitle{\ref{K:wdh}.\ref{S:Zugreifen} \stitle}

Mit dem Operator \code{[]} wird auf bestimmte Feldkomponenten zugegriffen.
\lstinputlisting[style=JAVAlines]{\getexercisefolder/FelderZugreifen.java}

\end{frame}
%%%%%%%%%%%%%%%%%%%%%%%%%%%%%%%%%%%%%%%%%%%%%%%%%%%%%%%%%%%%%%%%%%%%%%%%

%%%%%%%%%%%%%%%%%%%%%%%%%%%%%%%%%%%%%%%%%%%%%%%%%%%%%%%%%%%%%%%%%%%%%%%%
%%%%%%%%%%%%%%%%%%%%%%%%%%%%%%%%%%%%%%%%%%%%%%%%%%%%%%%%%%%%%%%%%%%%%%%%
\def\stitle{Felder Bibliothek in Java}
\subsection{\stitle}\label{S:bequem}
\begin{frame}[t]%
  \frametitle{\ref{K:wdh}.\ref{S:bequem} \stitle}

In \code{java.util.Arrays} werden n\"utzliche Funktion der Java Bibliothek bereit gestellt.
\lstinputlisting[style=JAVAlines]{\getexercisefolder/FelderAdv.java}

\end{frame}
%%%%%%%%%%%%%%%%%%%%%%%%%%%%%%%%%%%%%%%%%%%%%%%%%%%%%%%%%%%%%%%%%%%%%%%%


%%%%%%%%%%%%%%%%%%%%%%%%%%%%%%%%%%%%%%%%%%%%%%%%%%%%%%%%%%%%%%%%%%%%%%%%
%%%%%%%%%%%%%%%%%%%%%%%%%%%%%%%%%%%%%%%%%%%%%%%%%%%%%%%%%%%%%%%%%%%%%%%%
\def\stitle{Felder Beispiel Anwendung, Ger\"ust}
\subsection{\stitle}\label{S:BeispielG}
\begin{frame}[t]%
  \frametitle{\ref{K:wdh}.\ref{S:BeispielG} \stitle}


Lese Vektor ein und berechne die Norm.
\lstinputlisting[style=JAVAsmalllines]{\getexercisefolder/FelderAnwBare.java}

\end{frame}
%%%%%%%%%%%%%%%%%%%%%%%%%%%%%%%%%%%%%%%%%%%%%%%%%%%%%%%%%%%%%%%%%%%%%%%%


%%%%%%%%%%%%%%%%%%%%%%%%%%%%%%%%%%%%%%%%%%%%%%%%%%%%%%%%%%%%%%%%%%%%%%%%
%%%%%%%%%%%%%%%%%%%%%%%%%%%%%%%%%%%%%%%%%%%%%%%%%%%%%%%%%%%%%%%%%%%%%%%%
\def\stitle{Felder Beispiel Anwendung}
\subsection{\stitle}\label{S:Beispiel}
\begin{frame}[t]%
  \frametitle{\ref{K:wdh}.\ref{S:Beispiel} \stitle}

Lese Vektor ein und berechne die Norm.
\lstinputlisting[style=JAVAfootnote]{\getexercisefolder/FelderAnw.java}

\end{frame}
