\def\stitle{\theexercise\ - Klassenhierarchien Angewandt}
\section{\stitle}
\begin{frame}%
  \frametitle{\stitle}%
\tableofcontents[current]
\end{frame}

\begin{frame}%
  \frametitle{\theexercise\ - A) Klassendef.}%

Betrachten Sie die nachfolgenden Klassendefinitionen:
\lstinputlisting[style=JAVAfootnote]{\getexercisefolder/Wal.java}
\end{frame}


\begin{frame}[t]%
  \frametitle{\theexercise\ - A) Hauptprogramm}%

\lstinputlisting[style=JAVAfootnote]{\getexercisefolder/WalApp.java}

\begin{enumerate}
\item Geben Sie die jeweilige Konsolenausgabe an, mit Begründung.
\item Begründen Sie den auftretenden Fehler.
\end{enumerate}
\end{frame}




\begin{frame}%
  \frametitle{\theexercise\ - B) Klassendef.}%

Betrachten Sie die nachfolgenden Klassendefinitionen:
\lstinputlisting[style=JAVAfootnote]{\getexercisefolder/WalF.java}
\end{frame}

\begin{frame}[t]%
  \frametitle{\theexercise\ - B) Hauptprogramm}%

\lstinputlisting[style=JAVAfootnote]{\getexercisefolder/WalAppF.java}

\begin{enumerate}
\item Geben Sie die jeweilige Konsolenausgabe an, mit Begründung.
\item Begründen Sie den auftretenden Fehler.
\end{enumerate}
\end{frame}
