%%%%%%%%%%%%%%%%%%%%%%%%%%%%%%%%%%%%%%%%%%%%%%%%%%%%%%%%%%%%%%%%%%%%%%%%
\def\stitle{Wiederholung Schleifen}
\section{Wiederholung}\label{K:wdh}
\begin{frame}
  \frametitle{\ref{K:wdh} Wiederholung}%
\tableofcontents[current]
\end{frame}


%%%%%%%%%%%%%%%%%%%%%%%%%%%%%%%%%%%%%%%%%%%%%%%%%%%%%%%%%%%%%%%%%%%%%%%%
\def\stitle{Grundlagen in Java}
\subsection{\stitle}\label{S:GrundlageninJava}
\begin{frame}[fragile]%
  \frametitle{\ref{K:wdh}.\ref{S:GrundlageninJava} \stitle}%

\begin{description}[leftmargin=*,style=nextline]
\item[\textcolor{KITgreen}{\textbf{Datentypen}}]
\item[Ganzzahlige Typen]  \code{byte, short, int, long}
\item[Gleitkomma Typen]   \code{float, double}
\item[Zeichen]            \code{char, String}
\item[Boolscher Typen]    \code{boolean}
\end{description}
\medskip

\begin{description}[leftmargin=*,style=nextline]
\item[\textcolor{KITgreen}{\textbf{Variablen}}]
\item[Deklaration] \code{double x;}
\item[Definition] \code{int n = -1;}
\item[Wertzuweisung] \code{int a = n;}
\end{description}

\end{frame}


%%%%%%%%%%%%%%%%%%%%%%%%%%%%%%%%%%%%%%%%%%%%%%%%%%%%%%%%%%%%%%%%%%%%%%%%
\begin{frame}[fragile]%
  \frametitle{\ref{K:wdh}.\ref{S:GrundlageninJava} \stitle}%

\textcolor{KITgreen}{\heading{Keywords}} \code{int, for, const, double, else, final, import, class, this, short, while, ...}

\end{frame}


%%%%%%%%%%%%%%%%%%%%%%%%%%%%%%%%%%%%%%%%%%%%%%%%%%%%%%%%%%%%%%%%%%%%%%%%
\def\stitle{Rechner Demo}
\subsection{\stitle}\label{S:RechnerDemo0}
\begin{frame}[fragile]%
  \frametitle{\ref{K:wdh}.\ref{S:RechnerDemo0} \stitle}%

\lstinputlisting[style=java,frame=single]{./uebungsfolien/helloWorld/HelloWorldWDH.java}
\end{frame}


%%%%%%%%%%%%%%%%%%%%%%%%%%%%%%%%%%%%%%%%%%%%%%%%%%%%%%%%%%%%%%%%%%%%%%%%
\def\stitle{Kompilieren und Ausführen}
\subsection{\stitle}\label{S:CompilierenUexec}
\begin{frame}[fragile]%
  \frametitle{\ref{K:wdh}.\ref{S:CompilierenUexec} \stitle}%

Um das Programm Hello World auszuführen werden folgende Schritte auf dem Terminal durchgeführt.

\begin{lstlisting}[style=BASH]
$ javac HelloWorld.java
$ java HelloWorld
  Hello World!
\end{lstlisting}

\end{frame}


%%%%%%%%%%%%%%%%%%%%%%%%%%%%%%%%%%%%%%%%%%%%%%%%%%%%%%%%%%%%%%%%%%%%%%%%
\def\stitle{Rechner Demo, FOR Schleife}
\subsection{\stitle}\label{S:RechnerDemo}
\begin{frame}[fragile]%
  \frametitle{\ref{K:wdh}.\ref{S:RechnerDemo} \stitle}%


\begin{lstlisting}[title={Beispiel Programm},style=java, frame=single]
for( int i = 0; i < 3; i++ ) {
  System.out.println(i);
}
\end{lstlisting}
\bigskip

\begin{lstlisting}[title=output,style=java, frame=single]
0
1
2
\end{lstlisting}
\hfill

\begin{itemize}
\item Unendliche Schleife
\item Inkrement
\item \textbf{Verändere nicht die Zählvariable}
\end{itemize}
\end{frame}


%%%%%%%%%%%%%%%%%%%%%%%%%%%%%%%%%%%%%%%%%%%%%%%%%%%%%%%%%%%%%%%%%%%%%%%%
\def\stitle{Rechner Demo, WHILE Schleife}
\subsection{\stitle}\label{S:RechnerDemo2}
\begin{frame}[fragile]%
  \frametitle{\ref{K:wdh}.\ref{S:RechnerDemo2} \stitle}%


\begin{lstlisting}[title={Beispiel Programm},style=java, frame=single]
int i = 0;
while( i < 3 ) {
  i++;
  System.out.println(i);
}
\end{lstlisting}
\bigskip

\begin{lstlisting}[title=output,style=java, frame=single]
0
1
2
\end{lstlisting}
\hfill

\begin{itemize}
\item Unendliche Schleife
\item Inkrement
\end{itemize}
\end{frame}


%%%%%%%%%%%%%%%%%%%%%%%%%%%%%%%%%%%%%%%%%%%%%%%%%%%%%%%%%%%%%%%%%%%%%%%%
\def\stitle{Rechner Demo, Nested Loops}
\subsection{\stitle}\label{S:RechnerDemo3}
\begin{frame}[fragile]%
  \frametitle{\ref{K:wdh}.\ref{S:RechnerDemo3} \stitle}%


\begin{lstlisting}[title={Beispiel Programm},style=java, frame=single]
for( int i = 0; i < 4; i++ ) {
  for( int j = 0; j < i; j++ ) {
    System.out.print("*");
  }
  System.out.println(); // new line
}
\end{lstlisting}
\bigskip
\pause

\begin{lstlisting}[title=output,style=java, frame=single]
*
**
***
\end{lstlisting}
\end{frame}
