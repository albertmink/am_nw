\def\stitle{\theexercise\ - Matrixrechnung}

\section{\stitle}
\begin{frame}
  \frametitle{\stitle}%
\tableofcontents[current]
\end{frame}


\begin{frame}%
  \frametitle{\stitle}%
Matrizen als mehrdimensionale Felder.
Beachte: Java arbeitet nach dem \emph{row-major} Prinzip (Vergleich Vorlesung Abschnitt 14.4.)

\lstinputlisting[style=JAVAsmall]{\getexercisefolder/FelderMatrixBare.java}
\end{frame}


\begin{frame}%
  \frametitle{\theexercise\ - Aufgabenstellung}%

\begin{enumerate}
\item Implementieren Sie zunächst die Klassenmethode \code{getIdentity} die eine $3\times 3$~Einheitsmatrix zurück geben soll.
\item Implementieren Sie die Klassenmethode \code{getTrace} die zu einer gegebenen Matrix die Spur als Feld zurück gibt.
\item Berechnen Sie in der Klassenmethode \code{getMaxValue} den betragsmäßig größten Eintrag einer gegebenen Matrix.
\item Berechnen Sie in der Klassenmethode \code{getSpaltensummennorm} die Spaltensummennorm $||\cdot||_1$  $$||A||_1 := \max_{\nu=1,\ldots,n} \sum_{\mu=1}^m |a_{\mu,\nu}| $$ für eine gegebene Matrix $A\in\mathbf{R}^{m\times n}$ und $m=n=3$.
\end{enumerate}

\end{frame}


\begin{frame}%
  \frametitle{\theexercise\ - L\"osungsvorschlag}%

L\"osung durch Implementierung, \code{getIdentity, getTrace}.
\lstinputlisting[style=JAVA]{\getexercisefolder/FelderMatrixA.java}
\end{frame}


\begin{frame}%
  \frametitle{\theexercise\ - L\"osungsvorschlag}%

L\"osung durch Implementierung, \code{getMaxValue}.
\lstinputlisting[style=JAVA]{\getexercisefolder/FelderMatrixB.java}
\end{frame}


\begin{frame}%
  \frametitle{\theexercise\ - L\"osungsvorschlag}%

L\"osung durch Implementierung, \code{getSpaltensummennorm}.
\lstinputlisting[style=JAVAsmall]{\getexercisefolder/FelderMatrixC.java}
\end{frame}
