\def\stitle{\theexercise\ - Matrixrechnung}
\section{\stitle}
\begin{frame}%
  \frametitle{\stitle}%

Betrachten Sie das folgende Java-Programm in dem Matrizen als mehrdimensionale Felder implementiert sind.
Beachten Sie, dass Java nach dem \emph{row-major} Prinzip arbeitet (vergleich Vorlesung Abschnitt 14.4.)
\medskip

\lstinputlisting[style=JAVAsmall]{felder-matrix/FelderMatrixBare.java}
\end{frame}


\begin{frame}%
  \frametitle{\stitle\ - Aufgabenstellung}%
\medskip

\begin{enumerate}
\item Implementieren Sie zun\"achst die Klassenmethode \code{getIdentity} die eine $3\times 3$~Einheitsmatrix zur\"uck geben soll.
\item Implementieren Sie die Klassenmethode \code{getTrace} die zu einer gegebenen Matrix die Spur als Feld zur\"uck gibt.
\item Berechnen Sie in der Klassenmethode \code{getMaxValue} den betragsm\"a\ss ig gr\"o\ss ten Eintrag einer gegebenen Matrix.
\item Berechnen Sie in der Klassenmethode \code{getSpaltensummennorm} die Spaltensummennorm $||\cdot||_1$  $$||A||_1 := \max_{\nu=1,\ldots,n} \sum_{\mu=1}^m |a_{\mu,\nu}| $$ f\"ur eine gegebene Matrix $A\in\mathbf{R}^{m\times n}$ und $m=n=3$.
\end{enumerate}

\end{frame}


\begin{frame}%
  \frametitle{\stitle\ - L\"osungsvorschlag}%
\medskip

L\"osung durch Implementierung, \code{getIdentity, getTrace}.
\lstinputlisting[style=JAVA]{felder-matrix/FelderMatrixA.java}
\end{frame}


\begin{frame}%
  \frametitle{\stitle\ - L\"osungsvorschlag}%
\medskip

L\"osung durch Implementierung, \code{getMaxValue}.
\lstinputlisting[style=JAVA]{felder-matrix/FelderMatrixB.java}
\end{frame}


\begin{frame}%
  \frametitle{\stitle\ - L\"osungsvorschlag}%
\medskip

L\"osung durch Implementierung, \code{getSpaltensummennorm}.
\lstinputlisting[style=JAVA]{felder-matrix/FelderMatrixC.java}
\end{frame}
