\def\stitle{\theexercise\ - Schleifen 2}
\section{\stitle}
\begin{frame}
  \frametitle{\stitle}%
\tableofcontents[current]
\end{frame}

\begin{frame}[t]%
    \frametitle{\stitle}

\lstinputlisting[style=JAVAlines]{\getexercisefolder/Schleifen2.java}

\begin{itemize}
\item[(a)] Was wird auf dem Bildschirm ausgegeben?
\item[(b)] Realisieren Sie diesen Programmausschnitt mit einer \code{for}-Schleife.
\end{itemize}
\end{frame}


\begin{frame}[fragile]%
 \frametitle{\theexercise\ - a) L\"osungsweg}%
\lstinputlisting[style=JAVAlines]{\getexercisefolder/Schleifen2.java}
\end{frame}


\begin{frame}[fragile]%
 \frametitle{\theexercise\ - b) Als \code{for} Schleife}%
\lstinputlisting[style=JAVAlines]{\getexercisefolder/Schleifen2_for.java}
\end{frame}
