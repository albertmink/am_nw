\begin{frame}[t]%

\begin{exercise}{Schleifen}
\begin{body}
Gegeben sei der folgende Ausschnitt eines Java-Programms.

\lstinputlisting[style=JAVAlines]{schleifen-2/Schleifen2.java}

\begin{parts}
\item[(a)] Was wird auf dem Bildschirm ausgegeben?
\item[(b)] Realisieren Sie diesen Programmausschnitt mit einer \code{for}-Schleife.
\end{parts}
\end{body}
\end{exercise}
\end{frame}


\begin{frame}[fragile]%
 \frametitle{a) L\"osungsweg}%
\lstinputlisting[style=JAVAlines]{schleifen-2/Schleifen2.java}
\end{frame}


\begin{frame}[fragile]%
 \frametitle{b) Als \code{for} Schleife}%
\lstinputlisting[style=JAVAlines]{schleifen-2/Schleifen2_for.java}
\end{frame}
