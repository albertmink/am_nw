\def\stitle{\theexercise\ - Boolsche Algebra}
\section{\stitle}
\begin{frame}[t]
  \frametitle{\stitle}


In Java repräsentiert der Basistyp \code{bool} eine Boolsche Algebra.
Die Symbole $0, 1$ sind in Java als \code{false}, \code{true} definiert.
Daneben gibt es das logische Und, logische Oder und das Komplement \code{\&\&}, \code{||} und \code{!}.
Geben Sie das Ergebnis der nachfolgenden Java-Ausdrücke an
\medskip

\begin{minipage}{0.49\textwidth}
\begin{itemize}
\item[(a)] \code{true && false}
\item[(b)] \code{true || false}
\item[(c)] \code{(0 == 1) || (1 < 2)}
\item[(d)] \code{(0 != 1) && !(2 < 1)}
\end{itemize}
\end{minipage}
\begin{minipage}{0.49\textwidth}
\begin{itemize}
\item[(e)] \code{!!!true}
\item[(f)] \code{(5 == 1) || false}
\item[(g)] \code{(a == 0) && (a != 0)}
\item[(h)] \code{(a == 0) || !(a == 0)}
\end{itemize}
\end{minipage}
\end{frame}

\begin{frame}[t]
  \frametitle{\stitle\ -L\"osungsvorschlag}
\begin{center}
\begin{minipage}{0.49\textwidth}
\begin{itemize}
\item[(a)] \code{false}
\item[(b)] \code{true}
\item[(c)] \code{true}
\item[(d)] \code{true}
\end{itemize}
\end{minipage}
\begin{minipage}{0.49\textwidth}
\begin{itemize}
\item[(e)] \code{false}
\item[(f)] \code{false}
\item[(g)] \code{false}
\item[(h)] \code{true}
\end{itemize}
\end{minipage}
\end{center}

\end{frame}
