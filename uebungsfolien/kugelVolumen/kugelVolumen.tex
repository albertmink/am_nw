\AtBeginSection{}
%%%%%%%%%%%%%%%%%%%%%%%%%%%%%%%%%%%%%%%%%%%%%%%%%%%%%%%%%%%%%%%%%%%%%%%%
\section{Beispiel Kugelvolumen}
\begin{frame}
  \frametitle{\kap. Beispiel Kugelvolumen}%
\tableofcontents[current]
\end{frame}


%%%%%%%%%%%%%%%%%%%%%%%%%%%%%%%%%%%%%%%%%%%%%%%%%%%%%%%%%%%%%%%%%%%%%%%%
\def\stitle{Definition Kugelvolumen}%
\subsection{\stitle}\label{S:BeispielKugelvolumen}
\begin{frame}[t]%
  \frametitle{\kap.\ref{S:BeispielKugelvolumen} \stitle}%
\medskip

Das Kugelvolumen $V$ ist der Rauminhalt einer Kugel und abh"angig vom Kugelradius $r>0$ und ist beschrieben durch
$$ V(r) := \frac{4}{3} \pi r^3. $$
\begin{itemize}
  \item Schreiben Sie ein Java-Programm, welches das Kugelvolumen einer beliebigen Kugel berechnet.
\end{itemize}
\medskip

Vorgehen:
\begin{itemize}
\item Lese Variable $r$ ein
\item W\"ahle geeigneten Datentyp f"ur Volumen $V$
\item Lade Wert von $\pi$ aus Bibliothek
\item Berechne Volumen
\item Gebe berechneten Wert auf Konsole aus
\end{itemize}
\end{frame}


%%%%%%%%%%%%%%%%%%%%%%%%%%%%%%%%%%%%%%%%%%%%%%%%%%%%%%%%%%%%%%%%%%%%%%%%
\def\stitle{Beispiel Programm}%
\subsection{\stitle}\label{S:BeispielProgramm}
\begin{frame}[t]%
  \frametitle{\kap.\ref{S:BeispielProgramm} \stitle}%
\heading{Lese Kugelradius $r$ ein}

\lstinputlisting[style=JAVAlines,frame=single,linerange={1-10, 13-14}]
{\getexercisefolder/KugelVolumen.java}
\end{frame}


%%%%%%%%%%%%%%%%%%%%%%%%%%%%%%%%%%%%%%%%%%%%%%%%%%%%%%%%%%%%%%%%%%%%%%%%
\def\stitle{Schritt f\"ur Schritt}%
\subsection{\stitle}\label{S:SchrittSchritt}
\begin{frame}[t]%
  \frametitle{\kap.\ref{S:SchrittSchritt} \stitle}%

\heading{Lade Wert $\pi$ aus Bilbliothek und berechne das Volumen}

\lstinputlisting[style=JAVAlines,frame=single,linerange={1-11, 13-14}]
{\getexercisefolder/KugelVolumen.java}

\end{frame}


%%%%%%%%%%%%%%%%%%%%%%%%%%%%%%%%%%%%%%%%%%%%%%%%%%%%%%%%%%%%%%%%%%%%%%%%
\def\stitle{Schritt f\"ur Schritt}%
\begin{frame}[t]%
  \frametitle{\kap.\ref{S:SchrittSchritt} \stitle}%

\heading{Gebe das Volumen auf Konsole aus}
\lstinputlisting[style=JAVA,linerange={1-14}]
{\getexercisefolder/KugelVolumen.java}

\end{frame}


%%%%%%%%%%%%%%%%%%%%%%%%%%%%%%%%%%%%%%%%%%%%%%%%%%%%%%%%%%%%%%%%%%%%%%%%
\def\stitle{Schritt f\"ur Schritt}%
\begin{frame}[fragile]%
  \frametitle{\kap.\ref{S:SchrittSchritt} \stitle}%
\medskip

\begin{lstlisting}[title={Um das Programm \code{KugelVolumen} auszuf\"uhren werden folgende Schritte auf dem Terminal durchgef\"uhrt.},style=BASH]
$ javac KugelVolumen.java
$ java KugelVolumen
Bitte Kugelradius eingeben: 1
Das Volumen betraegt v = 4.1887902047863905
\end{lstlisting}

\end{frame}
