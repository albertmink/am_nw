\def\stitle{\theexercise\ - Anweisungen 1}
\section{\stitle}
\begin{frame}
    \frametitle{\stitle}
\tableofcontents[current]    
\end{frame}


%%%%%%%%%%%%%%%%%%%%%%%%%%%%%%%%%%%%%%%%%%%%%%%%%%%%%%%%%%%%%%%%%%%%%%%%
\begin{frame}[t]%
    \frametitle{\stitle}

\begin{itemize}
\item[a)] Welcher Buchstabe wird auf dem Bildschirm ausgegeben, falls \code{int n} den Wert $1$, $2$, $3$ bzw. $4$ besitzt?
\end{itemize}
\lstinputlisting[style=JAVAsmalllines]{\getexercisefolder/Anweis1_snippet.java}
\pause

\begin{center}
\begin{minipage}{0.4\textwidth}
\begin{itemize}
\item n = 1; B
\item n = 2;
\end{itemize}
\end{minipage}
\begin{minipage}{0.5\textwidth}
\begin{itemize}
\item n = 3;
\item n = 4:
\end{itemize}
\end{minipage}
\end{center}

\end{frame}


%%%%%%%%%%%%%%%%%%%%%%%%%%%%%%%%%%%%%%%%%%%%%%%%%%%%%%%%%%%%%%%%%%%%%%%%
\begin{frame}[t]%
    \frametitle{\stitle}

\begin{itemize}
\item[b)] Welcher Buchstabe wird auf dem Bildschirm ausgegeben, falls \code{int n} den Wert $1$, $2$, $3$ bzw. $4$ besitzt und \textbf{jeder} Boolesche Ausdruck in dem Programm\-abschnitt durch seine Negation ersetzt wurde?
\end{itemize}
\lstinputlisting[style=JAVAsmalllines]{\getexercisefolder/Anweis1_snippet.java}

\end{frame}


%%%%%%%%%%%%%%%%%%%%%%%%%%%%%%%%%%%%%%%%%%%%%%%%%%%%%%%%%%%%%%%%%%%%%%%%
\begin{frame}[fragile]%
 \frametitle{c) Beispiel Programm}%
\lstinputlisting[style=javasmall, basicstyle=\footnotesize\ttfamily]{\getexercisefolder/Anweis1.java}
\end{frame}
