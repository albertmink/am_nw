\def\stitle{\theexercise\ - Literale}
\section{\stitle}
\begin{frame}%
  \frametitle{\stitle}%
\tableofcontents[current]
\end{frame}

\begin{frame}[fragile]%
  \frametitle{\stitle}%


Nachfolgend finden Sie Beispiele von Java-\emph{Literalen}, d.h. von Zeichenfolgen die in einem Java-Quelltext konstante Werte repräsentieren.
Geben Sie jeweils den Datentyp des Literals an.
Mögliche Datentypen sind \code{int}, \code{long}, \code{double}, \code{float}, \code{boolean}, \code{char} und \code{String}.

\begin{center}

\begin{minipage}{0.3\textwidth}
\begin{enumerate}
\item \code{"Hello!"}
\item \code{-156}
\item \code{1E6}
\item \code{" "}
\end{enumerate}
\end{minipage}
\hfill
\begin{minipage}{0.3\textwidth}
\begin{enumerate}
\setcounter{enumi}{4}
\item \code{0.5}
\item \code{'0'}
\item \code{1.5e-1}
\item \code{124f}
\end{enumerate}
\end{minipage}
\hfill
\begin{minipage}{0.3\textwidth}
\begin{enumerate}
\setcounter{enumi}{8}
\item \code{"13"}
\item \code{true}
\item \code{'a'}
\item \code{.2}
\end{enumerate}
\end{minipage}

\end{center}
\end{frame}


% Lösungen
\begin{frame}[fragile]%
  \frametitle{\stitle}%


Die L"osungen lauten:

\begin{center}

\begin{minipage}{0.3\textwidth}
\begin{enumerate}
\item \code{"Hello!"} $\rightarrow$ \code{String}\\
\item \code{-156}     $\rightarrow$ \code{int}\\
\item\code{1E6}       $\rightarrow$ \code{double}\\
\item \code{" "}      $\rightarrow$ \code{String}\\
\end{enumerate}
\end{minipage}
\hfill
\begin{minipage}{0.3\textwidth}
\begin{enumerate}
\setcounter{enumi}{4}
\item \code{0.5}   $\rightarrow$ \code{double}\\
\item \code{'0'}   $\rightarrow$ \code{char}\\
\item \code{1.5e-1}$\rightarrow$ \code{double}\\
\item \code{124f}  $\rightarrow$ \code{float}\\
\end{enumerate}
\end{minipage}
\hfill
\begin{minipage}{0.3\textwidth}
\begin{enumerate}
\setcounter{enumi}{8}
\item \code{"13"} $\rightarrow$ \code{String}\\
\item \code{true} $\rightarrow$ \code{bool}\\
\item \code{'a'}  $\rightarrow$ \code{char}\\
\item \code{.2}   $\rightarrow$ \code{double}\\
\end{enumerate}
\end{minipage}

\end{center}
\end{frame}


\begin{frame}[t]%
  \frametitle{\stitle\, Bsp Programm}%

\lstinputlisting[style=JAVAsmall]
{\getexercisefolder Literale.java}
\end{frame}
