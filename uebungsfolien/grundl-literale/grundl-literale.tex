\begin{frame}[fragile]%
 \frametitle{Literale}
% todo title and numbering
\medskip


Nachfolgend finden Sie Beispiele von Java-\emph{Literalen}, d.h. von Zeichenfolgen die in einem Java-Quelltext konstante Werte repr\"asentieren.
Geben Sie jeweils den Datentyp des Literals an.
M\"ogliche Datentypen sind \code{int}, \code{long}, \code{double}, \code{float}, \code{boolean}, \code{char} und \code{String}.

\begin{center}

\begin{minipage}{0.3\textwidth}
\begin{itemize}
\item[(a)] \code{"Hello!"}
\item[(b)] \code{-156}
\item[(c)] \code{1E6}
\item[(d)] \code{" "}
\item[(e)] \code{0.5}
\end{itemize}
\end{minipage}

\begin{minipage}{0.3\textwidth}
\begin{itemize}
\item[(f)] \code{'0'}
\item[(g)] \code{1.5e-1}
\item[(h)] \code{124f}
\item[(i)] \code{0341}
\item[(j)] \code{"13"}
\end{itemize}
\end{minipage}

\begin{minipage}{0.3\textwidth}
\begin{itemize}
\item[(k)] \code{true}
\item[(l)] \code{'a'}
\item[(m)] \code{0xABC}
\item[(n)] \code{.2}
\item[(o)] \code{0xCEL}
\end{itemize}
\end{minipage}

\end{center}


\end{frame}
