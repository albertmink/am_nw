\begin{frame}[t]%
\medskip

\begin{exercise}{Literale}

\begin{body}
Nachfolgend finden Sie Beispiele von Java-\emph{Literalen}, d.h. von Zeichenfolgen die in einem Java-Quelltext konstante Werte repr\"asentieren.
Geben Sie jeweils den Datentyp des Literals an.
M\"ogliche Datentypen sind \code{int}, \code{long}, \code{double}, \code{float}, \code{boolean}, \code{char} und \code{String}.
\begin{center}
\begin{minipage}{0.3\textwidth}
\begin{itemize}
\item[(a)] \code{"Hello!"}
\item[(b)] \code{-156}
\item[(c)] \code{1E6}
\item[(d)] \code{"\phantom{0}\!\!"}
\item[(e)] \code{0.5}
\end{itemize}
\end{minipage}
\begin{minipage}{0.3\textwidth}
\begin{itemize}
\item[(f)] \code{'0'}
\item[(g)] \code{1.5e-1}
\item[(h)] \code{124f}
\item[(i)] \code{0341}
\item[(j)] \code{"13"}
\end{itemize}
\end{minipage}
\begin{minipage}{0.3\textwidth}
\begin{itemize}
\item[(k)] \code{true}
\item[(l)] \code{'a'}
\item[(m)] \code{0xABC}
\item[(n)] \code{.2}
\item[(o)] \code{0xCEL}
\end{itemize}
\end{minipage}
\end{center}
\end{body}

\begin{solution}
\begin{center}
\begin{minipage}{0.3\textwidth}
\begin{itemize}
\item[(a)] \code{String} \\ (Hello!)
\item[(b)] \code{int}    \\ ($-156$)
\item[(c)] \code{double} \\ ($1000000$)
\item[(d)] \code{String} \\ (leere Zeichenkette)
\item[(e)] \code{double} \\ ($0{,}5$)
\end{itemize}
\end{minipage}
\begin{minipage}{0.3\textwidth}
\begin{itemize}
\item[(f)] \code{char}   \\ (Das Zeichen \glqq 0\grqq)
\item[(g)] \code{double} \\ ($0{,}15$)
\item[(h)] \code{float}  \\ ($124$)
\item[(i)] \code{int}    \\ ($341_8 = 1*8^0 + 4*8 + 3*8^2 = 225$)
\item[(j)] \code{String} \\ (die Zeichenkette \glqq 13\grqq)
\end{itemize}
\end{minipage}
\begin{minipage}{0.3\textwidth}
\begin{itemize}
\item[(k)] \code{boolean} \\ (\code{true})
\item[(l)] \code{char}    \\ (a)
\item[(m)] \code{int hexadecimal}     \\ ($\mathrm{ABC}_{16} = 12*16^0+11*16^1+10*16^2 = 2748$)
\item[(n)] \code{double}  \\ ($0{,}2$)
\item[(o)] \code{long hexadecimal}    \\ ($\mathrm{CE}_{16} = 14*16^0 + 12*16^1 = 206$)
\end{itemize}
\end{minipage}
\end{center}
\end{solution}

\end{exercise}
\end{frame}
