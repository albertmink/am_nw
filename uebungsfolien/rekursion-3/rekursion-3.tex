\def\stitle{\theexercise\ - Rekursion}
\section{\stitle}
\begin{frame}
  \frametitle{\stitle}%
\tableofcontents[current]
\end{frame}


\begin{frame}%
  \frametitle{\stitle}%

Die Folge der \emph{Fibonacci--Zahlen} $a_0, a_1, a_2, \dotsc$ ist folgendermaßen definiert:
\begin{equation*}
a_0 := 1,\qquad
a_1 := 1,\qquad
a_n := a_{n-1} + a_{n-2}\quad\text{f\"ur } n = 2,3,\dotsc
\end{equation*}

\begin{enumerate}
\item Implementieren Sie eine Klassenmethode die die Fibonacci--Zahlen rekursiv berechnet.
\item Geben Sie die ersten zehn Folgenglieder der Folge der Fibonacci--Zahlen an.
\item Wie oft wird die Methode durch den Aufruf \code{fibonacci(4)} aufgerufen?
\end{enumerate}
\end{frame}


\begin{frame}%
  \frametitle{\theexercise\ - L\"osungsvorschlag}%
\lstinputlisting[style=JAVA]{\getexercisefolder/RekursionFib.java}
\end{frame}

\begin{frame}%
  \frametitle{\theexercise\ - L\"osungsvorschlag}%
\begin{itemize}
\item Die ersten zehn Folgenglieder lauten 0, 1, 1, 2, 3, 5, 8, 13, 21, 34
\item Methode \code{fibonacci} wird für den Übergabeparameter $4$ 9-mal gerufen
\end{itemize}
\end{frame}