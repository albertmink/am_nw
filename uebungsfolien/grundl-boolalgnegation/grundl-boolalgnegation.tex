\def\stitle{\theexercise\ - Gleikomma-Datentypen}
\section{\stitle}
\begin{frame}[t]%
  \frametitle{\stitle}
\medskip

Unter der Negation eines Boolschen Ausdrucks $A$ versteht man den Ausdruck $\neg A$. Dieser ist genau dann wahr, wenn $A$ falsch ist. Seien $A$ und $B$ Boolsche Ausdrücke, so gelten die folgenden Negationsregeln
\begin{align*}
\neg \neg A &= A,\\
\neg (A \wedge B) &= \neg A \vee \neg B,\\
\neg (A \vee B)   &= \neg A \wedge \neg B. \\
\end{align*}
Geben Sie die \textbf{Negation} der nachfolgenden Java--Ausdrücke jeweils als Java--Ausdruck an. Hierbei seien \code{a} und \code{b} Variablen vom Typ \code{int}.\\[1em]
\begin{center}
\begin{minipage}{0.35\textwidth}
\begin{itemize}
\item[(a)] \code{a == 1}
\item[(b)] \code{(a == 1) || (b == 2)}
\item[(c)] \code{(a >= 0) \&\& (a <= 1)}
\end{itemize}
\end{minipage}
\quad
\begin{minipage}{0.6\textwidth}
\begin{itemize}
\item[(d)] \code{(a != 1) || (b == 4)}
\item[(e)] \code{(a != 0) \&\& (a != 1) \&\& (a != 2)}
\item[(f)] \code{(b < -1) || (b > 1) || (a == 0)}
\end{itemize}
\end{minipage}
\end{center}

\end{frame}
%%%%%%%%%%%%%%%%%%%%%%%%%%%%%%%%%%%%% solution %%%%%%%%%%%%%%%%%%%%%%%%%%%%%%%%%%%%%%%%%%%%%%%%%%%%%%%%%%%%%%%
% \begin{solution}
% \begin{center}
% \begin{minipage}{0.45\textwidth}
% \begin{itemize}
% \item[(a)] a!=1
% \item[(b)] \code{(a != 1) && (b != 2)}
% \item[(c)] \code{(a < 0) || (a > 1)}
% \end{itemize}
% \end{minipage}
% \begin{minipage}{0.45\textwidth}
% \begin{itemize}
% \item[(d)] \code{(a == 1) && (b != 4)}
% \item[(e)] \code{(a == 0)||(a == 1)||(a == 2)}
% \item[(f)] \code{(b >= -1) && (b <= 1) && (a != 0)}
% \end{itemize}
% \end{minipage}
% \end{center}
% \end{solution}
