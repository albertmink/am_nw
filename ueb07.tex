\documentclass[c,18pt]{beamer}
\listfiles

\mode<presentation>
{
  \usetheme[deutsch,titlepage0]{KIT}
  \setbeamercovered{transparent}
  \setbeamertemplate{enumerate items}[ball]
}
\usepackage[utf8]{inputenc}
\date{04.12.2018}

\newlength{\Ku}
\setlength{\Ku}{1.43375pt}

\usepackage{templateSlide/exercises}
\usepackage[java]{code}

\makeatletter
\def\input@path{{uebungsfolien/}}
\graphicspath{{uebungsfolien/}}
\makeatother

%% title slide
\title[Übung 07: Einstieg in die Informatik und algorithmische]
  {Übung 07: Einstieg in die Informatik und algorithmische \\ Mathematik}
\subtitle{Albert Mink | Wintersemester 2018/2019}

\author[Albert Mink, ]{KIT}

\AuthorTitleSep{\relax}

\institute[Institut für Angewandte und Numerische Mathematik (IANM)]{Institut für Angewandte und Numerische Mathematik}

\TitleImage[width=\titleimagewd]{logos/KIT-Titel}
\logo{\includegraphics{logos/lbrg_logo}}

%%%%%%%%%%%%%%%%%%%%%%%%%%%%%%%%%%%%%%%%%%%%%%%%%%%%%%%%%%%%%%%%%%%%%%%%
% Document
%%%%%%%%%%%%%%%%%%%%%%%%%%%%%%%%%%%%%%%%%%%%%%%%%%%%%%%%%%%%%%%%%%%%%%%%
\begin{document}
\begin{frame}
  \maketitle
\end{frame}
%%%%%%%%%%%%%%%%%%%%%%%%%%%%%%%%%%%%%%%%%%%%%%%%%%%%%%%%%%%%%%%%%%%%%%%%

\begin{frame}
  \frametitle{Arbeitsblatt 7}%
\tableofcontents
\end{frame}

\def\kap{0}
%%%%%%%%%%%%%%%%%%%%%%%%%%%%%%%%%%%%%%%%%%%%%%%%%%%%%%%%%%%%%%%%%%%%%%%%
%%%%%%%%%%%%%%%%%%%%%%%%%%%%%%%%%%%%%%%%%%%%%%%%%%%%%%%%%%%%%%%%%%%%%%%%
\section{Ank\"undigung}\label{K:ank}
\begin{frame}
  \frametitle{\ref{K:ank} Ank\"undigung}%

\begin{itemize}
\item Am Dienstag den 05.12.2017 findet die Evaluation der \"Ubung statt.
\item Im Januar 2018 findet eine 90 min\"utige Probeklausur statt, in der Sie die M\"oglichkeit haben unter realen Bedingungen eine Altklausur zu l\"osen.
\item In der darauf folgenden Woche wird die L\"osung besprochen.
\end{itemize}

\end{frame}


\def\kap{1}
%%%%%%%%%%%%%%%%%%%%%%%%%%%%%%%%%%%%%%%%%%%%%%%%%%%%%%%%%%%%%%%%%%%%%%%%
%%%%%%%%%%%%%%%%%%%%%%%%%%%%%%%%%%%%%%%%%%%%%%%%%%%%%%%%%%%%%%%%%%%%%%%%
\section{Wiederholung}\label{K:wdh}
\begin{frame}
  \frametitle{\ref{K:wdh} Wiederholung}%
\tableofcontents[current]
\end{frame}
%%%%%%%%%%%%%%%%%%%%%%%%%%%%%%%%%%%%%%%%%%%%%%%%%%%%%%%%%%%%%%%%%%%%%%%%

%%%%%%%%%%%%%%%%%%%%%%%%%%%%%%%%%%%%%%%%%%%%%%%%%%%%%%%%%%%%%%%%%%%%%%%%
%%%%%%%%%%%%%%%%%%%%%%%%%%%%%%%%%%%%%%%%%%%%%%%%%%%%%%%%%%%%%%%%%%%%%%%%
\def\stitle{Erzeugen von mehrdimensionalen Feldern}
\subsection{\stitle}\label{S:Erzeugen}
\begin{frame}[t]%
  \frametitle{\ref{K:wdh}.\ref{S:Erzeugen} \stitle}
\medskip

Ein Feld kann mehrere Variablen vom selben Datentyp enthalten, hier \code{int} und \code{char}.
\lstinputlisting[style=JAVA]{wdh-felderMehrdim/FelderErzeugen.java}

\end{frame}
%%%%%%%%%%%%%%%%%%%%%%%%%%%%%%%%%%%%%%%%%%%%%%%%%%%%%%%%%%%%%%%%%%%%%%%%


%%%%%%%%%%%%%%%%%%%%%%%%%%%%%%%%%%%%%%%%%%%%%%%%%%%%%%%%%%%%%%%%%%%%%%%%
%%%%%%%%%%%%%%%%%%%%%%%%%%%%%%%%%%%%%%%%%%%%%%%%%%%%%%%%%%%%%%%%%%%%%%%%
\def\stitle{Zugreifen auf mehrdimensionale Felder}
\subsection{\stitle}\label{S:Zugreifen}
\begin{frame}[t]%
  \frametitle{\ref{K:wdh}.\ref{S:Zugreifen} \stitle}
\medskip

Mit dem Operator \code{[]} wird auf bestimmte Feldkomponenten zu gegriffen.
\lstinputlisting[style=JAVA]{wdh-felderMehrdim/FelderZugreifen.java}

\end{frame}
%%%%%%%%%%%%%%%%%%%%%%%%%%%%%%%%%%%%%%%%%%%%%%%%%%%%%%%%%%%%%%%%%%%%%%%%

\setcounter{exercise}{27}
\def\stitle{\theexercise\ - Matrixrechnung}
\section{\stitle}
\begin{frame}%
  \frametitle{\stitle}%

Betrachten Sie das folgende Java-Programm in dem Matrizen als mehrdimensionale Felder implementiert sind.
Beachten Sie, dass Java nach dem \emph{row-major} Prinzip arbeitet (vergleich Vorlesung Abschnitt 14.4.)
\medskip

\lstinputlisting[style=JAVAsmall]{\getexercisefolder/FelderMatrixBare.java}
\end{frame}


\begin{frame}%
  \frametitle{\stitle\ - Aufgabenstellung}%
\medskip

\begin{enumerate}
\item Implementieren Sie zun\"achst die Klassenmethode \code{getIdentity} die eine $3\times 3$~Einheitsmatrix zur\"uck geben soll.
\item Implementieren Sie die Klassenmethode \code{getTrace} die zu einer gegebenen Matrix die Spur als Feld zur\"uck gibt.
\item Berechnen Sie in der Klassenmethode \code{getMaxValue} den betragsm\"a\ss ig gr\"o\ss ten Eintrag einer gegebenen Matrix.
\item Berechnen Sie in der Klassenmethode \code{getSpaltensummennorm} die Spaltensummennorm $||\cdot||_1$  $$||A||_1 := \max_{\nu=1,\ldots,n} \sum_{\mu=1}^m |a_{\mu,\nu}| $$ f\"ur eine gegebene Matrix $A\in\mathbf{R}^{m\times n}$ und $m=n=3$.
\end{enumerate}

\end{frame}


\begin{frame}%
  \frametitle{\stitle\ - L\"osungsvorschlag}%
\medskip

L\"osung durch Implementierung, \code{getIdentity, getTrace}.
\lstinputlisting[style=JAVA]{\getexercisefolder/FelderMatrixA.java}
\end{frame}


\begin{frame}%
  \frametitle{\stitle\ - L\"osungsvorschlag}%
\medskip

L\"osung durch Implementierung, \code{getMaxValue}.
\lstinputlisting[style=JAVA]{\getexercisefolder/FelderMatrixB.java}
\end{frame}


\begin{frame}%
  \frametitle{\stitle\ - L\"osungsvorschlag}%
\medskip

L\"osung durch Implementierung, \code{getSpaltensummennorm}.
\lstinputlisting[style=JAVA]{\getexercisefolder/FelderMatrixC.java}
\end{frame}

\setcounter{exercise}{28}
\begin{exercise}{Rekursion}
\begin{body}
Nachfolgend ist die Definition einer Klassenmethode gegeben, die zu einer Zahl $n \in \mathbb{N}$ die Fakultät $n! := n (n-1) (n-2)  \dotsm 1$ rekursiv berechnet.
\medskip
\begin{displaycode}
    static int fakultaet(int n) {
        if (n <= 1) {
            return 1;
        } else {
            return n * (fakultaet(n-1));
        }
    }
\end{displaycode}
\medskip
\noindent
\begin{parts}
\item
Wie oft wird die Methode durch den Aufruf \code|fakultaet(4);| aufgerufen?

\item
Geben Sie eine äquivalente Definition der Methode ohne rekursiven Aufruf an.
\end{parts}
\end{body}

\begin{solution}
\begin{parts}
\item
Die Methode wird viermal aufgerufen.

\item
Eine mögliche Klassendefinition ohne rekursiven Aufruf lautet
\medskip
\begin{displaycode}
    static int fakultaet(int n) {   
        if (n <= 1) {
            return 1;
        } else {
            int fak = 1;
            for (int i = 2; i <= n; i++) {
                fak *= i;
            }
            return fak;
        }
    }
\end{displaycode}
\end{parts}
\end{solution}
\end{exercise}

\setcounter{exercise}{29}
\begin{exercise}{Effizienz}

\begin{body}
\begin{parts}
  \item[(a)] Gegeben ist folgender Programmausschnitt.
  \begin{parts}
  \item[1)] Beschreiben Sie das Ergebnis der Rechnung.
  \item[2)] Z\"ahlen Sie die zur Berechnung n\"otige Anzahl an Rechenoperationen.
            Im weiteren sei \code{n} eine beliebige nat\"urliche Zahl.
            Charakterisieren Sie den Rechenaufwand des Ausschnittes mit Hilfe eines Ausdrucks in der Landau-Notation.

  \end{parts}
\end{parts}

\inputcode[frame=lines,title=Effizienz.java,numbers=left]{\filename{Effizienz.java}}

\begin{parts}
  \item [(b)] Vereinfachen Sie die folgenden Ausdr\"ucke zur Komplexit\"at durch geeignetes Runden zu jeweils einen Ausdruck in der Landau-Notation, z.B. $\mathcal{O}(n^3+n) = \mathcal{O}(n^3)$:
  \begin{parts}
    \item [1)] $\mathcal{O}(10n^3/(2n^2)+n^2)$
    \item [2)] $\mathcal{O}(n(n^3+n)-n^2)$
    \item [3)] $\mathcal{O}(e^{1000}+ln (n))$
    \item [4)] $\mathcal{O}(-1+e+(ln(n))*n +1/n )$
  \end{parts}
\end{parts}

\end{body}
\begin{solution}
\begin{small}
\begin{itemize}
  \item [(a)]
  \begin{itemize}
    \item[(1)]
    \begin{itemize}
      \item Mit Hilfe von einer \code{for}-Schleife werden die Elemente des neuen Vektors \code{res} berechnet.
      \item Dabei wird mit der Hilfsvariablen \code{skp} und einer weiteren \code{for}-Schleife das Skalarprodukt der \code{mu}-ten Spalte der Matrix \code{a} und des Vektors \code{b} berechnet.
      \item Zum Schluss wird noch das jeweilige Element des Vektors \code{c} addiert.
      \item Das enstpricht der Berechnung \code{d = a * b + c},
    \end{itemize}
    \item[(2)]
    \begin{itemize}
      \item Die \"au\ss ere Schleife wird \code{n}-mal durchgelaufen, das entspicht $n$-mal die Anzahl der Operationen in der inneren Schleife.
      \item Die innere Schleife wird auch \code{n}-mal durchgelaufen.
            Dort wird eine Addition und eine  Multiplikation durchgef\"uhrt.
            Dazu kommt noch eine Addition.
            Das ergibt $2n+1$ Operationen.
      \item Insgesamt sind  das wegen der Schachtelung der Schleifen $n(2n+1)=2n^2+n$ Operationen.
            Das Entspricht einer Komplexit\"at von $\mathcal{O}(n^2)$.
    \end{itemize}
  \end{itemize}
  \item[(b)]
  \begin{itemize}
    \item [1)] $\mathcal{O}(10n^3/(2n^2)+n^2) = \mathcal{O}(5n + n^2) = \mathcal{O}(n^2)$
    \item [2)] $\mathcal{O}(n(n^3+n)-n^2) = \mathcal{O}(n^4 + n^2 - n^2) = \mathcal{O}(n^4)$
    \item [3)] $\mathcal{O}(e^{1000}+ln (n)) = \mathcal{O}(ln (n))$
    \item [4)] $\mathcal{O}(-1+e+(ln(n))*n +1/n) = \mathcal{O}((ln(n)) * n + 1/n) = \mathcal{O}((ln(n)) * n)$
  \end{itemize}
\end{itemize}
\end{small}
\end{solution}
\end{exercise}

\setcounter{exercise}{30}

\begin{exercise}{Komplexität}
\begin{body}
Betrachten Sie die nachfolgenden Methodendefinitionen. Jede Methode besitzt einen formalen Parameter vom Typ \code|int|, über den eine natürliche Zahl $n \in \mathbb{N}$ an die Methode übergeben werden kann. Geben Sie jeweils den \emph{asymptotischen Rechenaufwand} (hier: Anzahl der Multiplikationen) der Methoden in Abhängigkeit von $n$ als \emph{Landau-Symbol} an. M\"ogliche Landau-Symbole sind $\mathcal{O}(1)$, $\mathcal{O}(\log n)$, $\mathcal{O}(n)$, $\mathcal{O}(n \log n)$, $\mathcal{O}(n^2)$, $\mathcal{O}(n^3)$, $\mathcal{O}(b^n)$, $\mathcal{O}(n!)$.
\begin{parts}
\item
\begin{displaycode}
    static int methode(int n) {
        int s = 1;
        for (int i=1; i <= n; i++) {
            s *= i;
        }
        return s;
    }
\end{displaycode}

\item
\begin{displaycode}
    static int methode(int n) {
        if ( n <= 0 ) {
            return 1;
        } else {
            return methode(n-1) * methode(n-1);
        }
    }
\end{displaycode}

\item
\begin{displaycode}
    static int methode(int n) {
        int s = 1;
        for (int i = 0; i < n; i++) {
            for (int j = 1; j <= n; j++) {
            	s *= j;
            }
        }
        return s;
    }
\end{displaycode}

\item
\begin{displaycode}
    static int methode(int n) {
        if ( n <= 0 ) {
            return 1;
        } else {
            return n * methode(n/2);
        }
    }
\end{displaycode}


\item
\begin{displaycode}
    static int methode(int n) {
        if (n <= 0) {
            return 1;
        } else {
            int s = 1;
            for (int i = 0; i < n; i++) {
            	s *= methode(n-1);
            }
            return s;
        }
    }
\end{displaycode}


\item
\begin{displaycode}
    static int methode(int n) {
        if ( n <= 0 ) {
            return 1;
        } else {
            return n * methode(n-1);
        } 
    }
\end{displaycode}
\end{parts}
\end{body}


\begin{solution}
\begin{parts}
\item
$O(n)$ (lineare Komplexität). Die Multiplikationen werden in einer Schleife ausgeführt, die $n$-mal durchlaufen wird. Pro Schleifendurchlauf wird genau eine Multiplikation ausgeführt. Also ergeben sich für $n$ Schleifendurchläufe genau $n$ Multiplikationen. 

\item
$O(2^n)$ (exponentielle Komplexität). Die Multiplikationen werden in einer rekursiv definierten Methode ausgeführt. Ein Methodenaufruf für einen Parameter $n > 0$ führt zu einer Multiplikation sowie zu zwei Methodenaufrufen für den Parameter $n-1$.
Für $n \leq 0$ wird keine Multiplikation ausgeführt und keine weitere Methode aufgerufen. Daher ergeben sich insgesamt $2^{n}$ Methodenaufrufe mit jeweils einer Multiplikation.


\item
$O(n^2)$ (quadratische Komplexität). Die Multiplikationen werden in einer geschachtelten Schleife ausgeführt.
Die äußere Schleife wird $n$-mal durchlaufen. Bei jedem Durchlauf der äußeren Schleife wird die innere Schleife $n$-mal Durchlaufen. Bei jedem Durchlauf der inneren Schleife wird genau eine Multiplikation ausgeführt. Also ergeben sich insgesamt $n\cdot n = n^2$ Multiplikationen.

\item
$O(\log n)$ (logarithmische Komplexität). Die Multiplikationen werden in einer rekursiv definierten Methode ausgeführt. Ein Methodenaufruf für einen Parameter $n > 0$ führt zu einer Multiplikation sowie zu einem Methodenaufruf für den Parameter $\lfloor n/2\rfloor$. Für $n \leq 0$ wird keine Multiplikation ausgeführt und keine weitere Methode aufgerufen. Daher ergeben sich insgesamt $\lfloor\log_2(n)\rfloor + 1$ Methodenaufrufe mit jeweils einer Multiplikation. 

\item
$O(n!)$ (faktorielle Komplexität). Die Multiplikationen werden in einer Schleife innerhalb einer rekursiv definierten Methode ausgeführt. Ein Methodenaufruf für einen Parameter $n > 0$ führt zu $n$ Schleifendurchläufen mit jeweils einer Multiplikation sowie zu $n$ Methodenaufrufen für den Parameter $n-1$. Für $n \leq 0$ wird keine Multiplikation ausgeführt und keine weitere Methode aufgerufen. Daher ergeben sich insgesamt $n \cdot (n-1) \dotsm 1 = n!$ Multiplikationen.

\item
$O(n)$ (lineare Komplexität). Die Multiplikationen werden in einer rekursiv definierten Methode ausgeführt. Ein Methodenaufruf mit für einen Parameter $n > 0$ führt zu einer Multiplikation sowie zu einem Methodenaufruf für den Parameter $n-1$. Für $n = 0$ wird keine Multiplikation ausgeführt und keine weitere Methode aufgerufen. Daher ergeben sich insgesamt $n$ Methodenaufrufe mit jeweils einer Multiplikation.
\end{parts}
\end{solution}
\end{exercise}


\begin{frame}
\centering
\Huge\textcolor{KITgreen}{Fragen?}
\vspace{2cm}

{\LARGE
N\"achste \"Ubung: 11. Dezember\\
Besprechung Arbeitsblatt 8
}
\end{frame}


%%%%%%%%%%%%%%%%%%%%%%%%%%%%%%%%%%%%%%%%%%%%%%%%%%%%%%%%%%%%%%%%%%%%%%%%
\end{document}
