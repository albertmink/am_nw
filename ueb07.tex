\documentclass[9pt,german]{beamer}%
\usepackage{master/templates/beamerthemeKIT}
\usepackage{master/templates/exercises}

%\documentclass[9pt,trans]{beamer}%

%%%%%%%%%%%%%%%%%%%%%%%%%%%%%%%%%%%%%%%%%%%%%%%%%%%%%%%%%%%%%%%%%%%%%%%%
% Packages
%%%%%%%%%%%%%%%%%%%%%%%%%%%%%%%%%%%%%%%%%%%%%%%%%%%%%%%%%%%%%%%%%%%%%%%%
%\usepackage{beamerthemeKIT}
\usepackage[utf8]{inputenc}
% \usepackage[german]{babel}
\usepackage{helvet}
\usepackage[T1]{fontenc}
\usepackage{amsmath, amsthm, amssymb}
\usepackage{graphicx}
\usepackage{listings}
\usepackage{hyperref}
\usepackage{KITcolors}
%%%%%%%%%%%%%%%%%%%%%%%%%%%%%%%%%%%%%%%%%%%%%%%%%%%%%%%%%%%%%%%%%%%%%%%%
% Layout
%%%%%%%%%%%%%%%%%%%%%%%%%%%%%%%%%%%%%%%%%%%%%%%%%%%%%%%%%%%%%%%%%%%%%%%%
\renewcommand{\rmdefault}{phv}
\newlength{\movetitlegrafics}
\setlength{\movetitlegrafics}{0.0\paperwidth}
%%%%%%%%%%%%%%%%%%%%%%%%%%%%%%%%%%%%%%%%%%%%%%%%%%%%%%%%%%%%%%%%%%%%%%%%
% Definitions
%%%%%%%%%%%%%%%%%%%%%%%%%%%%%%%%%%%%%%%%%%%%%%%%%%%%%%%%%%%%%%%%%%%%%%%%
%%% Counter
\newcounter{ctlit}%
\setcounter{ctlit}{1}%
\newcounter{kap}

%%% Textfarben
\newcommand\BLUE[1]{\textcolor{KITblue}{#1}}%
\newcommand\BLACK[1]{\textcolor{KITblack}{#1}}%
\newcommand\GREEN[1]{\textcolor{KITgreen}{#1}}%
\newcommand\LBLUE[1]{\textcolor{KITblue15}{#1}}%
\newcommand\RED[1]{\textcolor{KITred}{#1}}%
%%% Layout
\def\cmt#1{\BLUE{#1}}%
\def\code#1{\texttt{#1}}%
\def\codeblock#1{\GREEN{\texttt{#1}}}%
\def\ftext#1{\textbf{{#1}}}%
\def\headline#1{\large\textbf{\GREEN{#1}}}%
\def\KITitem{\KITBullet[2mm]\ }%
\def\q{\quad}%
\def\qq{\qquad}%
\def\subtitle#1{{\small #1}}%
\def\tab{\hspace*{0.5cm}}%
%%% Standards
%\input{D:/TeX/formate/mystyle/defs}%
%\input{/home/ae01/tex/formate/mystyle/defs}%
%%% Text
%%% Symbols
\def\bs{$\backslash$}%
\def\hpm{\hphantom{$-$}}%
\def\meps{\epsilon}%
%%% code environment
\lstset{
  showstringspaces=false,
  numbers=none,
  keywordstyle=\color{KITgreen},  % coloring and formatting of keywords as public, class, import 
  commentstyle=\color{KITblue}\small\ttfamily, % Color of comments
  stringstyle=\color{KITred},
  breaklines=true
}

\lstdefinestyle{JAVA}
{
  language=Java,
  basicstyle=\ttfamily, % defines text formatting
  frame=tb  % print top and bottom lines, frame=single, L
}

\lstdefinestyle{JAVAsmall}
{
  language=Java,
  basicstyle=\small\ttfamily, % defines text formatting
  frame=tb  % print top and bottom lines, frame=single, L
}

\lstdefinestyle{JAVAlines}
{
  language=Java,
  numbers=left,
  numberstyle=\color{KITblack50}\ttfamily,
  numbersep=4pt, % distance between numbers and code1_1
  xleftmargin=4pt, % distance from frame to listing
  xrightmargin=4pt, % distance from frame to listing
  basicstyle=\ttfamily, % defines text formatting
  keywordstyle=\color{KITgreen},  % coloring and formatting of keywords as public, class, import 
  commentstyle=\color{KITblue}\ttfamily, % Color of comments
  frame=tb  % print top and bottom lines, frame=single, L
}

\lstdefinestyle{JAVAsmalllines}
{
  language=Java,
  numbers=left,
  numberstyle=\color{KITblack50}\ttfamily,
  numbersep=4pt, % distance between numbers and code1_1
  xleftmargin=4pt, % distance from frame to listing
  xrightmargin=4pt, % distance from frame to listing
  basicstyle=\small\ttfamily, % defines text formatting
  keywordstyle=\color{KITgreen},  % coloring and formatting of keywords as public, class, import 
  commentstyle=\color{KITblue}\ttfamily, % Color of comments
  frame=tb  % print top and bottom lines, frame=single, L
}


\lstdefinestyle{BASH}{
  basicstyle=\Large\ttfamily, % defines text formatting
  frame=none,
  language=bash
}


%%%%%%%%%%%%%%%%%%%%%%%%%%%%%%%%%%%%%%%%%%%%%%%%%%%%%%%%%%%%%%%%%%%%%%%%
% Titlepage
%%%%%%%%%%%%%%%%%%%%%%%%%%%%%%%%%%%%%%%%%%%%%%%%%%%%%%%%%%%%%%%%%%%%%%%%
\title[Institut f\"ur Angewandte und Numerische Mathematik]%
 {\fontsize{15}{15}\selectfont{}
  \"Ubung:
  \textit{Einstieg in die Informatik}\\[1.5mm]
  \textit{\phantom{\"Ubung:}\; und algorithmische Mathematik}\\[1.5mm]
  }%\hspace*{3cm}\normalsize{f\"ur Mathematiker}}
\author{\fontsize{9}{9}\selectfont{}
 Albert Mink\
  }
\institute[Institut f\"ur Angewandte und Numerische Mathematik]
 {\fontsize{6}{6}\selectfont{}%
  Institut f\"ur Angewandte und Numerische Mathematik}
\date{Wintersemester 2018/19}%
\subject{}%
%\beamerdefaultoverlayspecification{<+->}
%%%%%%%%%%%%%%%%%%%%%%%%%%%%%%%%%%%%%%%%%%%%%%%%%%%%%%%%%%%%%%%%%%%%%%%%



\makeatletter
\def\input@path{{uebungsfolien/}}
\graphicspath{{uebungsfolien/}}
\makeatother


%%%%%%%%%%%%%%%%%%%%%%%%%%%%%%%%%%%%%%%%%%%%%%%%%%%%%%%%%%%%%%%%%%%%%%%%
% Document
%%%%%%%%%%%%%%%%%%%%%%%%%%%%%%%%%%%%%%%%%%%%%%%%%%%%%%%%%%%%%%%%%%%%%%%%
\begin{document}
\maketitle%
\addtocounter{framenumber}{-1}%
%%%%%%%%%%%%%%%%%%%%%%%%%%%%%%%%%%%%%%%%%%%%%%%%%%%%%%%%%%%%%%%%%%%%%%%%

\begin{frame}
  \frametitle{Arbeitsblatt 7}%
\tableofcontents
\end{frame}

\def\kap{0}
%%%%%%%%%%%%%%%%%%%%%%%%%%%%%%%%%%%%%%%%%%%%%%%%%%%%%%%%%%%%%%%%%%%%%%%%
%%%%%%%%%%%%%%%%%%%%%%%%%%%%%%%%%%%%%%%%%%%%%%%%%%%%%%%%%%%%%%%%%%%%%%%%
\section{Ank\"undigung}\label{K:ank}
\begin{frame}
  \frametitle{\ref{K:ank} Ank\"undigung}%

\begin{itemize}
\item Am Dienstag den 05.12.2017 findet die Evaluation der \"Ubung statt.
\item Im Januar 2018 findet eine 90 min\"utige Probeklausur statt, in der Sie die M\"oglichkeit haben unter realen Bedingungen eine Altklausur zu l\"osen.
\item In der darauf folgenden Woche wird die L\"osung besprochen.
\end{itemize}

\end{frame}


\def\kap{1}
%%%%%%%%%%%%%%%%%%%%%%%%%%%%%%%%%%%%%%%%%%%%%%%%%%%%%%%%%%%%%%%%%%%%%%%%
%%%%%%%%%%%%%%%%%%%%%%%%%%%%%%%%%%%%%%%%%%%%%%%%%%%%%%%%%%%%%%%%%%%%%%%%
\section{Wiederholung}\label{K:wdh}
\begin{frame}
  \frametitle{\ref{K:wdh} Wiederholung}%
\tableofcontents[current]
\end{frame}
%%%%%%%%%%%%%%%%%%%%%%%%%%%%%%%%%%%%%%%%%%%%%%%%%%%%%%%%%%%%%%%%%%%%%%%%

%%%%%%%%%%%%%%%%%%%%%%%%%%%%%%%%%%%%%%%%%%%%%%%%%%%%%%%%%%%%%%%%%%%%%%%%
%%%%%%%%%%%%%%%%%%%%%%%%%%%%%%%%%%%%%%%%%%%%%%%%%%%%%%%%%%%%%%%%%%%%%%%%
\def\stitle{Erzeugen von mehrdimensionalen Feldern}
\subsection{\stitle}\label{S:Erzeugen}
\begin{frame}[t]%
  \frametitle{\ref{K:wdh}.\ref{S:Erzeugen} \stitle}
\medskip

Ein Feld kann mehrere Variablen vom selben Datentyp enthalten, hier \code{int} und \code{char}.
\lstinputlisting[style=JAVA]{wdh-felderMehrdim/FelderErzeugen.java}

\end{frame}
%%%%%%%%%%%%%%%%%%%%%%%%%%%%%%%%%%%%%%%%%%%%%%%%%%%%%%%%%%%%%%%%%%%%%%%%


%%%%%%%%%%%%%%%%%%%%%%%%%%%%%%%%%%%%%%%%%%%%%%%%%%%%%%%%%%%%%%%%%%%%%%%%
%%%%%%%%%%%%%%%%%%%%%%%%%%%%%%%%%%%%%%%%%%%%%%%%%%%%%%%%%%%%%%%%%%%%%%%%
\def\stitle{Zugreifen auf mehrdimensionale Felder}
\subsection{\stitle}\label{S:Zugreifen}
\begin{frame}[t]%
  \frametitle{\ref{K:wdh}.\ref{S:Zugreifen} \stitle}
\medskip

Mit dem Operator \code{[]} wird auf bestimmte Feldkomponenten zu gegriffen.
\lstinputlisting[style=JAVA]{wdh-felderMehrdim/FelderZugreifen.java}

\end{frame}
%%%%%%%%%%%%%%%%%%%%%%%%%%%%%%%%%%%%%%%%%%%%%%%%%%%%%%%%%%%%%%%%%%%%%%%%

\setcounter{exercise}{27}
\def\stitle{\theexercise\ - Matrixrechnung}
\section{\stitle}
\begin{frame}%
  \frametitle{\stitle}%

Betrachten Sie das folgende Java-Programm in dem Matrizen als mehrdimensionale Felder implementiert sind.
Beachten Sie, dass Java nach dem \emph{row-major} Prinzip arbeitet (vergleich Vorlesung Abschnitt 14.4.)
\medskip

\lstinputlisting[style=JAVAsmall]{felder-matrix/FelderMatrixBare.java}
\end{frame}


\begin{frame}%
  \frametitle{\stitle\ - Aufgabenstellung}%
\medskip

\begin{enumerate}
\item Implementieren Sie zun\"achst die Klassenmethode \code{getIdentity} die eine $3\times 3$~Einheitsmatrix zur\"uck geben soll.
\item Implementieren Sie die Klassenmethode \code{getTrace} die zu einer gegebenen Matrix die Spur als Feld zur\"uck gibt.
\item Berechnen Sie in der Klassenmethode \code{getMaxValue} den betragsm\"a\ss ig gr\"o\ss ten Eintrag einer gegebenen Matrix.
\item Berechnen Sie in der Klassenmethode \code{getSpaltensummennorm} die Spaltensummennorm $||\cdot||_1$  $$||A||_1 := \max_{\nu=1,\ldots,n} \sum_{\mu=1}^m |a_{\mu,\nu}| $$ f\"ur eine gegebene Matrix $A\in\mathbf{R}^{m\times n}$ und $m=n=3$.
\end{enumerate}

\end{frame}


\begin{frame}%
  \frametitle{\stitle\ - L\"osungsvorschlag}%
\medskip

L\"osung durch Implementierung, \code{getIdentity, getTrace}.
\lstinputlisting[style=JAVA]{felder-matrix/FelderMatrixA.java}
\end{frame}


\begin{frame}%
  \frametitle{\stitle\ - L\"osungsvorschlag}%
\medskip

L\"osung durch Implementierung, \code{getMaxValue}.
\lstinputlisting[style=JAVA]{felder-matrix/FelderMatrixB.java}
\end{frame}


\begin{frame}%
  \frametitle{\stitle\ - L\"osungsvorschlag}%
\medskip

L\"osung durch Implementierung, \code{getSpaltensummennorm}.
\lstinputlisting[style=JAVA]{felder-matrix/FelderMatrixC.java}
\end{frame}

\setcounter{exercise}{28}
\begin{exercise}{Rekursion}
\begin{body}
Nachfolgend ist die Definition einer Klassenmethode gegeben, die zu einer Zahl $n \in \mathbb{N}$ die Fakultät $n! := n (n-1) (n-2)  \dotsm 1$ rekursiv berechnet.
\medskip
\begin{displaycode}
    static int fakultaet(int n) {
        if (n <= 1) {
            return 1;
        } else {
            return n * (fakultaet(n-1));
        }
    }
\end{displaycode}
\medskip
\noindent
\begin{parts}
\item
Wie oft wird die Methode durch den Aufruf \code|fakultaet(4);| aufgerufen?

\item
Geben Sie eine äquivalente Definition der Methode ohne rekursiven Aufruf an.
\end{parts}
\end{body}

\begin{solution}
\begin{parts}
\item
Die Methode wird viermal aufgerufen.

\item
Eine mögliche Klassendefinition ohne rekursiven Aufruf lautet
\medskip
\begin{displaycode}
    static int fakultaet(int n) {   
        if (n <= 1) {
            return 1;
        } else {
            int fak = 1;
            for (int i = 2; i <= n; i++) {
                fak *= i;
            }
            return fak;
        }
    }
\end{displaycode}
\end{parts}
\end{solution}
\end{exercise}

\setcounter{exercise}{29}
%%%%%%%%%%%%%%%%%%%%%%%%%%%%%%%%%%%%%%%%%%%%%%%%%%%%%%%%%%%%%%%%%%%%%%%%
\def\stitle{\theexercise\ - Effizienz}
\def\sAtitle{\theexercise\ - Effizienz Teil A}
\def\sBtitle{\theexercise\ - Effizienz Teil B}

\section{\stitle}
\begin{frame}
  \frametitle{\stitle}%
\tableofcontents[current]
\end{frame}

\subsection{\sAtitle}
\begin{frame}%
  \frametitle{\sAtitle}%

\begin{enumerate}
\item Beschreiben Sie das Ergebnis der Rechnung.
\item Charakterisieren Sie den Rechenaufwand des Ausschnittes mit Hilfe eines Ausdrucks in der Landau-Notation.
\end{enumerate}

\lstinputlisting[style=JAVAsmall]{\getexercisefolder/Effizienz.java}
\end{frame}

%%%%%%%%%%%%%%%%%%%%%%%%%%%%%%%%%%%%%%%%%%%%%%%%%%%%%%%%%%%%%%%%%%%%%%%%
\begin{frame}%
  \frametitle{\sAtitle\ - L\"osungsvorschlag}%

\begin{enumerate}
  \item Beschreiben Sie das Ergebnis der Rechnung.
  \begin{itemize}
    \item Mit Hilfe von einer \code{for}-Schleife werden die Elemente des neuen Vektors \code{res} berechnet.
    \item Dabei wird mit der Hilfsvariablen \code{skp} und einer weiteren \code{for}-Schleife das Skalarprodukt der \code{mu}-ten Spalte der Matrix \code{a} und des Vektors \code{b} berechnet.
    \item Zum Schluss wird noch das jeweilige Element des Vektors \code{c} addiert.
    \item Das entspricht der Berechnung \code{d = a * b + c},
  \end{itemize}
  \item Charakterisieren Sie den Rechenaufwand des Ausschnittes mit Hilfe eines Ausdrucks in der Landau-Notation.
  \begin{itemize}
  \item Die \"au\ss ere Schleife wird \code{n}-mal durchgelaufen, das entspricht $n$-mal die Anzahl der Operationen in der inneren Schleife.
  \item Die innere Schleife wird auch \code{n}-mal durchgelaufen.
  Dort wird eine Addition und eine  Multiplikation durchgef\"uhrt.
  Dazu kommt noch eine Addition.
  Das ergibt $2n+1$ Operationen.
  \item Insgesamt sind  das wegen der Schachtelung der Schleifen $n(2n+1)=2n^2+n$ Operationen.
  Das Entspricht einer Komplexit\"at von $\mathcal{O}(n^2)$.
  \end{itemize}
\end{enumerate}
\end{frame}


%%%%%%%%%%%%%%%%%%%%%%%%%%%%%%%%%%%%%%%%%%%%%%%%%%%%%%%%%%%%%%%%%%%%%%%%
\subsection{\sBtitle}
\begin{frame}%
  \frametitle{\sBtitle}%

Vereinfachen Sie die folgenden Ausdr\"ucke zur Komplexit\"at in der Landau-Notation, z.B.~$\mathcal{O}(n^3+n) = \mathcal{O}(n^3)$:
\begin{enumerate}
  \item $\mathcal{O}(10n^3/(2n^2)+n^2)$
  \item $\mathcal{O}(n(n^3+n)-n^2)$
  \item $\mathcal{O}(e^{1000}+ln (n))$
  \item $\mathcal{O}(-1+e+(ln(n))*n +1/n )$
\end{enumerate}
\end{frame}

%%%%%%%%%%%%%%%%%%%%%%%%%%%%%%%%%%%%%%%%%%%%%%%%%%%%%%%%%%%%%%%%%%%%%%%%
\begin{frame}%
  \frametitle{\sBtitle\ - L\"osungsvorschlag}%

\begin{enumerate}
  \item $\mathcal{O}(10n^3/(2n^2)+n^2) = \mathcal{O}(5n + n^2) = \mathcal{O}(n^2)$
  \item $\mathcal{O}(n(n^3+n)-n^2) = \mathcal{O}(n^4 + n^2 - n^2) = \mathcal{O}(n^4)$
  \item $\mathcal{O}(e^{1000}+ln (n)) = \mathcal{O}(ln (n))$
  \item $\mathcal{O}(-1+e+(ln(n))*n +1/n) = \mathcal{O}((ln(n)) * n + 1/n) = \mathcal{O}((ln(n)) * n)$
\end{enumerate}
\end{frame}

\setcounter{exercise}{30}
%%%%%%%%%%%%%%%%%%%%%%%%%%%%%%%%%%%%%%%%%%%%%%%%%%%%%%%%%%%%%%%%%%%%%%%%
\def\stitle{\theexercise\ - Komplexit\"at}
\section{\stitle}
\begin{frame}%
  \frametitle{\stitle}%

Geben Sie jeweils den \emph{asymptotischen Rechenaufwand} der Methoden in Abh\"angigkeit von $n$ als \emph{Landau-Symbol} an.
M\"ogliche Landau-Symbole sind $\mathcal{O}(1)$, $\mathcal{O}(\log n)$, $\mathcal{O}(n)$, $\mathcal{O}(n \log n)$, $\mathcal{O}(n^2)$, $\mathcal{O}(n^3)$, $\mathcal{O}(b^n)$, $\mathcal{O}(n!)$.
\medskip

\lstinputlisting[style=JAVA,frame=l]{\getexercisefolder/a.java}

\end{frame}

%%%%%%%%%%%%%%%%%%%%%%%%%%%%%%%%%%%%%%%%%%%%%%%%%%%%%%%%%%%%%%%%%%%%%%%%
\begin{frame}%
  \frametitle{\stitle\ - Teilaufgaben}%
\lstinputlisting[style=JAVA,frame=l]{\getexercisefolder/b.java}
\lstinputlisting[style=JAVA,frame=l]{\getexercisefolder/c.java}
\end{frame}

%%%%%%%%%%%%%%%%%%%%%%%%%%%%%%%%%%%%%%%%%%%%%%%%%%%%%%%%%%%%%%%%%%%%%%%%
\begin{frame}[t]%
  \frametitle{\stitle\ - Teilaufgaben}%
\lstinputlisting[style=JAVA,frame=l]{\getexercisefolder/d.java}
\lstinputlisting[style=JAVA,frame=l]{\getexercisefolder/e.java}
\end{frame}

%%%%%%%%%%%%%%%%%%%%%%%%%%%%%%%%%%%%%%%%%%%%%%%%%%%%%%%%%%%%%%%%%%%%%%%%
\begin{frame}[t]%
  \frametitle{\stitle\ - Teilaufgaben}%
\lstinputlisting[style=JAVA,frame=l]{\getexercisefolder/f.java}
\end{frame}



\begin{frame}
\centering
\Huge\GREEN{Fragen?}
\vspace{2cm}

{\LARGE
N\"achste \"Ubung: 11. Dezember\\
Besprechung Arbeitsblatt 8
}
\end{frame}


%%%%%%%%%%%%%%%%%%%%%%%%%%%%%%%%%%%%%%%%%%%%%%%%%%%%%%%%%%%%%%%%%%%%%%%%
\end{document}
