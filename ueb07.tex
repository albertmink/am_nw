\documentclass[c,18pt]{beamer}
\listfiles

\mode<presentation>
{
  \usetheme[deutsch,titlepage0]{KIT}
  \setbeamercovered{transparent}
  \setbeamertemplate{enumerate items}[ball]
}
\usepackage[utf8]{inputenc}
\date{03.12.2019}

\newlength{\Ku}
\setlength{\Ku}{1.43375pt}

\usepackage{templateSlide/exercises}
\usepackage[java]{code}

\exercisespath{uebungsfolien/}

%% title slide
\title{Übung 7: Einstieg in die Informatik und algorithmische Mathematik}
\subtitle{Albert Mink | Wintersemester 2019/2020}

\author[Albert Mink, ]{KIT}

\AuthorTitleSep{\relax}

\institute[Institut für Angewandte und Numerische Mathematik (IANM)]{Institut für Angewandte und Numerische Mathematik}

\TitleImage[width=\titleimagewd]{logos/KIT-Titel}
\logo{\includegraphics[width=\KITlogowd]{logos/lbrg_logo}}

%%%%%%%%%%%%%%%%%%%%%%%%%%%%%%%%%%%%%%%%%%%%%%%%%%%%%%%%%%%%%%%%%%%%%%%%
% Document
%%%%%%%%%%%%%%%%%%%%%%%%%%%%%%%%%%%%%%%%%%%%%%%%%%%%%%%%%%%%%%%%%%%%%%%%
\begin{document}
\begin{frame}
  \maketitle
\end{frame}
%%%%%%%%%%%%%%%%%%%%%%%%%%%%%%%%%%%%%%%%%%%%%%%%%%%%%%%%%%%%%%%%%%%%%%%%

\begin{frame}
  \frametitle{Übung 7}%
\tableofcontents[hideallsubsections]
\end{frame}

\def\kap{0}
%%%%%%%%%%%%%%%%%%%%%%%%%%%%%%%%%%%%%%%%%%%%%%%%%%%%%%%%%%%%%%%%%%%%%%%%
%%%%%%%%%%%%%%%%%%%%%%%%%%%%%%%%%%%%%%%%%%%%%%%%%%%%%%%%%%%%%%%%%%%%%%%%
%\section{Ank\"undigung}\label{K:ank}
%\begin{frame}
%  \frametitle{\ref{K:ank}. Ank\"undigung}%
%\tableofcontents[current]
%\end{frame}
%
%\begin{frame}
%  \frametitle{\ref{K:ank}. Ank\"undigung}%
%
%\begin{itemize}
%\item Am Dienstag den 05.12.2017 findet die Evaluation der \"Ubung statt.
%\item Im Januar 2018 findet eine 90 min\"utige Probeklausur statt, in der Sie die M\"oglichkeit haben unter realen Bedingungen eine Altklausur zu l\"osen.
%\item In der darauf folgenden Woche wird die L\"osung besprochen.
%\end{itemize}
%
%\end{frame}


\def\kap{2}
\inputexercise{wdh-felderMehrdim}
\setcounter{exercise}{20}
\inputexercise{felder-matrix}
\setcounter{exercise}{21}
\inputexercise{rekursion-1}
\setcounter{exercise}{22}
\inputexercise{komplexitaet-eff}
\setcounter{exercise}{23}
\inputexercise{komplexitaet-1}

%%%%%%%%%%%%%%%%%%%%%%%%%%%%%%%%%%%%%%%%%%%%%%%%%%%%%%%%%%%%%%%%%%%%%%%%
\section{Zusammenfassung}
\begin{frame}
  \frametitle{Zusammenfassung}%
\tableofcontents[hideallsubsections]
\end{frame}

\begin{frame}
\centering
\Huge\textcolor{KITgreen}{Fragen?}
\vspace{2cm}

{\LARGE
N\"achste \"Ubung: 10. Dezember
}
\end{frame}


%%%%%%%%%%%%%%%%%%%%%%%%%%%%%%%%%%%%%%%%%%%%%%%%%%%%%%%%%%%%%%%%%%%%%%%%
\end{document}
