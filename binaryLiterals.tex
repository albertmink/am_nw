\documentclass[11pt,helvet,german,worksheet]{ianmdoc_javaarbeitsblatt}
\usepackage{amsmath}
\usepackage{amssymb}
\usepackage{amsthm}
\usepackage{dsfont}
\usepackage[java]{code}
\usepackage{exercises}
\usepackage{JAVA1718ws}


\date{\today}
\sheetnumber{98}


\begin{document}
\exercisespath{./uebungsblaetter/}
\setcounter{exercise}{98}

\begin{exercise}{Ergänzung zur Diskussion um Aufgabe 6}
\begin{body}
Der Datentyp \code{byte} kann Ganzzahlen von -128 bis 127 darstellen.
Soweit war ja noch alles klar.
Irreführend war dann der Teil der die Aufgabe gar nicht abdeckte, wir uns aber in der Übungs fragten.
Wenn $01111111$ die Zahl 127 ist, wie schreibt sich dann $-127$?
Grob gesagt: Die Darstellung negativer Zahlen und das führende bit.

\textbf{Ein Bespiel:}
Die Zahl 3 in lautet in binär Darstellung $11_2$ bzw. in 8 Bits $00000011$.
Für den negativen Ausdruck werden die Bits geflippt und 1 drauf addiert, d.h. die Zahl -3 schreibt sich dann $11111101$ (\textbf{Zweierkomplement}).

\begin{itemize}
\item[] Java-Code, vorangestellte \code{0b} zeigt binär Darstellung an.
\item \code{byte b = (byte)(0b00000011); // 3}
\item \code{byte h = (byte)(0b11111101); //-3}
\item \code{byte h = (byte)(0b01111111); // 127}
\item \code{byte h = (byte)(0b10000001); // -127}
\item \code{byte h = (byte)(0b00000000); // 0}
\item \code{byte h = (byte)(0b11111111); // -128}
\end{itemize}

Dieser Sachverhalt wird in der Vorlesung noch behandelt.
Da wir aber so wissbergierdig waren, haben uns in der Übung diese spannenden Fragen gestellt, deswegen an dieser Stelle die Antwort.
\end{body}
\end{exercise}
\end{document}
