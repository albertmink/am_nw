\documentclass[c,18pt]{beamer}
\listfiles

\mode<presentation>
{
  \usetheme[deutsch,titlepage0]{KIT}
  \setbeamercovered{transparent}
  \setbeamertemplate{enumerate items}[ball]
}
\usepackage[utf8]{inputenc}
\date{07.01.2020}

\newlength{\Ku}
\setlength{\Ku}{1.43375pt}

\usepackage{templateSlide/exercises}
\usepackage[java]{code}

\exercisespath{uebungsfolien/}

%% title slide
\title{Übung 10: Einstieg in die Informatik und algorithmische Mathematik}
\subtitle{Albert Mink | Wintersemester 2019/2020}

\author[Albert Mink, ]{KIT}

\AuthorTitleSep{\relax}

\institute[Institut für Angewandte und Numerische Mathematik (IANM)]{Institut für Angewandte und Numerische Mathematik}

\TitleImage[width=\titleimagewd]{logos/KIT-Titel}
\logo{\includegraphics[width=\KITlogowd]{logos/lbrg_logo}}

%%%%%%%%%%%%%%%%%%%%%%%%%%%%%%%%%%%%%%%%%%%%%%%%%%%%%%%%%%%%%%%%%%%%%%%%
% Document
%%%%%%%%%%%%%%%%%%%%%%%%%%%%%%%%%%%%%%%%%%%%%%%%%%%%%%%%%%%%%%%%%%%%%%%%
\begin{document}
\begin{frame}
  \maketitle
\end{frame}
%%%%%%%%%%%%%%%%%%%%%%%%%%%%%%%%%%%%%%%%%%%%%%%%%%%%%%%%%%%%%%%%%%%%%%%%



\begin{frame}
  \frametitle{Ankündigung}%
  
  \heading{Termine der Übung in 2020}
  \begin{description}
      \item[14. Januar] Probeklausur (Hauptklausur aus WS2016/2017)
      \item[21. Januar] Besprechung Probeklausur
      \item[28. Januar] Entfällt
  \end{description}
  \hfill
  
  \heading{Klausur am 30. Januar, siehe Merkblatt}
\end{frame}


\begin{frame}
  \frametitle{Übung 10}%
\tableofcontents[hideallsubsections]
\end{frame}

\inputexercise{wdh-klassen}

\setcounter{exercise}{29}
\inputexercise{klassen-komplex}
%TODO AM erstelle wdh vererbung
\setcounter{exercise}{30}
\inputexercise{klassen-hierarchie}

%%%%%%%%%%%%%%%%%%%%%%%%%%%%%%%%%%%%%%%%%%%%%%%%%%%%%%%%%%%%%%%%%%%%%%%%
\section{Zusammenfassung}
\begin{frame}
  \frametitle{Zusammenfassung}%
\tableofcontents[hideallsubsections]
\end{frame}

\begin{frame}
\centering
\Huge\textcolor{KITgreen}{Fragen?}
\vspace{2cm}

{\LARGE
N\"achste \"Ubung: 14. Januar
}
\end{frame}


%%%%%%%%%%%%%%%%%%%%%%%%%%%%%%%%%%%%%%%%%%%%%%%%%%%%%%%%%%%%%%%%%%%%%%%%
\end{document}
