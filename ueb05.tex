\documentclass[9pt,german]{beamer}%
\usepackage{master/templates/beamerthemeKIT}
\usepackage{master/templates/exercises}

%\documentclass[9pt,trans]{beamer}%

%%%%%%%%%%%%%%%%%%%%%%%%%%%%%%%%%%%%%%%%%%%%%%%%%%%%%%%%%%%%%%%%%%%%%%%%
% Packages
%%%%%%%%%%%%%%%%%%%%%%%%%%%%%%%%%%%%%%%%%%%%%%%%%%%%%%%%%%%%%%%%%%%%%%%%
%\usepackage{beamerthemeKIT}
\usepackage[utf8]{inputenc}
% \usepackage[german]{babel}
\usepackage{helvet}
\usepackage[T1]{fontenc}
\usepackage{amsmath, amsthm, amssymb}
\usepackage{graphicx}
\usepackage{listings}
\usepackage{hyperref}
\usepackage{KITcolors}
%%%%%%%%%%%%%%%%%%%%%%%%%%%%%%%%%%%%%%%%%%%%%%%%%%%%%%%%%%%%%%%%%%%%%%%%
% Layout
%%%%%%%%%%%%%%%%%%%%%%%%%%%%%%%%%%%%%%%%%%%%%%%%%%%%%%%%%%%%%%%%%%%%%%%%
\renewcommand{\rmdefault}{phv}
\newlength{\movetitlegrafics}
\setlength{\movetitlegrafics}{0.0\paperwidth}
%%%%%%%%%%%%%%%%%%%%%%%%%%%%%%%%%%%%%%%%%%%%%%%%%%%%%%%%%%%%%%%%%%%%%%%%
% Definitions
%%%%%%%%%%%%%%%%%%%%%%%%%%%%%%%%%%%%%%%%%%%%%%%%%%%%%%%%%%%%%%%%%%%%%%%%
%%% Counter
\newcounter{ctlit}%
\setcounter{ctlit}{1}%
\newcounter{kap}

%%% Textfarben
\newcommand\BLUE[1]{\textcolor{KITblue}{#1}}%
\newcommand\BLACK[1]{\textcolor{KITblack}{#1}}%
\newcommand\GREEN[1]{\textcolor{KITgreen}{#1}}%
\newcommand\LBLUE[1]{\textcolor{KITblue15}{#1}}%
\newcommand\RED[1]{\textcolor{KITred}{#1}}%
%%% Layout
\def\cmt#1{\BLUE{#1}}%
\def\code#1{\texttt{#1}}%
\def\codeblock#1{\GREEN{\texttt{#1}}}%
\def\ftext#1{\textbf{{#1}}}%
\def\headline#1{\large\textbf{\GREEN{#1}}}%
\def\KITitem{\KITBullet[2mm]\ }%
\def\q{\quad}%
\def\qq{\qquad}%
\def\subtitle#1{{\small #1}}%
\def\tab{\hspace*{0.5cm}}%
%%% Standards
%\input{D:/TeX/formate/mystyle/defs}%
%\input{/home/ae01/tex/formate/mystyle/defs}%
%%% Text
%%% Symbols
\def\bs{$\backslash$}%
\def\hpm{\hphantom{$-$}}%
\def\meps{\epsilon}%
%%% code environment
\lstset{
  showstringspaces=false,
  numbers=none,
  keywordstyle=\color{KITgreen},  % coloring and formatting of keywords as public, class, import 
  commentstyle=\color{KITblue}\small\ttfamily, % Color of comments
  stringstyle=\color{KITred},
  breaklines=true
}

\lstdefinestyle{JAVA}
{
  language=Java,
  basicstyle=\ttfamily, % defines text formatting
  frame=tb  % print top and bottom lines, frame=single, L
}

\lstdefinestyle{JAVAsmall}
{
  language=Java,
  basicstyle=\small\ttfamily, % defines text formatting
  frame=tb  % print top and bottom lines, frame=single, L
}

\lstdefinestyle{JAVAlines}
{
  language=Java,
  numbers=left,
  numberstyle=\color{KITblack50}\ttfamily,
  numbersep=4pt, % distance between numbers and code1_1
  xleftmargin=4pt, % distance from frame to listing
  xrightmargin=4pt, % distance from frame to listing
  basicstyle=\ttfamily, % defines text formatting
  keywordstyle=\color{KITgreen},  % coloring and formatting of keywords as public, class, import 
  commentstyle=\color{KITblue}\ttfamily, % Color of comments
  frame=tb  % print top and bottom lines, frame=single, L
}

\lstdefinestyle{JAVAsmalllines}
{
  language=Java,
  numbers=left,
  numberstyle=\color{KITblack50}\ttfamily,
  numbersep=4pt, % distance between numbers and code1_1
  xleftmargin=4pt, % distance from frame to listing
  xrightmargin=4pt, % distance from frame to listing
  basicstyle=\small\ttfamily, % defines text formatting
  keywordstyle=\color{KITgreen},  % coloring and formatting of keywords as public, class, import 
  commentstyle=\color{KITblue}\ttfamily, % Color of comments
  frame=tb  % print top and bottom lines, frame=single, L
}


\lstdefinestyle{BASH}{
  basicstyle=\Large\ttfamily, % defines text formatting
  frame=none,
  language=bash
}


%%%%%%%%%%%%%%%%%%%%%%%%%%%%%%%%%%%%%%%%%%%%%%%%%%%%%%%%%%%%%%%%%%%%%%%%
% Titlepage
%%%%%%%%%%%%%%%%%%%%%%%%%%%%%%%%%%%%%%%%%%%%%%%%%%%%%%%%%%%%%%%%%%%%%%%%
\title[Institut f\"ur Angewandte und Numerische Mathematik]%
 {\fontsize{15}{15}\selectfont{}
  \"Ubung:
  \textit{Einstieg in die Informatik}\\[1.5mm]
  \textit{\phantom{\"Ubung:}\; und algorithmische Mathematik}\\[1.5mm]
  }%\hspace*{3cm}\normalsize{f\"ur Mathematiker}}
\author{\fontsize{9}{9}\selectfont{}
 Albert Mink\
  }
\institute[Institut f\"ur Angewandte und Numerische Mathematik]
 {\fontsize{6}{6}\selectfont{}%
  Institut f\"ur Angewandte und Numerische Mathematik}
\date{Wintersemester 2018/19}%
\subject{}%
%\beamerdefaultoverlayspecification{<+->}
%%%%%%%%%%%%%%%%%%%%%%%%%%%%%%%%%%%%%%%%%%%%%%%%%%%%%%%%%%%%%%%%%%%%%%%%



\makeatletter
\def\input@path{{uebungsfolien/}}
\graphicspath{{uebungsfolien/}}
\makeatother


%%%%%%%%%%%%%%%%%%%%%%%%%%%%%%%%%%%%%%%%%%%%%%%%%%%%%%%%%%%%%%%%%%%%%%%%
% Document
%%%%%%%%%%%%%%%%%%%%%%%%%%%%%%%%%%%%%%%%%%%%%%%%%%%%%%%%%%%%%%%%%%%%%%%%
\begin{document}
\maketitle%
\addtocounter{framenumber}{-1}%
%%%%%%%%%%%%%%%%%%%%%%%%%%%%%%%%%%%%%%%%%%%%%%%%%%%%%%%%%%%%%%%%%%%%%%%%

%%%%%%%%%%%%%%%%%%%%%%%%%%%%%%%%%%%%%%%%%%%%%%%%%%%%%%%%%%%%%%%%%%%%%%%%
%%%%%%%%%%%%%%%%%%%%%%%%%%%%%%%%%%%%%%%%%%%%%%%%%%%%%%%%%%%%%%%%%%%%%%%%
\begin{frame}
  \frametitle{Arbeitsblatt 5}%
\tableofcontents
\end{frame}
\setcounter{exercise}{20}
%%%%%%%%%%%%%%%%%%%%%%%%%%%%%%%%%%%%%%%%%%%%%%%%%%%%%%%%%%%%%%%%%%%%%%%%


%%%%%%%%%%%%%%%%%%%%%%%%%%%%%%%%%%%%%%%%%%%%%%%%%%%%%%%%%%%%%%%%%%%%%%%%
%%%%%%%%%%%%%%%%%%%%%%%%%%%%%%%%%%%%%%%%%%%%%%%%%%%%%%%%%%%%%%%%%%%%%%%%
\section{Wiederholung}\label{K:wdh}
\begin{frame}
  \frametitle{\ref{K:wdh} Wiederholung}%
\tableofcontents[current]
\end{frame}
%%%%%%%%%%%%%%%%%%%%%%%%%%%%%%%%%%%%%%%%%%%%%%%%%%%%%%%%%%%%%%%%%%%%%%%%

%%%%%%%%%%%%%%%%%%%%%%%%%%%%%%%%%%%%%%%%%%%%%%%%%%%%%%%%%%%%%%%%%%%%%%%%
%%%%%%%%%%%%%%%%%%%%%%%%%%%%%%%%%%%%%%%%%%%%%%%%%%%%%%%%%%%%%%%%%%%%%%%%
\def\stitle{Erzeugen von Feldern}
\subsection{\stitle}\label{S:Erzeugen}
\begin{frame}[t]%
  \frametitle{\ref{K:wdh}.\ref{S:Erzeugen} \stitle}
\medskip

Ein Feld kann mehrere Variablen vom selben Datentyp enthalten, hier \code{int}.
\lstinputlisting[style=JAVAlines]{wdh-felder/FelderErzeugen.java}

\end{frame}
%%%%%%%%%%%%%%%%%%%%%%%%%%%%%%%%%%%%%%%%%%%%%%%%%%%%%%%%%%%%%%%%%%%%%%%%


%%%%%%%%%%%%%%%%%%%%%%%%%%%%%%%%%%%%%%%%%%%%%%%%%%%%%%%%%%%%%%%%%%%%%%%%
%%%%%%%%%%%%%%%%%%%%%%%%%%%%%%%%%%%%%%%%%%%%%%%%%%%%%%%%%%%%%%%%%%%%%%%%
\def\stitle{Zugreifen auf Felder}
\subsection{\stitle}\label{S:Zugreifen}
\begin{frame}[t]%
  \frametitle{\ref{K:wdh}.\ref{S:Zugreifen} \stitle}
\medskip

Mit dem Operator \code{[]} wird auf bestimmte Feldkomponenten zugegriffen.
\lstinputlisting[style=JAVAlines]{wdh-felder/FelderZugreifen.java}

\end{frame}
%%%%%%%%%%%%%%%%%%%%%%%%%%%%%%%%%%%%%%%%%%%%%%%%%%%%%%%%%%%%%%%%%%%%%%%%

%%%%%%%%%%%%%%%%%%%%%%%%%%%%%%%%%%%%%%%%%%%%%%%%%%%%%%%%%%%%%%%%%%%%%%%%
%%%%%%%%%%%%%%%%%%%%%%%%%%%%%%%%%%%%%%%%%%%%%%%%%%%%%%%%%%%%%%%%%%%%%%%%
\def\stitle{Felder Bibliothek in Java}
\subsection{\stitle}\label{S:bequem}
\begin{frame}[t]%
  \frametitle{\ref{K:wdh}.\ref{S:bequem} \stitle}
\medskip

In \code{java.util.Arrays} werden n\"utzliche Funktion der Java Bibliothek bereit gestellt.
\lstinputlisting[style=JAVAlines]{wdh-felder/FelderAdv.java}

\end{frame}
%%%%%%%%%%%%%%%%%%%%%%%%%%%%%%%%%%%%%%%%%%%%%%%%%%%%%%%%%%%%%%%%%%%%%%%%


%%%%%%%%%%%%%%%%%%%%%%%%%%%%%%%%%%%%%%%%%%%%%%%%%%%%%%%%%%%%%%%%%%%%%%%%
%%%%%%%%%%%%%%%%%%%%%%%%%%%%%%%%%%%%%%%%%%%%%%%%%%%%%%%%%%%%%%%%%%%%%%%%
\def\stitle{Felder Beispiel Anwendung, Ger\"ust}
\subsection{\stitle}\label{S:BeispielG}
\begin{frame}[t]%
  \frametitle{\ref{K:wdh}.\ref{S:BeispielG} \stitle}
\medskip

Lese Vektor ein und berechne die Norm.
\lstinputlisting[style=JAVAsmalllines]{wdh-felder/FelderAnwBare.java}

\end{frame}
%%%%%%%%%%%%%%%%%%%%%%%%%%%%%%%%%%%%%%%%%%%%%%%%%%%%%%%%%%%%%%%%%%%%%%%%


%%%%%%%%%%%%%%%%%%%%%%%%%%%%%%%%%%%%%%%%%%%%%%%%%%%%%%%%%%%%%%%%%%%%%%%%
%%%%%%%%%%%%%%%%%%%%%%%%%%%%%%%%%%%%%%%%%%%%%%%%%%%%%%%%%%%%%%%%%%%%%%%%
\def\stitle{Felder Beispiel Anwendung}
\subsection{\stitle}\label{S:Beispiel}
\begin{frame}[t]%
  \frametitle{\ref{K:wdh}.\ref{S:Beispiel} \stitle}
\medskip

Lese Vektor ein und berechne die Norm.
\lstinputlisting[style=JAVAsmalllines]{wdh-felder/FelderAnw.java}

\end{frame}
%%%%%%%%%%%%%%%%%%%%%%%%%%%%%%%%%%%%%%%%%%%%%%%%%%%%%%%%%%%%%%%%%%%%%%%%


\setcounter{exercise}{17}
\begin{frame}[t]%
\medskip

\begin{exercise}{Felder: Grundlagen}
\begin{body}
Felder k"onnen aus den Datentypen \code{double}, \code{int}, \code{char}, \code{String}, $\ldots$ bestehen.
Der Index aller Felder beginnt mit '0', d.h. ein Feld mit 10 Elementen hat die Indizes von 0 bis 9.
\medskip

\lstinputlisting[style=JAVAlines]{felder-grundl/FelderGrundlagen.java}

Was wird auf dem Bildschirm ausgegeben?
\end{body}
\end{exercise}
\end{frame}

\setcounter{exercise}{18}
\def\stitle{\theexercise\ - Felder: Binäre Suche}
\section{\stitle}

\begin{frame}[t]%
    \frametitle{\stitle}
\medskip
Gegeben sei untenstehende Implementierung der Binären Suche.
\lstinputlisting[style=JAVAsmalllines]{\getexercisefolder/Binsearch.java}
\end{frame}


\begin{frame}
 \frametitle{\stitle \, - Aufgabenstellung}%
\medskip

Hier ein Programmausschnitt.
Welche Ausgabe liefert das Programm f"ur die Eingaben 9 bzw. 10 zur"uck?
\lstinputlisting[style=JAVAlines]{\getexercisefolder/Binsearch_snippet.java}
\end{frame}

\setcounter{exercise}{19}
\begin{frame}[t]%
  \setcounter{exercise}{22} % TODO make exercise number dynamic
  \medskip
  \begin{exercise}{Felder (a)}
  \begin{body}
  Betrachten Sie den folgenden Ausschnitt eines Java-Programms.
  \lstinputlisting[style=JAVAlines]{felder-felder/Feldera.java}
  \begin{parts}
  \item Was wird auf der Konsole ausgegeben?
  \pause
  \item \code{[1, 3, 6, 10, 15]}
  \end{parts}
  \end{body}
  \end{exercise}
\end{frame}


\begin{frame}[t]%
  \setcounter{exercise}{22} % TODO make exercise number dynamic
  \medskip
  \begin{exercise}{Felder (b)}
  \begin{body}
  Betrachten Sie den folgenden Ausschnitt eines Java-Programms.
  \lstinputlisting[style=JAVAlines]{felder-felder/Felderb.java}
  \begin{parts}
  \item Was wird auf der Konsole ausgegeben?
  \pause
  \item \code{[1, 3, 5, 7, 9]}
  \end{parts}
  \end{body}
  \end{exercise}
\end{frame}

\begin{frame}[t]%
  \setcounter{exercise}{22} % TODO make exercise number dynamic
  \medskip
  \begin{exercise}{Felder (c)}
  \begin{body}
  Betrachten Sie den folgenden Ausschnitt eines Java-Programms.
  \lstinputlisting[style=JAVAlines]{felder-felder/Felderc.java}
  \begin{parts}
  \item Was wird auf der Konsole ausgegeben?
  \pause
  \item \code{[1, 2, 3, 2, 1]}
  \end{parts}
  \end{body}
  \end{exercise}
\end{frame}


\begin{frame}[t]%
  \setcounter{exercise}{22} % TODO make exercise number dynamic
  \medskip
  \begin{exercise}{Felder (d)}
  \begin{body}
  Betrachten Sie den folgenden Ausschnitt eines Java-Programms.
  \lstinputlisting[style=JAVAlines]{felder-felder/Felderd.java}
  \begin{parts}
  \item Was wird auf der Konsole ausgegeben?
  \pause
  \item \code{[0, 0, 3, 3, 3]}
  \end{parts}
  \end{body}
  \end{exercise}
\end{frame}

\begin{frame}[t]%
  \setcounter{exercise}{22} % TODO make exercise number dynamic
  \medskip
  \begin{exercise}{Felder (e)}
  \begin{body}
  Betrachten Sie den folgenden Ausschnitt eines Java-Programms.
  \lstinputlisting[style=JAVAlines]{felder-felder/Feldere.java}
  \begin{parts}
  \item Was wird auf der Konsole ausgegeben?
  \pause
  \item \code{[1, 4, 9, 16, 25]}
  \end{parts}
  \end{body}
  \end{exercise}
\end{frame}

\begin{frame}[t]%
\setcounter{exercise}{22} % TODO make exercise number dynamic
  \medskip
  \begin{exercise}{Felder (f)}
  \begin{body}
  Betrachten Sie den folgenden Ausschnitt eines Java-Programms.
  \lstinputlisting[style=JAVAlines]{felder-felder/Felderf.java}
  \begin{parts}
  \item Was wird auf der Konsole ausgegeben?
  \pause
  \item \code{[5, 4, 3, 2, 1]}
  \end{parts}
  \end{body}
  \end{exercise}
\end{frame}


\begin{frame}
\centering
\Huge\GREEN{Fragen?}
\vspace{2cm}

{\LARGE
N\"achste \"Ubung: 27. November\\
Besprechung Arbeitsblatt 6
}
\end{frame}


%%%%%%%%%%%%%%%%%%%%%%%%%%%%%%%%%%%%%%%%%%%%%%%%%%%%%%%%%%%%%%%%%%%%%%%%
\end{document}
