\begin{exercise}{Bias-Darstellung}

\begin{body}
Die Bias-Darstellung ist eine Möglichkeit, ganze Zahlen in einem bestimmten Bereich durch vorzeichenlose Zahlen darzustellen. Dazu definiert man zunächst eine nichtnegative, ganze Zahl $B$ genannt \emph{Bias} (engl. Neigung, Verzerrung). Die \emph{Bias-$B$-Darstellung} einer ganzen Zahl $z \in \mathbb{Z}$ ist dann durch $z + B$ gegeben. 
Auf diese Weise können ganze Zahlen größer oder gleich $-B$ durch vorzeichenlose Zahlen dargestellt werden.

Geben Sie die Bias-$7$-Darstellung der folgenden Dezimalzahlen in Binärdarstellung an
\begin{center}
\begin{minipage}{0.3\textwidth}
\begin{parts}
\item[(a)] $3$
\item[(b)] $5$
\item[(c)] $-1$
\end{parts}
\end{minipage}
\begin{minipage}{0.3\textwidth}
\begin{parts}
\item[(d)] $-3$
\item[(e)] $8$
\item[(f)] $4$
\end{parts}
\end{minipage}
\begin{minipage}{0.3\textwidth}
\begin{parts}
\item[(g)] $-5$
\item[(h)] $-7$
\item[(i)] $0$
\end{parts}
\end{minipage}
\end{center}
\end{body}


\begin{solution}
\begin{center}
\begin{minipage}{0.3\textwidth}
\begin{parts}
\item[(a)] $3 + 7  = 10 = 1010_2$
\item[(b)] $5 + 7  = 12 = 1100_2$
\item[(c)] $-1 + 7 = 6  = 110_2$
\end{parts}
\end{minipage}
\begin{minipage}{0.3\textwidth}
\begin{parts}
\item[(d)] $-3 + 7 = 4 = 100_2$
\item[(e)] $8 + 7 = 15 = 1111_2$
\item[(f)] $4 + 7 = 11 = 1011_2$
\end{parts}
\end{minipage}
\begin{minipage}{0.3\textwidth}
\begin{parts}
\item[(g)] $-5 + 7 = 2 = 10_2$
\item[(h)] $-7 + 7 = 0 = 0_2$
\item[(i)] $0 + 7 = 7  = 111_2$
\end{parts}
\end{minipage}
\end{center}
\end{solution}
\end{exercise}
