\begin{exercise}{Funktionen}
\begin{body}
Es seien die folgenden Java-Funktionen f1 und f2 gegeben:
\begin{verbatim}
static int f1(int x, int y)
{
   if (y <= 0) return 1;
   if (y % 2 == 0)
      return f1(x*x, y/2);
   else
      return x*f1(x, y-1);
}
\end{verbatim}
\begin{verbatim}
static int f2(int x, int y)
{
   if (y <= 0) return 1;
   int result = 1;
   while(...){
     if(y % 2 == 0){
        x *= x;
        y /= 2;
     }
     else{
        result *= x;
        y--;
     }
   }  
   return result;
}
\end{verbatim}
\begin{itemize}
	\item[a)] Welche Funktion wird in f1 berechnet?
	\item[b)] Wieviele Multiplikationen werden f"ur den Aufruf f1(3, 4) ben"otigt?
	\item[c)] Die (noch unvollst"andige) Funktion f2 soll f"ur alle Eingaben exakt das gleiche Ergebnis 
	          zur"uckgeben wie f1. Welche Bedingung ist in der while--Schleife einzusetzen?
\end{itemize}
\end{body}

%%%%%%%%%%%%%%%%%%%%%%%%%%%%%%%%%%%%% solution %%%%%%%%%%%%%%%%%%%%%%%%%%%%%%%%%%%%%%%%%%%%%%%%%%%%%%%%%%%%%%%
\begin{solution}
\begin{itemize}
	  \item[(a)] $f(x)=x^{y}, \quad y\geq0$.
	  \item[(b)] 3
	  \item[(c)] $while(y>=1)$
  \end{itemize}
\end{solution}

\end{exercise}