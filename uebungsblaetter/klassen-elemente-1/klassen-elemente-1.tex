\begin{exercise}{Klassen- und Instanzelemente}

\begin{body}
Die Klasse \code|Klasse| sei folgendermaßen definiert
\medskip
\begin{displaycode}
public class Klasse {
    public static int variable1 = 1;
    public int variable2 = 2;
}
\end{displaycode}
\medskip
In der \code|main|-Methode eines Java-Programms findet man die Zeile
\medskip
\begin{displaycode}
        Klasse instanz = new Klasse();
\end{displaycode}
\medskip
Entscheiden Sie, welche der nachfolgenden Ausdrücke gültig sind. Begründen Sie Ihre Entscheidung.
\begin{center}
\begin{minipage}{0.45\textwidth}
\begin{parts}
\item[(a)]
\code|Klasse.variable1 += 1;|

\item[(b)]
\code|instanz.variable2 += 1;|
\end{parts}
\end{minipage}
\begin{minipage}{0.45\textwidth}
\begin{parts}
\item[(c)]
\code|Klasse.variable2 += 1;|

\item[(d)]
\code|instanz.variable1 += 1;|
\end{parts}
\end{minipage}
\end{center}
\end{body}

\begin{solution}
Klassenelemente (d.h{.} Klassenvariablen und -methoden) existieren unabhängig von möglichen Instanzen einer Klasse. Instanzelemente (d.h{.} Instanzvariablen und -methoden) existieren nur für einzelne Instanzen einer Klasse. Daher ergeben sich folgende Lösungen:
\begin{parts}
\item
Der Ausdruck ist gültig.

\item
Der Ausdruck ist gültig. 

\item
Der Ausdruck ist ungültig.

\item
Der Ausdruck ist gültig. 
\end{parts}
\end{solution}
\end{exercise}
