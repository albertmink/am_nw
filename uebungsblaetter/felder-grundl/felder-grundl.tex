\begin{exercise}{Felder: Grundlagen}
\begin{body}
Felder k"onnen aus den Datentypen \code{double}, \code{int}, \code{char}, \code{String}, $\ldots$ bestehen.
Der Index aller Felder beginnt mit '0', d.h. ein Feld mit 10 Elementen hat die Indizes von 0 bis 9.
\\\\
Gegeben sei der folgende Ausschnitt eines \code{Java}--Programms.
\begin{verbatim}
  int[] feld = new int[10];
  for(int i=0; i<feld.length; i++){
    feld[i]=1;
    for(int j=0; j<i; j++){
      feld[i]=feld[i]+feld[j];
    }
  }
  for(int i=0; i<feld.length; i++){
    System.out.println("feld("+i+") = "+feld[i]);
  }
\end{verbatim}
Was wird auf dem Bildschirm ausgegeben?
\end{body}

%%%%%%%%%%%%%%%%%%%%%%%%%%%%%%%%%%%%% solution %%%%%%%%%%%%%%%%%%%%%%%%%%%%%%%%%%%%%%%%%%%%%%%%%%%%%%%%%%%%%%%
\begin{solution}
Es wird
\begin{verbatim}
feld(0) = 1
feld(1) = 2
feld(2) = 4
feld(3) = 8
feld(4) = 16
feld(5) = 32
feld(6) = 64
feld(7) = 128
feld(8) = 256
feld(9) = 512
\end{verbatim}
ausgegeben.
\end{solution}

\end{exercise}
