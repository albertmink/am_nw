\begin{exercise}{Rekursion}
\begin{body}
Die Folge der \emph{Fibonacci--Zahlen} $a_0, a_1, a_2, \dotsc$ ist folgenderma\ss en definiert:
\begin{equation*}
a_0 := 1,\qquad
a_1 := 1,\qquad
a_n := a_{n-1} + a_{n-2}\quad\text{für } n = 2,3,\dotsc
\end{equation*}
Die Klassenmethode \code{fibonacci} berechnet die Fibonacci--Zahlen rekursiv.
Sie ist folgenderma\ss en definiert:
\begin{displaycode}
    static int fibonacci(int n) {
        if ( n <= 1 ) {
            return 1;
        } else {
            return fibonacci(n-1) + fibonacci(n-2);
        }
    }
\end{displaycode}
\begin{parts}
\item Geben Sie die ersten zehn Folgenglieder der Folge der Fibonacci--Zahlen an.
\item Wie oft wird die Methode durch den Aufruf \code|fibonacci(4)| aufgerufen?
%\item Geben Sie eine Definition der Methode ohne rekursive Aufrufe an.
\end{parts}
\end{body}

\begin{solution}
\begin{parts}
\item
Die ersten zehn Folgenglieder lauten: $1$, $1$, $2$, $3$, $5$, $8$, $13$, $21$, $34$, $55$.

\item
Die Methode wird insgesamt neunmal aufgerufen.

%\item
%Eine mögliche Klassendefinition ohne rekursiven Aufruf lautet
%\begin{displaycode}
%    static int fibonacci(int n) {
%        if ( n <= 1 ) {
%            return 1;
%        } else {
%            int folgenglied = 1;
%            int vorgaenger1 = 1;
%            int vorgaenger2;
%            for (int i = 2; i <= n; i++) {
%                vorgaenger2 = vorgaenger1;
%                vorgaenger1 = folgenglied;
%                folgenglied = vorgaenger1 + vorgaenger2;
%            }
%            return folgenglied;
%        }
%    }
%\end{displaycode}
\end{parts}
\end{solution}
\end{exercise}
