\begin{exercise}{Addition von Binärzahlen}
\begin{body}
Die Addition von Binärzahlen kann wie bei Dezimalzahlen stellenweise durchgeführt werden. Stellen, bei denen kein Übertrag vorliegt, werden folgendermaßen addiert (fettgedruckte Zahlen stellen Ergebnisse dar):
\medskip
\begin{center}
\begin{tabular}{l}
\textbf{Summand 1} \\
\textbf{Summand 2} \\
{\footnotesize\ } \\
\textbf{Summe}
\end{tabular}
\hspace{2em}
\begin{tabular}{rrr}
$\dotsb$   & $0$ & $\dotsb$  \\
$\dotsb$   & $0$ & $\dotsb$  \\
{\footnotesize\ } & {\footnotesize\ } & {\footnotesize\ } \\
\hline
           & $\boldsymbol{0}$ & $\dotsb$
\end{tabular}
\hspace{1.5em}
\begin{tabular}{rrr}
$\dotsb$   & $1$ & $\dotsb$  \\
$\dotsb$   & $0$ & $\dotsb$  \\
{\footnotesize\ } & {\footnotesize\ } & {\footnotesize\ } \\
\hline
           & $\boldsymbol{1}$ & $\dotsb$
\end{tabular}
\hspace{1.5em}
\begin{tabular}{rrr}
$\dotsb$   & $0$ & $\dotsb$  \\
$\dotsb$   & $1$ & $\dotsb$  \\
{\footnotesize\ } & {\footnotesize\ } & {\footnotesize\ } \\
\hline
           & $\boldsymbol{1}$ & $\dotsb$
\end{tabular}
\hspace{1.5em}
\begin{tabular}{rrr}
$\dotsb$     & $1$ & $\dotsb$  \\
$\dotsb$     & $1$ & $\dotsb$  \\
{\footnotesize $\boldsymbol{1}$} & {\footnotesize\ } & {\footnotesize\ } \\
\hline
             & $\boldsymbol{0}$ & $\dotsb$
\end{tabular}
\end{center}
\medskip
Für Stellen, bei denen Überträge vorliegen, gilt:
\medskip
\begin{center}
\begin{tabular}{l}
\textbf{Summand 1} \\
\textbf{Summand 2} \\
{\footnotesize\ } \\
\textbf{Summe}
\end{tabular}
\hspace{2em}
\begin{tabular}{rrr}
$\dotsb$     & $0$          & $\dotsb$   \\
$\dotsb$     & $0$          & $\dotsb$   \\
{\footnotesize\ }   & {\footnotesize $1$} & {\footnotesize\ } \\
\hline
             & $\boldsymbol{1}$ & $\dotsb$
\end{tabular}
\hspace{1.5em}
\begin{tabular}{rrr}
$\dotsb$     & $1$ & $\dotsb$  \\
$\dotsb$     & $0$ & $\dotsb$  \\
{\footnotesize $\boldsymbol{1}$} & {\footnotesize $1$} & {\footnotesize\ } \\
\hline
             & $\boldsymbol{0}$ & $\dotsb$
\end{tabular}
\hspace{1.5em}
\begin{tabular}{rrr}
$\dotsb$     & $0$          & $\dotsb$  \\
$\dotsb$     & $1$          & $\dotsb$  \\
{\footnotesize $\boldsymbol{1}$} & {\footnotesize $1$} & {\footnotesize\ } \\
\hline
             & $\boldsymbol{0}$ & $\dotsb$
\end{tabular}
\hspace{1.5em}
\begin{tabular}{rrr}
$\dotsb$     & $1$ & $\dotsb$   \\
$\dotsb$     & $1$ & $\dotsb$   \\
{\footnotesize $\boldsymbol{1}$} & {\footnotesize $1$} & {\footnotesize\ } \\
\hline
             & $\boldsymbol{1}$ & $\dotsb$
\end{tabular}
\end{center}
\medskip
Berechnen Sie die folgenden Binärzahlsummen, ohne eine Umrechung in ein anderes Stellenwertsystem (wie beispielsweise das der Dezimalzahlen) vorzunehmen.
\begin{center}
\begin{minipage}{0.45\textwidth}
\begin{itemize}
\item[(a)] $1010_2 +   11_2$
\item[(b)] $1101_2 + 1010_2$
\item[(c)] $1010_2 +  101_2$
\item[(d)] $1000_2 + 1000_2$
\end{itemize}
\end{minipage}
\begin{minipage}{0.45\textwidth}
\begin{itemize}
\item[(e)] $10010000_2 + 1111_2$
\item[(f)] $10001111_2 + 1_2$
\item[(g)] $10101010_2 + 11101_2$
\item[(h)] $10010001_2 + 1010011_2$
\end{itemize}
\end{minipage}
\end{center}
\end{body}

%%%%%%%%%%%%%%%%%%%%%%%%%%%%%%%%%%%%% solution %%%%%%%%%%%%%%%%%%%%%%%%%%%%%%%%%%%%%%%%%%%%%%%%%%%%%%%%%%%%%%%
\begin{solution}
\begin{center}
\begin{minipage}{0.45\textwidth}
\begin{itemize}
\item[(a)] $1010_2 +   11_2 = 1101_2$
\item[(b)] $1101_2 + 1010_2 = 10111_2$
\item[(c)] $1010_2 +  101_2 = 1111_2$
\item[(d)] $1000_2 + 1000_2 = 10000_2$
\end{itemize}
\end{minipage}
\begin{minipage}{0.45\textwidth}
\begin{itemize}
\item[(e)] $10010000_2 + 1111_2 = 10011111_2$
\item[(f)] $10001111_2 + 1_2 = 10010000_2$
\item[(g)] $10101010_2 + 11101_2 = 11000111_2$
\item[(h)] $10010001_2 + 1010011_2 = 11100100_2$
\end{itemize}
\end{minipage}
\end{center}
\end{solution}

\end{exercise}