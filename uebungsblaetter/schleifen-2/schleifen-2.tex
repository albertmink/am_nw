
\begin{exercise}{Schleifen}
\begin{body}
Gegeben sei der folgende Ausschnitt eines Java-Programms:
\begin{displaycode}
int k = 0;
while (k < 8) {
    if (k % 4 == 0) {
        System.out.print("+");
    } else {
        System.out.print("//");
    }
    k += 2;
}
\end{displaycode}
\begin{parts}
\item Was wird auf dem Bildschirm ausgegeben?
\item Realisieren Sie diesen Programmausschnitt mit einer \code|for|-Schleife.
\end{parts}
\end{body}

\begin{solution}
\begin{parts}
\item[(a)] Für die einzelnen Schleifendurchläufe ergeben sich
\begin{center}
\begin{tabular}{|c||c|c|c|c|}
\hline
\code|k| & \code|k < 8| & \code|k % 4 == 0|  & \textbf{Ausgabe} & \code|k += 2| \\
\hline
$0$      & \emph{wahr}   & \emph{wahr}        & \glqq +\grqq    & $2$ \\
$2$      & \emph{wahr}   & \emph{falsch}      & \glqq //\grqq   & $4$ \\
$4$      & \emph{wahr}   & \emph{wahr}        & \glqq +\grqq    & $6$ \\
$6$      & \emph{wahr}   & \emph{wahr}        & \glqq //\grqq   & $8$ \\
$8$      & \emph{falsch} & --                 & --              & --  \\
\hline
\end{tabular}
\end{center}
Es wird also die Zeichenkette \glqq +//+//\grqq\ auf der Konsole ausgegeben.

\item[(b)] Realisation mit einer \code|for|-Schleife:
\begin{displaycode}
for (int k = 0; k < 8; k += 2) {
    if (k % 4 == 0) {
        System.out.print("+");
    } else {
        System.out.print("//");
    }
} 
\end{displaycode}
\end{parts}
\end{solution}
\end{exercise}
