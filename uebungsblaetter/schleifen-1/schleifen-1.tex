
\begin{exercise}{Schleifen}
\begin{body}
Gegeben sei der folgende Ausschnitt eines Java-Programms:
\begin{displaycode}
int n = 0;
for (int i=0; i < 7; i++) {
    if (i == 4) {
        i = i + 1;
        n = n + i;
    } else {
        n += i;
    }
}
System.out.println(n);
\end{displaycode}
\begin{parts}
\item[(a)] Welche Zahl wird auf dem Bildschirm ausgegeben?
\item[(b)] Realisieren Sie diesen Programmausschnitt mit einer \code|while|-Schleife.
\end{parts}
\end{body}

\begin{solution}
\begin{parts}
\item[(a)] Für die einzelnen Schleifendurchläufe ergeben sich
\begin{center}
\begin{tabular}{|c|c||c|c|c|c|c|c|}
\hline
\code|i| & \code|n| & \code|i < 7|  & \code|i == 4| & \code|i = i + 1| & \code|n = n + i| & \code|n += i| & \code|i++|\\
\hline
$0$      & $0$      & \emph{wahr}   & \emph{falsch} & --               & --               & $0$           & $1$ \\
$1$      & $0$      & \emph{wahr}   & \emph{falsch} & --               & --               & $1$           & $2$ \\
$2$      & $1$      & \emph{wahr}   & \emph{falsch} & --               & --               & $3$           & $3$ \\
$3$      & $3$      & \emph{wahr}   & \emph{falsch} & --               & --               & $6$           & $4$ \\
$4$      & $6$      & \emph{wahr}   & \emph{wahr}   & $5$              & $11$             & --            & $6$ \\
$6$      & $11$     & \emph{wahr}   & \emph{falsch} & --               & --               & $17$          & $7$ \\
$7$      & $17$     & \emph{falsch} & --            & --               & --               & --            & --  \\
\hline
\end{tabular}
\end{center}
Es wird also die Zahl $17$ auf der Konsole ausgegeben.

\item[(b)] Realisation mit einer \code|while|-Schleife:
\begin{displaycode}
int n;
int i = 0;
while (i < 7) {
    if (i == 4) {
        i = i + 1;
        n = n + i;
    } else {
        n += i;
    }
    i++;
}
System.out.println(n);
\end{displaycode}
\end{parts}
\end{solution}
\end{exercise}
