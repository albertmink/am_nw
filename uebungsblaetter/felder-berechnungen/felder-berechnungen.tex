\begin{exercise}{Felder: Berechnungen}
\begin{body}
Betrachten Sie die folgenden Ausschnitte eines Java-Programms:

\begin{parts}
\item 
\begin{displaycode}
    int[] a = new  int[] {1,3,4,2,3};
    int  e = a[0];
    for( int i = 0; i < a.length; i++ ) {
      if( a[i] > e ) {
        e = a[i];
        System.out.println(e);
      }
    }
    System.out.println("Fertig mit " + e + ".");
\end{displaycode}

\item
\begin{displaycode}
    int[] a = new  int[] {1,3,4,2,3};
    int  e = 0;
    for( int i = 0; i < a.length; i++ ) {
      e += a[i];
      System.out.println(e);
    }
\end{displaycode}
\end{parts}
Was wird jeweils auf der Konsole ausgegeben?\\
Was wird in den Programmen berechnet?
\end{body}


\begin{solution}
\begin{parts}
\item
Es wird \glqq 3, 4, Fertig mit 4. \grqq\ auf der Konsole ausgegeben.

\item
Es wird \glqq 1, 4, 8, 10, 13 \grqq\ auf der Konsole ausgegeben.
\end{parts}
\end{solution}
\end{exercise}
