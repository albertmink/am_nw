
\begin{exercise}{Klassenhierarchien}
\begin{body}
Betrachten Sie die nachfolgenden Klassendefinitionen:
\medskip
\begin{displaycode}
public class Wal { }

public class Zahnwal extends Wal { }

public class Pottwal extends Zahnwal { }

public class Delphin extends Zahnwal { }

public class Bartenwal extends Wal { }

public class Grauwal extends Bartenwal { }
\end{displaycode}
\medskip
In der \code|main|-Methode eines Java-Programms findet man die Zeilen
\medskip
\begin{displaycode}
        Wal w;
        Zahnwahl z;
\end{displaycode}
\medskip
Entscheiden Sie, welche der folgenden Ausdrücke gültig sind. Begründen Sie Ihre Entscheidung.
\begin{center}
\begin{minipage}{0.45\textwidth}
\begin{parts}
\item[(a)]
\code|z = new Zahnwal();|

\item[(b)]
\code|z = new Delphin();|

\item[(c)]
\code|w = new Bartenwal();|

\item[(d)]
\code|z = new Bartenwal();|

\item[(e)]
\code|w = new Grauwal();|
\end{parts}
\end{minipage}
\begin{minipage}{0.45\textwidth}
\begin{parts}
\item[(f)]
\code|z = new Wal();|

\item[(g)]
\code|z = (Zahnwal) new Wal();|

\item[(h)]
\code|z = (Zahnwal) new Pottwal();|

\item[(i)]
\code|z = (Zahnwal) new Grauwal();|

\item[(j)]
\code|z = (Zahnwal) null;|
\end{parts}
\end{minipage}
\end{center}
\end{body}


\begin{solution}
Für die Lösung dieser Aufgabe ist das folgende Klassendiagramm hilfreich. Jeder Pfeil zeigt von einer Klasse auf deren Basisklasse.
\begin{center}
\includegraphics[height=4cm]{\filename{wale}}
\end{center}
\begin{parts}
\item
Der Ausdruck ist gültig. Der Variable \code|z| vom Typ \code|Zahnwal| wird eine Instanz der Klasse \code|Zahnwal| zugewiesen.

\item
Der Ausdruck ist gültig. Die Klasse \code|Delphin| ist von der Klasse \code|Zahnwal| abgeleitet. Daher ist jede Instanz der Klasse \code|Delphin| auch eine Instanz der Klasse \code|Zahnwal| (\glqq Delphine sind Zahnwale.\grqq).

\item
Der Ausdruck ist gültig. Die Klasse \code|Bartenwal| ist von der Klasse \code|Wal| abgeleitet. Daher ist jede Instanz der Klasse \code|Bartenwal| auch eine Instanz der Klasse \code|Wal| (\glqq Bartenwale sind Wale.\grqq).

\item
Der Ausdruck ist ungültig. Die Klasse \code|Bartenwal| und die Klasse \code|Zahnwal| sind zwar beide von der Klasse \code|Wal| abgeleitet. Die Klasse \code|Bartenwal| ist jedoch nicht von der Klasse \code|Zahnwal| abgeleitet. Aus diesem Grund sind Instanzen der Klasse \code|Bartenwal| keine Instanzen der Klasse \code|Zahnwal| (\glqq Bartenwale sind keine Zahnwale.\grqq).

\item
Der Ausdruck ist gültig. Die Klasse \code|Grauwal| ist über die Klasse \code|Bartenwal| von der Klasse \code|Wal| abgeleitet. Jede Instanz der Klasse \code|Grauwal| ist daher auch eine Instanz der Klasse \code|Wal| (\glqq Grauwale sind Wale.\grqq).

\item
Der Ausdruck ist ungültig. Die Klasse \code|Wal| ist Basisklasse der Klasse \code|Zahnwal|. Daher ist ein Instanz der Klasse \code|Wal| keine Instanz der Klasse \code|Zahnwal| (\glqq Nicht jeder Wal ist ein Zahnwal.\grqq).

\item
Der Ausdruck ist ungültig. Die Klasse \code|Wal| ist Basisklasse der Klasse \code|Zahnwal|. Die Typumwandlung von einer Basisklasse zu einer abgeleiteten Klasse ist nicht möglich.

\item
Der Ausdruck ist gültig. Die Klasse \code|Pottwal| ist von der Klasse \code|Zahnwal| abgeleitet. Die Typumwandlung von einer abgeleiteten Klasse zur entsprechenden Basisklasse ist möglich. Man beachte, dass die Typumwandlung auf die Instanz selbst keinen Einfluss hat, d.h{.} nach der Initialisierung ist in der Variable \code|z| eine Instanz der Klasse \code|Pottwal| gespeichert.

\item
Der Ausdruck ist ungültig. Die Klasse \code|Grauwal| ist nicht von der Klasse \code|Zahnwal| abgeleitet. Die Typumwandlung ist daher nicht möglich.

\item
Der Ausdruck ist gültig. Die leere Referenz \code|null| verweist auf keine Instanz. Die Typumwandlung wird ohne jeden Effekt durchgeführt.
\end{parts}
\end{solution}
\end{exercise}
