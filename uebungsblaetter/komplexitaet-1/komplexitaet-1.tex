
\begin{exercise}{Komplexität}
\begin{body}
Betrachten Sie die nachfolgenden Methodendefinitionen. Jede Methode besitzt einen formalen Parameter vom Typ \code|int|, über den eine natürliche Zahl $n \in \mathbb{N}$ an die Methode übergeben werden kann. Geben Sie jeweils den \emph{asymptotischen Rechenaufwand} (hier: Anzahl der Multiplikationen) der Methoden in Abhängigkeit von $n$ als \emph{Landau-Symbol} an. M\"ogliche Landau-Symbole sind $\mathcal{O}(1)$, $\mathcal{O}(\log n)$, $\mathcal{O}(n)$, $\mathcal{O}(n \log n)$, $\mathcal{O}(n^2)$, $\mathcal{O}(n^3)$, $\mathcal{O}(b^n)$, $\mathcal{O}(n!)$.
\begin{parts}
\item
\begin{displaycode}
    static int methode(int n) {
        int s = 1;
        for (int i=1; i <= n; i++) {
            s *= i;
        }
        return s;
    }
\end{displaycode}

\item
\begin{displaycode}
    static int methode(int n) {
        if ( n <= 0 ) {
            return 1;
        } else {
            return methode(n-1) * methode(n-1);
        }
    }
\end{displaycode}

\item
\begin{displaycode}
    static int methode(int n) {
        int s = 1;
        for (int i = 0; i < n; i++) {
            for (int j = 1; j <= n; j++) {
            	s *= j;
            }
        }
        return s;
    }
\end{displaycode}

\item
\begin{displaycode}
    static int methode(int n) {
        if ( n <= 0 ) {
            return 1;
        } else {
            return n * methode(n/2);
        }
    }
\end{displaycode}


\item
\begin{displaycode}
    static int methode(int n) {
        if (n <= 0) {
            return 1;
        } else {
            int s = 1;
            for (int i = 0; i < n; i++) {
            	s *= methode(n-1);
            }
            return s;
        }
    }
\end{displaycode}


\item
\begin{displaycode}
    static int methode(int n) {
        if ( n <= 0 ) {
            return 1;
        } else {
            return n * methode(n-1);
        } 
    }
\end{displaycode}
\end{parts}
\end{body}


\begin{solution}
\begin{parts}
\item
$O(n)$ (lineare Komplexität). Die Multiplikationen werden in einer Schleife ausgeführt, die $n$-mal durchlaufen wird. Pro Schleifendurchlauf wird genau eine Multiplikation ausgeführt. Also ergeben sich für $n$ Schleifendurchläufe genau $n$ Multiplikationen. 

\item
$O(2^n)$ (exponentielle Komplexität). Die Multiplikationen werden in einer rekursiv definierten Methode ausgeführt. Ein Methodenaufruf für einen Parameter $n > 0$ führt zu einer Multiplikation sowie zu zwei Methodenaufrufen für den Parameter $n-1$.
Für $n \leq 0$ wird keine Multiplikation ausgeführt und keine weitere Methode aufgerufen. Daher ergeben sich insgesamt $2^{n}$ Methodenaufrufe mit jeweils einer Multiplikation.


\item
$O(n^2)$ (quadratische Komplexität). Die Multiplikationen werden in einer geschachtelten Schleife ausgeführt.
Die äußere Schleife wird $n$-mal durchlaufen. Bei jedem Durchlauf der äußeren Schleife wird die innere Schleife $n$-mal Durchlaufen. Bei jedem Durchlauf der inneren Schleife wird genau eine Multiplikation ausgeführt. Also ergeben sich insgesamt $n\cdot n = n^2$ Multiplikationen.

\item
$O(\log n)$ (logarithmische Komplexität). Die Multiplikationen werden in einer rekursiv definierten Methode ausgeführt. Ein Methodenaufruf für einen Parameter $n > 0$ führt zu einer Multiplikation sowie zu einem Methodenaufruf für den Parameter $\lfloor n/2\rfloor$. Für $n \leq 0$ wird keine Multiplikation ausgeführt und keine weitere Methode aufgerufen. Daher ergeben sich insgesamt $\lfloor\log_2(n)\rfloor + 1$ Methodenaufrufe mit jeweils einer Multiplikation. 

\item
$O(n!)$ (faktorielle Komplexität). Die Multiplikationen werden in einer Schleife innerhalb einer rekursiv definierten Methode ausgeführt. Ein Methodenaufruf für einen Parameter $n > 0$ führt zu $n$ Schleifendurchläufen mit jeweils einer Multiplikation sowie zu $n$ Methodenaufrufen für den Parameter $n-1$. Für $n \leq 0$ wird keine Multiplikation ausgeführt und keine weitere Methode aufgerufen. Daher ergeben sich insgesamt $n \cdot (n-1) \dotsm 1 = n!$ Multiplikationen.

\item
$O(n)$ (lineare Komplexität). Die Multiplikationen werden in einer rekursiv definierten Methode ausgeführt. Ein Methodenaufruf mit für einen Parameter $n > 0$ führt zu einer Multiplikation sowie zu einem Methodenaufruf für den Parameter $n-1$. Für $n = 0$ wird keine Multiplikation ausgeführt und keine weitere Methode aufgerufen. Daher ergeben sich insgesamt $n$ Methodenaufrufe mit jeweils einer Multiplikation.
\end{parts}
\end{solution}
\end{exercise}