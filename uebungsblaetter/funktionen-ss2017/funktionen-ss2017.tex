\begin{exercise}{Exponentialreihe}
\description{Berechnung der Exponentialreihe}

\begin{body}
Schreiben Sie ein kompilierbares Java-Programm zur Bestimmung der Funktionswerte der Exponentialfunktion $e^x$.
Hierzu soll die Reihendarstellung
\[
  e^x = \sum_{i=0}^{\infty} \frac{x^i}{i!}=
  1+x+\frac{x^2}{2}+\frac{x^3}{6}+\ldots
\]
verwendet werden.
Die Funktionswerte sollen durch endliche Summen
\[
  e^x \approx S(N):=\sum_{i=0}^{N} \frac{x^i}{i!}=
  1+x+\frac{x^2}{2}+\frac{x^3}{6}+\ldots+\frac{x^N}{N!}
\]
angen"ahert werden.
Die einzelnen Summanden $y_i := x^i/i! $ lassen sich dabei wie folgt berechnen:
\begin{eqnarray*}
  y_0 & := & 1 \\
  y_i & := & \frac{x}{i}\,y_{i-1}, \quad i = 1,2,\ldots \ .
\end{eqnarray*}
Gehen Sie dabei wie folgt vor:
\begin{itemize}
  \item Schreiben Sie das Hauptprogramm.
        Lesen Sie dabei erst den Wert $x$ von der Konsole ein und speichern Sie diesen in einem geeigneten Datentyp ab.
  \item Verwenden Sie zur Summation eine \code+do-while+-Schleife.
        Brechen Sie die Summation ab, wenn sich zwei aufeinanderfolgende Summen $S(N)$ und $S(N+1)$ um weniger als $\varepsilon=10^{-12}$ voneinander unterscheiden.
  \item Geben Sie den berechneten Wert und die Anzahl der zur Berechnung ben\"otigten Summanden auf dem Bildschirm aus.
        Vergleichen Sie den gen\"aherten Wert mit dem Wert der Standardfunktion, vgl. \code{Math.exp()}.
\end{itemize}

\end{body}

\begin{solution}
  \inputcode{\filename{src/Expo.java}}
  \newpage
\end{solution}

\end{exercise}
