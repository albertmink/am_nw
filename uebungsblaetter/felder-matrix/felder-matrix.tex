\begin{exercise}{Matrixrechnung}
\begin{body}
Betrachten Sie das folgende Java-Programm in dem Matrizen als mehrdimensionale Felder implementiert sind.
Beachten Sie, dass Java nach dem \emph{row-major} Prinzip arbeitet (vergleich Vorlesung Abschnitt 14.4.)
\medskip

\inputcode[frame=lines]{\filename{FelderMatrixBare.java}}

Als Hilfestellung dient die Klassenmethode \code{printMatrix}, welche beliebige Matrizen auf dem Terminal ausgibt.
\begin{parts}
\item Implementieren Sie zun\"achst die Klassenmethode \code{getIdentity} die eine $3\times 3$~Einheitsmatrix zur\"uck geben soll.
\item Implementieren Sie die Klassenmethode \code{getTrace} die zu einer gegebenen Matrix die Spur als Feld zur\"uck gibt.
\item Berechnen Sie in der Klassenmethode \code{getMaxValue} den betragsm\"a\ss ig gr\"o\ss ten Eintrag einer gegebenen Matrix.
\item Berechnen Sie in der Klassenmethode \code{getSpaltensummennorm} die Spaltensummennorm $||\cdot||_1$  $$||A||_1 := \max_{\nu=1,\ldots,n} \sum_{\mu=1}^m |a_{\mu,\nu}| $$ f\"ur eine gegebene Matrix $A\in\mathbf{R}^{m\times n}$ und $m=n=3$.
\end{parts}
\end{body}

\begin{solution}
%\begin{parts}
L\"osung durch Implementierung.
\inputcode[frame=lines,title=FelderFunktionen.java]{\filename{FelderMatrix.java}}
%\end{parts}
\end{solution}
\end{exercise}
