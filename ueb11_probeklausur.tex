\documentclass[9pt,german]{beamer}%
\usepackage{master/templates/beamerthemeKIT}
\usepackage{master/templates/exercises}

%\documentclass[9pt,trans]{beamer}%

%%%%%%%%%%%%%%%%%%%%%%%%%%%%%%%%%%%%%%%%%%%%%%%%%%%%%%%%%%%%%%%%%%%%%%%%
% Packages
%%%%%%%%%%%%%%%%%%%%%%%%%%%%%%%%%%%%%%%%%%%%%%%%%%%%%%%%%%%%%%%%%%%%%%%%
%\usepackage{beamerthemeKIT}
\usepackage[utf8]{inputenc}
% \usepackage[german]{babel}
\usepackage{helvet}
\usepackage[T1]{fontenc}
\usepackage{amsmath, amsthm, amssymb}
\usepackage{graphicx}
\usepackage{listings}
\usepackage{hyperref}
\usepackage{KITcolors}
%%%%%%%%%%%%%%%%%%%%%%%%%%%%%%%%%%%%%%%%%%%%%%%%%%%%%%%%%%%%%%%%%%%%%%%%
% Layout
%%%%%%%%%%%%%%%%%%%%%%%%%%%%%%%%%%%%%%%%%%%%%%%%%%%%%%%%%%%%%%%%%%%%%%%%
\renewcommand{\rmdefault}{phv}
\newlength{\movetitlegrafics}
\setlength{\movetitlegrafics}{0.0\paperwidth}
%%%%%%%%%%%%%%%%%%%%%%%%%%%%%%%%%%%%%%%%%%%%%%%%%%%%%%%%%%%%%%%%%%%%%%%%
% Definitions
%%%%%%%%%%%%%%%%%%%%%%%%%%%%%%%%%%%%%%%%%%%%%%%%%%%%%%%%%%%%%%%%%%%%%%%%
%%% Counter
\newcounter{ctlit}%
\setcounter{ctlit}{1}%
\newcounter{kap}

%%% Textfarben
\newcommand\BLUE[1]{\textcolor{KITblue}{#1}}%
\newcommand\BLACK[1]{\textcolor{KITblack}{#1}}%
\newcommand\GREEN[1]{\textcolor{KITgreen}{#1}}%
\newcommand\LBLUE[1]{\textcolor{KITblue15}{#1}}%
\newcommand\RED[1]{\textcolor{KITred}{#1}}%
%%% Layout
\def\cmt#1{\BLUE{#1}}%
\def\code#1{\texttt{#1}}%
\def\codeblock#1{\GREEN{\texttt{#1}}}%
\def\ftext#1{\textbf{{#1}}}%
\def\headline#1{\large\textbf{\GREEN{#1}}}%
\def\KITitem{\KITBullet[2mm]\ }%
\def\q{\quad}%
\def\qq{\qquad}%
\def\subtitle#1{{\small #1}}%
\def\tab{\hspace*{0.5cm}}%
%%% Standards
%\input{D:/TeX/formate/mystyle/defs}%
%\input{/home/ae01/tex/formate/mystyle/defs}%
%%% Text
%%% Symbols
\def\bs{$\backslash$}%
\def\hpm{\hphantom{$-$}}%
\def\meps{\epsilon}%
%%% code environment
\lstset{
  showstringspaces=false,
  numbers=none,
  keywordstyle=\color{KITgreen},  % coloring and formatting of keywords as public, class, import 
  commentstyle=\color{KITblue}\small\ttfamily, % Color of comments
  stringstyle=\color{KITred},
  breaklines=true
}

\lstdefinestyle{JAVA}
{
  language=Java,
  basicstyle=\ttfamily, % defines text formatting
  frame=tb  % print top and bottom lines, frame=single, L
}

\lstdefinestyle{JAVAsmall}
{
  language=Java,
  basicstyle=\small\ttfamily, % defines text formatting
  frame=tb  % print top and bottom lines, frame=single, L
}

\lstdefinestyle{JAVAlines}
{
  language=Java,
  numbers=left,
  numberstyle=\color{KITblack50}\ttfamily,
  numbersep=4pt, % distance between numbers and code1_1
  xleftmargin=4pt, % distance from frame to listing
  xrightmargin=4pt, % distance from frame to listing
  basicstyle=\ttfamily, % defines text formatting
  keywordstyle=\color{KITgreen},  % coloring and formatting of keywords as public, class, import 
  commentstyle=\color{KITblue}\ttfamily, % Color of comments
  frame=tb  % print top and bottom lines, frame=single, L
}

\lstdefinestyle{JAVAsmalllines}
{
  language=Java,
  numbers=left,
  numberstyle=\color{KITblack50}\ttfamily,
  numbersep=4pt, % distance between numbers and code1_1
  xleftmargin=4pt, % distance from frame to listing
  xrightmargin=4pt, % distance from frame to listing
  basicstyle=\small\ttfamily, % defines text formatting
  keywordstyle=\color{KITgreen},  % coloring and formatting of keywords as public, class, import 
  commentstyle=\color{KITblue}\ttfamily, % Color of comments
  frame=tb  % print top and bottom lines, frame=single, L
}


\lstdefinestyle{BASH}{
  basicstyle=\Large\ttfamily, % defines text formatting
  frame=none,
  language=bash
}


%%%%%%%%%%%%%%%%%%%%%%%%%%%%%%%%%%%%%%%%%%%%%%%%%%%%%%%%%%%%%%%%%%%%%%%%
% Titlepage
%%%%%%%%%%%%%%%%%%%%%%%%%%%%%%%%%%%%%%%%%%%%%%%%%%%%%%%%%%%%%%%%%%%%%%%%
\title[Institut f\"ur Angewandte und Numerische Mathematik]%
 {\fontsize{15}{15}\selectfont{}
  \"Ubung:
  \textit{Einstieg in die Informatik}\\[1.5mm]
  \textit{\phantom{\"Ubung:}\; und algorithmische Mathematik}\\[1.5mm]
  }%\hspace*{3cm}\normalsize{f\"ur Mathematiker}}
\author{\fontsize{9}{9}\selectfont{}
 Albert Mink\
  }
\institute[Institut f\"ur Angewandte und Numerische Mathematik]
 {\fontsize{6}{6}\selectfont{}%
  Institut f\"ur Angewandte und Numerische Mathematik}
\date{Wintersemester 2018/19}%
\subject{}%
%\beamerdefaultoverlayspecification{<+->}
%%%%%%%%%%%%%%%%%%%%%%%%%%%%%%%%%%%%%%%%%%%%%%%%%%%%%%%%%%%%%%%%%%%%%%%%


%%%%%%%%%%%%%%%%%%%%%%%%%%%%%%%%%%%%%%%%%%%%%%%%%%%%%%%%%%%%%%%%%%%%%%%%
% Document
%%%%%%%%%%%%%%%%%%%%%%%%%%%%%%%%%%%%%%%%%%%%%%%%%%%%%%%%%%%%%%%%%%%%%%%%
\begin{document}
\maketitle%
\addtocounter{framenumber}{-1}%
%%%%%%%%%%%%%%%%%%%%%%%%%%%%%%%%%%%%%%%%%%%%%%%%%%%%%%%%%%%%%%%%%%%%%%%%

\begin{frame}
  \frametitle{Probeklausur, Altklausur 2017}%
\tableofcontents
\end{frame}

%%%%%%%%%%%%%%%%%%%%%%%%%%%%%%%%%%%%%%%%%%%%%%%%%%%%%%%%%%%%%%%%%%%%%%%%
\def\stitle{Matrixpotenz}
\section{\stitle}\label{S:matrix}
\begin{frame}[t]%
  \frametitle{\ref{S:matrix} \stitle}
In der Vorlesung und der "Ubung haben Sie den Algorithmus von Pingala kennengelernt. Dieser erm"oglicht die
effiziente Berechnung einer Potenz $x^e, e\in\mathbb N, x\in\mathbb R$ oder $x\in\mathbb C$. Analog dazu
l"asst sich auch die Potenz $A^e, e\in\mathbb N$ einer Matrix $A\in\mathbb R^{n \times n}$ effizient
berechnen:
\begin{align}
 \label{eq_pingala}
 A^e =
  \begin{cases}
   I, & \text{falls } e=0,\\
   (A^2)^{\frac{e}{2}}, & \text{falls $e>0$ gerade},\\
   A\,A^{e-1}, & \text{falls $e>0$ ungerade}.
  \end{cases}
\end{align}
Hierbei bezeichnet $I\in\mathbb R^{n \times n}$ die Einheits-- bzw. Identit"atsmatrix. Die Diagonaleintr"age dieser
Matrix sind eins, w"ahrend alle anderen Eintr"age null sind, d.h.
\begin{align*}
 I_{i,j} =
 \begin{cases}
   1, &\text{falls } i=j,\\
   0, &\text{falls } i\neq j.\\
 \end{cases}
\end{align*}
Schreiben Sie ein kompilier-- und ausf"uhrbares Java--Programm, das die Potenz einer Matrix mithilfe des
Pingala--Algorithmus \eqref{eq_pingala} \textbf{rekursiv} berechnet. Gehen Sie wie folgt vor:\\[1em]

\end{frame}


%%%%%%%%%%%%%%%%%%%%%%%%%%%%%%%%%%%%%%%%%%%%%%%%%%%%%%%%%%%%%%%%%%%%%%%%
\begin{frame}
Erstellen Sie eine Klasse \code{Matrixpotenz} die folgende Elemente enth"alt:
\begin{itemize}
  \item Eine Klassenmethode \code{matrixEinlesen} ohne formale Parameter.\\
  Diese soll zun"achst die Anzahl der Zeilen und Spalten $n$ einer Matrix $A\in\mathbb R^{n \times n}$ von
  der Konsole einlesen. Anschlie\ss end sollen die Komponenten $A_{ij}$ der Matrix zeilenweise eingelesen
  werden. Geben Sie die Matrix als Variable vom Typ \code{double[][]} zur\"uck.
\end{itemize}
\end{frame}

%%%%%%%%%%%%%%%%%%%%%%%%%%%%%%%%%%%%%%%%%%%%%%%%%%%%%%%%%%%%%%%%%%%%%%%%
\begin{frame}
Schl\"usselw\"orter
\begin{itemize}
  \item Eine Klassenmethode \code{matrixEinlesen} \underline{ohne formalen Parameter}.\\
  Diese soll zun"achst die Anzahl der Zeilen und Spalten $n$ einer Matrix $A\in\mathbb R^{n \times n}$ von
  der Konsole einlesen. Anschlie\ss end sollen die Komponenten $A_{ij}$ der Matrix zeilenweise eingelesen
  werden. Geben Sie die Matrix als Variable vom Typ \underline{\code{double[][]} zur\"uck.}
\end{itemize}
\end{frame}


%%%%%%%%%%%%%%%%%%%%%%%%%%%%%%%%%%%%%%%%%%%%%%%%%%%%%%%%%%%%%%%%%%%%%%%%
\begin{frame}
Fortsetzung
\begin{itemize}
  \item Eine Klassenmethode \code{matrixMult} mit zwei \underline{formalen Parametern \code{matrixA} und \code{matrixB}
  vom Typ \code{double[][]}.}\\
  Hierbei bezeichnen \code{matrixA} und \code{matrixB} zwei "ubergebene Matrizen
  $A,B\in\mathbb R^{n\times n}$. Schreiben Sie einen Methodenrumpf, in dem das Matrixprodukt $AB$ berechnet
  wird und das Ergebnis in einer Variablen vom Typ \underline{\code{double[][]} zur"uck} gegeben wird.
\end{itemize}
\end{frame}

%%%%%%%%%%%%%%%%%%%%%%%%%%%%%%%%%%%%%%%%%%%%%%%%%%%%%%%%%%%%%%%%%%%%%%%%
\begin{frame}
Fortsetzung
\begin{itemize}
  \item Eine Klassenmethode \code{matrixPow} mit einem \underline{formalen Parameter \linebreak \code{matrixA} vom Typ \code{double[][]} }
  und einem \underline{ganzzahligen formalen Parameter \code{e}.}\\
  Hierbei bezeichnet \code{matrixA} eine Matrix $A\in\mathbb R^{n\times n}$
  und \code{e} einen Exponenten $e\in\mathbb N$. Berechnen Sie innerhalb des Methodenrumpfs die Potzenz $A^e$ der
  "ubergebenen Matrix $A$ nach dem Pingala--Algorithmus \eqref{eq_pingala}. Rufen Sie dazu die Klassenmethode
  \code{matrixPow} \textbf{rekursiv} auf und verwenden Sie die Klassenmethode \code{matrixMult}, um das Produkt zweier
  Matrizen zu berechnen. Geben Sie das Ergebnis in einer Variablen vom Typ \underline{\code{double[][]} zur"uck.}
\end{itemize}
\end{frame}

%%%%%%%%%%%%%%%%%%%%%%%%%%%%%%%%%%%%%%%%%%%%%%%%%%%%%%%%%%%%%%%%%%%%%%%%
\begin{frame}
Fortsetzung
\begin{itemize}
  \item Das Hauptprogramm.\\
  Lesen Sie zu Beginn eine Matrix $A\in\mathbb R^{n\times n}$ und einen Exponenten $e\in\mathbb N$ von der Konsole ein.
  Berechnen Sie anschlie\ss end die Matrixpotenz $A^e$. Verwenden dabei Sie die zuvor erstellten Klassenmethoden.
\end{itemize}
\end{frame}


%%%%%%%%%%%%%%%%%%%%%%%%%%%%%%%%%%%%%%%%%%%%%%%%%%%%%%%%%%%%%%%%%%%%%%%%
\def\stitle{Mobilfunk}
\section{\stitle}\label{S:mobil}
\begin{frame}[t]%
  \frametitle{\ref{S:mobil} \stitle}
Schreiben Sie ein kompilier-- und ausf"uhrbares Java--Programm, das ein einfaches Mobil\-funk\-netz
modelliert.


Verwenden Sie dazu die folgenden Klassen:
\begin{itemize}
\item
Eine Klasse \code{Teilnehmer}, die die einzelnen Teilnehmer
eines Betreibers simuliert. Diese k"onnen Textnachrichten
an andere Teilnehmer versenden sowie die zuletzt empfangene
Textnachricht im Speicher ihres Mobilfunkger"ates ablegen.
\item
Eine Klasse \code{Betreiber}, die den Betreiber
eines Mobilfunknetzes simuliert.
Der Betreiber verwaltet die Teilnehmer des Mobilfunknetzes.
\item
Eine Klasse \code{Mobilfunk}, die einen Markt mit einem
Mobilfunkbetreiber simuliert.
\end{itemize}

\end{frame}

%%%%%%%%%%%%%%%%%%%%%%%%%%%%%%%%%%%%%%%%%%%%%%%%%%%%%%%%%%%%%%%%%%%%%%%%
\begin{frame}

Schreiben Sie eine Klasse \code{Teilnehmer}, welche die folgenden
Elemente enth"alt:
\begin{itemize}
\item
Eine ganzzahlige Variable \code{nummer}
f"ur die Rufnummer des Teilnehmers und zwei Zeichenketten \code{name} und \code{speicher} vom Typ \code{String}
f"ur den Namen des Teilnehmers und den Inhalt der zuletzt
empfangenen Textnachricht. Au\ss erdem eine Variable \code{bt}
vom Typ \code{Betreiber}  f"ur den Betreiber des Mobilfunknetzes, welches der Teilnehmer nutzt. \\
Hierbei soll die Variable \code{nummer} als "offentliche Instanz--Variable deklariert werden,
w"ahrend alle anderen Variablen als private Instanz--Variablen deklariert werden.
\item
Einen "offentlichen Konstruktor mit zwei Parametern f"ur die
ganzzahlige Rufnummer des Teilnehmers und den Betreiber des
Mobilfunknetzes vom Typ \code{Betreiber}. Initialisieren Sie
die Instanz--Variablen mit den "ubergebenen Werten bzw.\ belegen
Sie diese mit sinnvollen Werten.
Lesen Sie dabei den Namen des Teilnehmers von der Konsole ein.
\end{itemize}
\end{frame}

%%%%%%%%%%%%%%%%%%%%%%%%%%%%%%%%%%%%%%%%%%%%%%%%%%%%%%%%%%%%%%%%%%%%%%%%
\begin{frame}
Fortsetzung:
\begin{itemize}
\item
Eine "offentliche Instanz--Methode \code{sendeNachricht}
mit zwei formalen Parametern f"ur die ganzzahlige Rufnummer des Empf"angers sowie
f"ur den Mobilfunkbetreiber des Empf"angers vom Typ \code{Betreiber}.
Lesen Sie dazu die Textnachricht von der Konsole ein. Geben Sie dabei
eine Meldung auf dem Bildschirm aus, aus der ersichtlich wird,
wer Sender und Empf"anger der Nachricht ist.
Speichern Sie die Textnachricht in der Komponente \code{speicher} des Empf"angers ab.

\emph{"Uber die Variable \code{bt} k"onnen Sie auf alle gespeicherten Teilnehmer \code{tn} des gleichen Betreibers zugreifen.}

\item
Eine "offentliche Instanz--Methode \code{toString} ohne formale Parameter mit einem
R"uckgabe--Parameter vom Typ \code{String} zur Ausgabe eines
Objektes vom Typ \code{Teilnehmer}.
Geben Sie dabei den Namen des Teilnehmers,
seine Rufnummer und die zuletzt empfangene
Textnachricht in einem \code{String} an die aufrufende Methode zur"uck.
\end{itemize}
\end{frame}

%%%%%%%%%%%%%%%%%%%%%%%%%%%%%%%%%%%%%%%%%%%%%%%%%%%%%%%%%%%%%%%%%%%%%%%%
\begin{frame}
Schreiben Sie eine weitere Klasse \code{Betreiber}, welche die folgenden
Elemente enth"alt:
\begin{itemize}
\item
Eine ganzzahlige Variable \code{anzahlTN} in der die Anzahl der Teilnehmer
des Betreibers gespeichert werden. Des Weiteren ein eindimensionales Feld
\code{tn} vom Typ \code{Teilnehmer[]},
in dem alle Teilnehmer des Betreibers gespeichert werden.
Deklarieren Sie diese Variablen als "offentliche Instanz--Variablen.
\item
Einen "offentlichen Konstruktor mit einem formalen Parameter f"ur die ganzzahlige Anzahl
der Teilnehmer des Betreibers, in dem die Instanz--Variablen mit
dem "ubergebenenen Wert sinnvoll initialisiert werden.
Iterieren Sie anschlie\ss end in einer Schleife "uber die Anzahl aller Teilnehmer und
legen Sie im $i$-ten Schleifendurchlauf den $i$-ten Teilnehmer
explizit durch Aufruf des entsprechenden Konstruktors an. Speichern Sie dabei die Instanzen
der Klasse \code{Teilnehmer} im Feld \code{tn} ab.
\emph{
Die Teilnehmer eines Betreibers werden,
beginnend mit der Zahl Null, fortlaufend durchnummeriert.
Die Nummer entspricht dabei gleichzeitig der Rufnummer
des Teilnehmers.}
\item
Eine "offentliche Instanz--Methode
\code{bericht} ohne formale Parameter, in der der aktuelle Status des Betreibers
und aller Teilnehmer auf der Konsole ausgegeben wird. Dieser besteht aus
der Anzahl der Teilnehmer und den Informationen jedes Teilnehmers, die mithilfe
der \code{toString} Methode der Klasse \code{Teilnehmer} ausgegeben werden k"onnen.
\end{itemize}

\end{frame}

%%%%%%%%%%%%%%%%%%%%%%%%%%%%%%%%%%%%%%%%%%%%%%%%%%%%%%%%%%%%%%%%%%%%%%%%
\begin{frame}
Fortsetzung:
Schreiben Sie eine Klasse \code{Mobilfunk}, welche nur das Hauptprogramm enth"alt:
\begin{itemize}
\item
Erstellen Sie zu Beginn einen Betreiber \code{fminus} mit drei Teilnehmern und f"uhren
Sie anschlie{\ss}end die folgenden Aktionen aus:
\begin{itemize}
\item[(1)]
Teilnehmer 1 sendet eine Textnachricht an Teilnehmer 2,
\item[(2)]
Teilnehmer 2 sendet eine Textnachricht an Teilnehmer 0.
\end{itemize}
Geben Sie im Anschlu{\ss} daran den aktuellen Status des Betreibers
mithilfe der Methode \code{bericht} auf dem Bildschirm aus.
\end{itemize}
\end{frame}



%%%%%%%%%%%%%%%%%%%%%%%%%%%%%%%%%%%%%%%%%%%%%%%%%%%%%%%%%%%%%%%%%%%%%%%%
\begin{frame}
\centering
\Huge\GREEN{Fragen?}
\end{frame}


%%%%%%%%%%%%%%%%%%%%%%%%%%%%%%%%%%%%%%%%%%%%%%%%%%%%%%%%%%%%%%%%%%%%%%%%
\end{document}
